This is a part of the code used to implement the SFA rate, where the calculations actually take place.
One can see the usage of the coefficients and the dipole transition elements.

\begin{lstlisting}[language=Python, basicstyle=\tiny]
    for state_idx in range(excitedStates):
        for state_range_idx in range(excitedStates):
            f0 = np.zeros((Tar.size, tar.size), dtype=np.cdouble)
            phase0 = np.zeros((Tar.size, tar.size), dtype=np.cdouble)
            cLeft = coefficients[state_idx, :]
            cRight = coefficients[state_range_idx, :]
            phaseleft = np.unwrap(np.angle(cLeft))
            phaseright = np.unwrap(np.angle(cRight))
            absleft = np.abs(cLeft)
            absright = np.abs(cRight)

            for i in prange(Tar.size):
                Ti=Ti_ar[i]
                for j in range(tar.size):
                    tj=N+nmin+j*n
                    tp=tj+Ti
                    tm=tj-Ti
                    if tp>=0 and tp<EF.size and tm>=0 and tm<EF.size:
                        VPt = 0
                        T= Ti*dT
                        DelA = (intA[tp] - intA[tm])-2*VPt*T
                        VP_p=VP[tp]-VPt
                        VP_m=VP[tm]-VPt
                        f_t_1= np.conjugate(transitionElement(config[state_idx], p, pz, VP_m, E_g))*transitionElement(config[state_range_idx], p, pz, VP_p, E_g)
                        G1_T_p=np.trapz(f_t_1*np.exp(1j*pz*DelA)*np.sin(theta), Theta_grid)
                        G1_T=np.trapz(G1_T_p*window*p_grid**2*np.exp(1j*p_grid**2*T), p_grid)
                        DelA = DelA + 2 * VPt * T
                        phase0[i, j]  = (intA2[tp] - intA2[tm])/2  + T*VPt**2-VPt*DelA +eigenEnergy[state_idx]*tp*dT-eigenEnergy[state_range_idx]*tm*dT -phaseleft[tm]+phaseright[tp]
                        f0[i, j] = EF[tp]*EF[tm]*G1_T*absleft[tm]*absright[tp]
            current_state_rate = 2*np.real(IOF(Tar, f0, (phase0)*1j))*4*np.pi
            rate += current_state_rate
    return rate
\end{lstlisting}

This is the main part where the tRecX coefficients are computed. It can be seen that in the \texttt{matrixElementUnscaled} method instead of computing $\braket{\Psi(t)\ket{\Psi_n}\bra{\Psi_n}\Psi(t)}=|c_n(t)|^2$ the left side of expectation has been modifyed to $\braket{\Psi_n\ket{\Psi_n}\bra{\Psi_n}\Psi(t)}=c_n(t)$.

\begin{lstlisting}[language=C++, basicstyle=\tiny]
static std::complex<double> eigenProjection(int IOp, std::vector<OperatorAbstract*> Ops, double Time, const Coefficients* Wf,bool Normalize, std::vector<std::shared_ptr<Coefficients>> Duals){
    Ops[IOp]->update(Time,Wf);
    if (IOp > 1 && !Duals.empty() && IOp - 2 < Duals.size()) {
        const Coefficients* constdual_raw = Duals[IOp - 2].get();
        complex<double> expec=Ops[IOp]->matrixElementUnscaled(*constdual_raw,*Wf);
        expec=Threads::sum(expec);
        if(Normalize){
            std::complex<double> nrm=1.;
            if(Ops[IOp]->name().find("Ovr(")==std::string::npos){
                for(auto o: Ops){
                    if(o->name().find("Ovr(")!=std::string::npos){
                        nrm=Threads::sum(o->matrixElementUnscaled(*constdual_raw,*Wf));
                        break;
                    }
                }
            }
            if(nrm!=0.)expec/=nrm;
        }
        return expec;
    }
\end{lstlisting}
    