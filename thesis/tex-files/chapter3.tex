The main reason for making these ionization models is because one want to avoid solveng the schroedinger equation numerically because its time consuming and contains numerical difficulties.
However, for comparing and verifying the models, one needs to solve it numerically.\\
Furthermore, numerically solving the TDSE is just good for ionization propabilities and not rates.\\
Especially for expanding an already existing model, it is important to compare and see if the changes of the model are going into the right direction.



%%%%%%%%%%%%%%%%%%%%%%%
\section{Numerical Methods}
To implement formula \ref{eq:smatrix} we need to solve the TDSE numerically in order to obtain the coefficients $c_n(t)$. 
For that I used two different tools (tRecX and ODE) each with its own advantages and disadvantages.
Further, there is this dilemma with ionization rates not being quantum mechaincal observables, so we need to have trustfull ionization propabilities to compare my results.


% Difference between length gauge and velocity gauge in numerics\\

\subsection{tRecX}
tRecX is a C++ code for solving generalized inhomogeneous time-dependent
Schroedinger-type equations in arbitrary dimensions and in a variety of coordinate systems \cite{Scrinzi_trecx}.
Usually ionization is difficult for an numerical solver because the electron "leaves" the subspace of the bound states, making the calculations very time consuming.
tRecX uses different techniques (irECS and tSURFF) making it especially suitable for light atom interaction. 
Further, tRecX allows to specify the gauge in which the TDSE is solved, making it more flexible. 
This will be important later.

\paragraph{irECS}
Lorem ipsum dolor sit amet, consetetur sadipscing elitr, sed diam nonumy eirmod tempor invidunt ut labore et dolore magna aliquyam erat, sed diam voluptua. At vero eos et accusam et justo duo dolores et ea rebum. Stet clita kasd gubergren, no sea takimata sanctus est Lorem ipsum dolor sit amet. Lorem ipsum dolor sit amet, consetetur sadipscing elitr, sed diam nonumy eirmod tempor invidunt ut labore et dolore magna aliquyam erat, sed diam voluptua. At vero eos et accusam et justo duo dolores et ea rebum. Stet clita kasd gubergren, no sea takimata sanctus est Lorem ipsum dolor sit amet.
\paragraph{tSURFF}
Lorem ipsum dolor sit amet, consetetur sadipscing elitr, sed diam nonumy eirmod tempor invidunt ut labore et dolore magna aliquyam erat, sed diam voluptua. At vero eos et accusam et justo duo dolores et ea rebum. Stet clita kasd gubergren, no sea takimata sanctus est Lorem ipsum dolor sit amet. Lorem ipsum dolor sit amet, consetetur sadipscing elitr, sed diam nonumy eirmod tempor invidunt ut labore et dolore magna aliquyam erat, sed diam voluptua. At vero eos et accusam et justo duo dolores et ea rebum. Stet clita kasd gubergren, no sea takimata sanctus est Lorem ipsum dolor sit amet.



% \paragraph{Challenges}
% dangling pointer: interesting problem actually, how to solve it, how to find it, etc




\subsection{ODE/Python}





%%%%%%%%%%%%%%%%%%%%%%%%
\section{TIPTOE}
This follows \cite{Park:18} and manorams master thesis.\\\\
\paragraph{Why do we need TIPTOE?} 
In order to verify the ionization rates determined by the models, we need to compare it with something that is experimentally accessible.
However an ionization rate is not a quantum mechanical observable, we can only measure the ionization yield so $\int \Gamma(t) \dd t$.

In some way, TIPTOE motivates this thesis, since TIPTOE's results showed us the break in time reversal symmetrie mentioned earlier.
But what is TIPTOE? TIPTOE \cite{Park:18} is short for tunneling ionization with a perturbation for the time-domain observation of an electric field. 
It is used for direct sampling an electric pulse in the femtosecond to attosecond regime using quasistatic subcycle tunneling ionization in a gaseous medium or air.
Tunneling ionization differes from multiphoton ionization in the sense that the laser field is so strong that it deforms the coulomb barrier of the atom, allowing the electron to tunnel through it.
However, our motivation to establish an instantanious ionization rate remains promising, especially because in the models that describe tunneling ionization, the ionization rate only depends on the field strength of the laser pulse.
% 16. M. V. Ammosov, N. B. Delone, and V. P. Krainov, “Tunnel ionization of
% complex atoms and of atomic ions in an alternating electromagnetic
% field,” Sov. Phys. JETP 64, 1191–1194 (1986).
% 17. L. V. Keldysh, “Ionization in the field of a strong electromagnetic wave,”
% Sov. Phys. JETP 20, 1307 (1965).
A typical TIPTOE simulation consist of two linear polarised lase pulses, a "fundamental" and a "signal" pulse, similar to common pump-probe experiments. 
The drive pulse is the pulse we want to sample at the end just by measuren the ionization yield for a certain medium.
In the first order we can approximate the ionization rate as
\begin{equation*}
    \Gamma(E_{\mathrm{F}}+E_{\mathrm{S}})=\Gamma(E_{\mathrm{F}})+\left.E_{\mathrm{S}}\frac{\dd \Gamma(E_{\mathrm{S}})}{\dd E}\right|_{E=E_{\mathrm{F}}}
\end{equation*}
However in this approximation we are going to neglect the depletion of the ground state. 
Then the total ionisation yield $N$ obtained by the two pulses is given by
\begin{equation*}
    N=N_0+\delta N = \int \dd t\,\Gamma(E_{\mathrm{F}}(t))+\int \dd t\,E_{\mathrm{S}(t)}\left.\frac{\dd \Gamma(E_{\mathrm{S}}(t))}{\dd E}\right|_{E=E_{\mathrm{F}}(t)}
\end{equation*}
By changing the delay $\tau$ between the two pulses, the ionization yield of course different values.
With this we can find that:
\begin{equation*}
    \delta N(\tau)=E_{\mathrm{S}}(\tau)
\end{equation*}
so we can sample the field amplitude of the signal pulse by measuring the ionization yield for different delays.
The TIPTOE method can be applied for the broad spectral range of the signal pulse, as long as the fundamental pulse is shorter than $1.5$ optical cycles.\\
This method is very usefull because the problem of numerical simulations giving good predicitons about ionization propabilities and analytical models about ionization rates.
Now we have a method to compare both ways and egt insight in the dynamics of the electron. 
We will later integrate the ionization rate of $E_{\mathrm{F}}+E_{\mathrm{S}}$ over the full time domain and compare the ionization yield for different delay with the resluts from the numerical solution of the TDSE.







%%%%%%%%%%%%%%%%%%
\section{GASFIR}
GASFIR is short for general approximator for strong field ionization rates.
It is an analytical retrieval tool for reconstructing data obtaned from numerical solutions of the TDSE. 
It was validated for hydrogen and shows good agreement with existing theories in the quasi static limit of tunneling ionization not only for Hydrogen, also for Helim and Neon [cite GASFIR].
The way how GASFIR works is that it uses ionization probabilities to retrieve ionization rates. 
It uses the idea from SFA that the rates can be written as $\int \dd T K(t,T)$ with $K(t,T)$ being a kernel function.
Later in the code you see also the kernel function, where the modifications happened. 
Its not really part of this thesis, but motivates it.
Goal: make GASFIRs predictions better by improving SFA rates




% \begin{align*}
%     K(t,T) &= E_\mathrm{+}E_\mathrm{-} \int_0^{\infty} dp\, p^2 \int_0^\pi d\theta \sin\theta e^{ i T (p^2+2\overline{\Delta A} p \cos{\theta})} \\
%            &\quad \times \int_0^{2\pi}d\phi \,d_z^*\bigl(\bm{p} + \bm{e}_z A_\mathrm{+}\bigr) d_z\bigl(\bm{p} + \bm{e}_z A_\mathrm{-}\bigr) e^{ i T (2I_\mathrm{p} + \overline{\Delta A^2})}.
% \end{align*}
