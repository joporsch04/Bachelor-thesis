This chapter introduces the methods used throughout the thesis including numerical methods for solving or implementing certain types of equations, as well as physical methods used for result comparison or data interpretation.


To compare and verify the ionization rate from Chapter 2, the TDSE must be solved numerically without approximations.
A reliable approach for comparison involves the use of the sampling method TIPTOE \cite{Park:18}, where the ionization yield from both are methods being compared.
When expanding an existing model, it is particularly important to perform such comparisons to assess whether modifications align with expected outcomes.

%%%%%%%%%%%%%%%%%%%%%%%
\section{Numerical Methods}
To implement formula \ref{eq:smatrix}, the TDSE must be solved numerically in two different ways to obtain the coefficients $c_n(t)$.  
Two methods were used (numerical solver and ODE), each with its own advantages and disadvantages.  

For numerically solving the TDSE without approximations, a solver called tRecX was used.  


\subsection{tRecX}
%\paragraph{\textcolor{red}{Why is tRecX special? More motivation:}}
tRecX is a C++ code designed for solving generalized inhomogeneous time-dependent Schrödinger-type equations in arbitrary dimensions and in a variety of coordinate systems \cite{Scrinzi_trecx}.
Ionization is typically challenging for numerical solvers because the electron leaves the subspace of bound states, making calculations computationally expensive.
Additionally, the Coulomb potential decays only as $1/r$, raising the question of how to discretize the space and set boundary conditions.

tRecX employs various techniques that make it particularly suitable for light-atom interactions.
For example, the method called ``infinite-range exterior complex scaling'' applies a unitary transformation to complex-scale space beyond a certain distance from the nucleus.
This can be likened to the analytic continuation of a function into the complex plane.
Furthermore, the wavefunction is `damped' by the complex-scaled region, preventing reflections at the boundary that could lead to numerical issues \cite{scrinzi_irecs}.

Moreover, tRecX allows the gauge (\eqref{eq:dipoleApprox}, \eqref{eq:dipoleApprox_velocity}) in which the TDSE is solved to be specified, increasing its flexibility.
This feature will be important later.

\medskip
Throughout this thesis, tRecX was employed for two purposes.

First, it was used to solve the TDSE in the entire Hilbert space to obtain the coefficients $c_n(t)$, which are required for the improved SFA rate.
For this purpose, the source code was modified to calculate the coefficients and store them in an appropriate format.
Further details regarding the implementation can be found in Chapter 4.

Second, tRecX served as a reference within the TIPTOE experiment to evaluate the performance of the SFA rate with excited states \eqref{eq:sfa_rate_improved}.


% \paragraph{\textcolor{red}{irECS}}
% Lorem ipsum dolor sit amet, consetetur sadipscing elitr, sed diam nonumy eirmod tempor invidunt ut labore et dolore magna aliquyam erat, sed diam voluptua. At vero eos et accusam et justo duo dolores et ea rebum. Stet clita kasd gubergren, no sea takimata sanctus est Lorem ipsum dolor sit amet. Lorem ipsum dolor sit amet, consetetur sadipscing elitr, sed diam nonumy eirmod tempor invidunt ut labore et dolore magna aliquyam erat, sed diam voluptua. At vero eos et accusam et justo duo dolores et ea rebum. Stet clita kasd gubergren, no sea takimata sanctus est Lorem ipsum dolor sit amet.
% \paragraph{\textcolor{red}{tSURFF}}
% Lorem ipsum dolor sit amet, consetetur sadipscing elitr, sed diam nonumy eirmod tempor invidunt ut labore et dolore magna aliquyam erat, sed diam voluptua. At vero eos et accusam et justo duo dolores et ea rebum. Stet clita kasd gubergren, no sea takimata sanctus est Lorem ipsum dolor sit amet. Lorem ipsum dolor sit amet, consetetur sadipscing elitr, sed diam nonumy eirmod tempor invidunt ut labore et dolore magna aliquyam erat, sed diam voluptua. At vero eos et accusam et justo duo dolores et ea rebum. Stet clita kasd gubergren, no sea takimata sanctus est Lorem ipsum dolor sit amet.



% \paragraph{Challenges}
% dangling pointer: interesting problem actually, how to solve it, how to find it, etc




\subsection{System of ODEs}
The coefficients of the wavefunction within the bound-state subspace were obtained using the interaction picture, and the resulting coupled system of differential equations was solved numerically.
The calculation was implemented in Python, utilizing the ODE solver \texttt{solve\_ivp} from SciPy’s \texttt{integrate} module.
The corresponding code is provided in the electronic appendix within the \texttt{HydrogenSolver} class.
Further details on the implementation can be found in Chapter 4.




%%%%%%%%%%%%%%%%%%%%%%%%%
\section{TIPTOE}
This section mostly follows \cite{Park:18} and \cite{manorammasterthesis}.

TIPTOE (Tunneling Ionization with a Perturbation for the Time-domain Observation of an Electric field) is a method for direct sampling of an electric pulse in the femtosecond to attosecond regime using quasistatic subcycle tunneling ionization in a gaseous medium or air.

A typical TIPTOE simulation consists of two linearly polarized laser pulses: a ``fundamental'' and a ``signal'' pulse, similar to common pump-probe experiments.
The drive pulse is the pulse to be sampled, with the ionization yield of a certain medium providing the measurement.
In first order, the ionization rate can be approximated as
\begin{equation}
    \Gamma(E_{\mathrm{F}}+E_{\mathrm{S}})\approx\Gamma(E_{\mathrm{F}})+\left.E_{\mathrm{S}}\frac{\dd \Gamma(E_{\mathrm{S}})}{\dd E}\right|_{E=E_{\mathrm{F}}}
\end{equation}
In this approximation, depletion of the ground state is neglected.
The total ionization yield $N_{\mathrm{total}}$ obtained by the two pulses is given by
\begin{equation*}
    N_{\mathrm{total}}=N_{\mathrm{F}}+\delta N = \int \dd t\,\Gamma(E_{\mathrm{F}}(t))+\int \dd t\,E_{\mathrm{S}}(t)\left.\frac{\dd \Gamma(E_{\mathrm{S}}(t))}{\dd E}\right|_{E=E_{\mathrm{F}}(t)}
\end{equation*}
By varying the delay $\tau$ between the two pulses, the ionization yield takes on different values.
From this, it follows that:
\begin{equation}
    \delta N(\tau)\propto E_{\mathrm{S}}(\tau) \label{eq:tiptoeprop}
\end{equation}
Thus, the field amplitude of the signal pulse can be sampled by measuring the ionization yield for different delays.
The TIPTOE method can be applied across a broad spectral range of the signal pulse, as long as the fundamental pulse is shorter than $1.5$ optical cycles.

\bigskip
This method provides a way to compare ionization dynamics predicted by a numerical solver and by the SFA rate \eqref{eq:sfa_rate_improved}.
TIPTOE is particularly useful because numerical simulations can provide good predictions about ionization probabilities, while analytical models describe ionization rates.
A TIPTOE simulation can help reconstruct the ionization dynamics, which is especially relevant in the context of this thesis.
Later, the ionization rate $\Gamma(E_{\mathrm{F}}+E_{\mathrm{S}})$ will be integrated over the full time domain, and the ionization yield for different delays will be compared with the results from the numerical solution of the TDSE.
The results are shown in chapter 5.

For better visualization of the underlying physics in TIPTOE, the background ionization $N_{\mathrm{F}}$ from $E_\mathrm{F}(t)$ is subtracted, and the ionization yield is normalized, so the formula in plot \ref{fig:tiptoe_sfa_comparison} reads
\begin{equation}
    \frac{N_{\mathrm{total}}-N_{\mathrm{F}}}{N_{\mathrm{max}}}=\frac{\delta N(\tau)}{N_{\mathrm{max}}}      \label{eq:tiptoeprop_normalized}
\end{equation}
However, interesting physics can also be observed by comparing only the total ionization yield, as discussed later in Chapter 5.

Typically, TIPTOE is not used for this kind of analysis but rather for its sampling capabilities.
Instantaneous ionization rates are highly useful because TIPTOE enables sampling of the electric field of a laser pulse in the femtosecond to attosecond regime, which has broad applications in fields such as laser spectroscopy and medical physics.






%%%%%%%%%%%%%%%%%%
\section{GASFIR}
GASFIR stands for General Approximator for Strong Field Ionization Rates.
It is an analytical retrieval tool designed to reconstruct data obtained from numerical solutions of the TDSE.
The model was validated for hydrogen and shows good agreement with existing theories in the quasi-static limit of tunneling ionization, not only for hydrogen but also for helium and neon \cite{agarwal2025generalapproximatorstrongfieldionization}.

The working principle of GASFIR is based on the use of ionization probabilities to retrieve ionization rates.
The approach employs the idea from SFA that the rates can be expressed as $\int \dd T K(t,T)$, where $K(t,T)$ is a kernel function.
Later in the code, the kernel function is also visible, with the modifications applied there.

GASFIR is not part of this thesis; however, this thesis was partially motivated by the idea that an improvement of the SFA formalism could benefit the GASFIR approach.


% \begin{align*}
%     K(t,T) &= E_\mathrm{+}E_\mathrm{-} \int_0^{\infty} dp\, p^2 \int_0^\pi d\theta \sin\theta e^{ i T (p^2+2\overline{\Delta A} p \cos{\theta})} \\
%            &\quad \times \int_0^{2\pi}d\phi \,d_z^*\bigl(\bm{p} + \bm{e}_z A_\mathrm{+}\bigr) d_z\bigl(\bm{p} + \bm{e}_z A_\mathrm{-}\bigr) e^{ i T (2I_\mathrm{p} + \overline{\Delta A^2})}.
% \end{align*}
