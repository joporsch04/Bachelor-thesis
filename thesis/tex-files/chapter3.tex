This chapter is about introducing the methods used throughout the thesis.
This includes numerical methods for solving or implementing some type of equation as well as more physical methods that are used for comparing results or get more meaning to the data.

For comparing and verifying the ionization rate from chapter 2, one needs to solve the TDSE numerically without any approximations.
A reliable choice for comparing is by using the sampling method called TIPTOE \cite{Park:18} and comparing the ionization yield from both approaches.
Especially for expanding an already existing model, it is important to compare and see if the changes of the model are going into the right direction.



%%%%%%%%%%%%%%%%%%%%%%%
\section{Numerical Methods}
To implement formula \ref{eq:smatrix} we need to solve the TDSE in two different ways numerically in order to obtain the coefficients $c_n(t)$. 
For that I used two different methods (numerical/direct and ODE) each with its own advantages and disadvantages.

For numerically solving the TDSE without any approximations I used a solver called tRecX.




\subsection{tRecX}
tRecX is a C++ code for solving generalized inhomogeneous time-dependent
Schroedinger-type equations in arbitrary dimensions and in a variety of coordinate systems \cite{Scrinzi_trecx}.
Usually ionization is difficult for an numerical solver because the electron "leaves" the subspace of the bound states, making the calculations very time consuming.
Further, the coulomb potential only decays as $1/r$ so there is the question of how to discretize the space and set the boundary conditions.

tRecX uses different techniques making it especially suitable for light atom interaction. 
For instance, with the method called "Infinite-range exterior complex scaling" after a certan distance from the nucleus the space gets complex scaled via a unitary transformation.
One could imagine this the same way as an function can be analytically continued into the complex plane.
Further the wavefunction gets also "damped" by the complex scaled region so there is reflection at the boundary which could cause numerical problems \cite{scrinzi_irecs}. 

Further, tRecX allows to specify the gauge in which the TDSE is solved, making it more flexible. 
This will be important later. 

\medskip
Throughout this thesis, tRecX was used for two things. 
First, the for solving the TDSE in the whole hilbertspace to obtain the coefficients $c_n(t)$ that are needed for the imporved SFA rate.
For this, the sourcecode had to be modified to calculate the coefficients and store them in an appropiate format.
For more details about the implementation, see chapter 4.

Second, as a reference within the TIPTOE experiment to evaluate the performance of the SFA rate with excited states \eqref{eq:sfa_rate_improved}.




% \paragraph{\textcolor{red}{irECS}}
% Lorem ipsum dolor sit amet, consetetur sadipscing elitr, sed diam nonumy eirmod tempor invidunt ut labore et dolore magna aliquyam erat, sed diam voluptua. At vero eos et accusam et justo duo dolores et ea rebum. Stet clita kasd gubergren, no sea takimata sanctus est Lorem ipsum dolor sit amet. Lorem ipsum dolor sit amet, consetetur sadipscing elitr, sed diam nonumy eirmod tempor invidunt ut labore et dolore magna aliquyam erat, sed diam voluptua. At vero eos et accusam et justo duo dolores et ea rebum. Stet clita kasd gubergren, no sea takimata sanctus est Lorem ipsum dolor sit amet.
% \paragraph{\textcolor{red}{tSURFF}}
% Lorem ipsum dolor sit amet, consetetur sadipscing elitr, sed diam nonumy eirmod tempor invidunt ut labore et dolore magna aliquyam erat, sed diam voluptua. At vero eos et accusam et justo duo dolores et ea rebum. Stet clita kasd gubergren, no sea takimata sanctus est Lorem ipsum dolor sit amet. Lorem ipsum dolor sit amet, consetetur sadipscing elitr, sed diam nonumy eirmod tempor invidunt ut labore et dolore magna aliquyam erat, sed diam voluptua. At vero eos et accusam et justo duo dolores et ea rebum. Stet clita kasd gubergren, no sea takimata sanctus est Lorem ipsum dolor sit amet.



% \paragraph{Challenges}
% dangling pointer: interesting problem actually, how to solve it, how to find it, etc




\subsection{ODE}
To acess the coefficients of the wavefunction inside the subspace of the bound states I used the interaction picture and solved the resulting coupled system of differential equations. 
To implement the calculation, Python and the ODE solver from SciPy integrate called \texttt{solve\_ivp} was used.
The code can be found in the electronic appendix in the class \texttt{HydrogenSolver}.
For more details about the implementation, see chapter 4.






%%%%%%%%%%%%%%%%%%%%%%%%%
\section{TIPTOE}
This section mostly follows \cite{Park:18} and \cite{manorammasterthesis}.

TIPTOE (Tunneling Ionization with a Perturbation for the Time-domain Observation of an Electric field) is a method for direct sampling of an electric pulse in the femtosecond to attosecond regime using quasistatic subcycle tunneling ionization in a gaseous medium or air.

A typical TIPTOE simulation consists of two linearly polarized laser pulses: a "fundamental" and a "signal" pulse, similar to common pump-probe experiments.
The drive pulse is the pulse to be sampled, with the ionization yield of a certain medium providing the measurement.
In first order, the ionization rate can be approximated as
\begin{equation}
\Gamma(E_{\mathrm{F}}+E_{\mathrm{S}})\approx\Gamma(E_{\mathrm{F}})+\left.E_{\mathrm{S}}\frac{\dd \Gamma(E_{\mathrm{S}})}{\dd E}\right|_{E=E_{\mathrm{F}}}
\end{equation}
In this approximation, depletion of the ground state is neglected.
The total ionization yield $N$ obtained by the two pulses is given by
\begin{equation*}
N_{\mathrm{total}}=N_0+\delta N = \int \dd t,\Gamma(E_{\mathrm{F}}(t))+\int \dd t,E_{\mathrm{S}}(t)\left.\frac{\dd \Gamma(E_{\mathrm{S}}(t))}{\dd E}\right|_{E=E_{\mathrm{F}}(t)}
\end{equation*}
By varying the delay $\tau$ between the two pulses, the ionization yield takes on different values.
From this, it follows that:
\begin{equation}
\delta N(\tau)\propto E_{\mathrm{S}}(\tau) \label{eq:tiptoeprop}
\end{equation}
Thus, the field amplitude of the signal pulse can be sampled by measuring the ionization yield for different delays.
The TIPTOE method can be applied across a broad spectral range of the signal pulse, as long as the fundamental pulse is shorter than $1.5$ optical cycles.

\bigskip
This method provides a way to compare ionization dynamics predicted by a numerical solver and by the SFA rate \eqref{eq:sfa_rate_improved}.
TIPTOE is particularly useful because numerical simulations can provide good predictions about ionization probabilities, while analytical models describe ionization rates.
A TIPTOE simulation can help reconstruct the ionization dynamics, which is especially relevant in the context of this thesis.
Later, the ionization rate $\Gamma(E_{\mathrm{F}}+E_{\mathrm{S}})$ will be integrated over the full time domain, and the ionization yield for different delays will be compared with the results from the numerical solution of the TDSE.
The results are shown in chapter 5.

For better visualization of the underlying physics in TIPTOE, the "background" ionization from $E_\mathrm{F}(t)$ is subtracted, and the ionization yield is normalized, so the formula in plot ???? reads
\begin{equation*}
\frac{N_{\mathrm{total}}-N_0}{N_{\mathrm{max}}}=\frac{\delta N(\tau)}{N_{\mathrm{max}}}
\end{equation*}
However, interesting physics can also be observed by comparing the net ionization yield $N_{\mathrm{total}}$, as discussed later in Chapter 5.

Typically, TIPTOE is not used for this kind of analysis but rather for its sampling capabilities.
Instantaneous ionization rates are highly useful because TIPTOE enables sampling of the electric field of a laser pulse in the femtosecond to attosecond regime, which has broad applications in fields such as laser spectroscopy and medical physics.






%%%%%%%%%%%%%%%%%%
\section{GASFIR}
GASFIR is short for general approximator for strong field ionization rates.
It is an analytical retrieval tool for reconstructing data obtaned from numerical solutions of the TDSE. 
It was validated for hydrogen and shows good agreement with existing theories in the quasi static limit of tunneling ionization not only for Hydrogen, also for Helim and Neon \cite{agarwal2025generalapproximatorstrongfieldionization}.
The way how GASFIR works is that it uses ionization probabilities to retrieve ionization rates. 
It uses the idea from SFA that the rates can be written as $\int \dd T K(t,T)$ with $K(t,T)$ being a kernel function.
Later in the code you see also the kernel function, where the modifications happened. 
GASFIR is not a part of this thesis, it was motivated by the idea that an improvement of the SFA formalism could benefit the GASFIR approach.





% \begin{align*}
%     K(t,T) &= E_\mathrm{+}E_\mathrm{-} \int_0^{\infty} dp\, p^2 \int_0^\pi d\theta \sin\theta e^{ i T (p^2+2\overline{\Delta A} p \cos{\theta})} \\
%            &\quad \times \int_0^{2\pi}d\phi \,d_z^*\bigl(\bm{p} + \bm{e}_z A_\mathrm{+}\bigr) d_z\bigl(\bm{p} + \bm{e}_z A_\mathrm{-}\bigr) e^{ i T (2I_\mathrm{p} + \overline{\Delta A^2})}.
% \end{align*}
