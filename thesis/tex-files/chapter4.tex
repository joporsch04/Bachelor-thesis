This Chapter describes the implementation of the formula \eqref{eq:sfa_rate_improved} and the associated challenges.
The majority of the code was originally developed by the authors of \cite{Theory_NPS}.
Modifications made to both the tRecX source code and the SFA rate are documented in the electronic appendix \cite{johannes_porsch_2025_16223179}.
The two main new components in this implementation are the coefficients $c_n(t)$ and the dipole matrix elements $d_{z,n}(\vec{p})$, while the remaining parts were straightforward.

%%%%%%%%%%%%%%%%
\section{Coefficients}
As mentioned in Chapter 2, the Dyson equation can be written in two ways, resulting in two different expressions for the strong field s-matrix.
The difference between the two approaches lies only in the coefficients.



\subsection{Subspace Ansatz}
To understand how the coefficients are determined, the defining equation can be examined:
\begin{equation*}
    \hat{\U}^{\mathrm{Sub}}_0(t',t_0)\ket{\Psi_0(t_0)} = \sum_{n}c_n(t')e^{-iE_nt'}\ket{\Psi_n}
\end{equation*}
with
\begin{equation}
    i\frac{\partial}{\partial t'} \, \hat{\U}^{\mathrm{Sub}}_0(t',t_0) = \hat{X}\hat{\Hs}(t')\hat{X}\,\hat{\U}^{\mathrm{Sub}}_0(t',t_0)    \label{eq:odeU}
\end{equation}
This can be interpreted as the wavefunction being restricted to the subspace.
The derivation for the coefficients is outlined below.

First, $\hat{\Hs}(t)$ is split into two parts: $\hat{\Hs}_0$ and $\hat{\Hs}_{\mathrm{I}}(t)$, where the eigenstates of $\hat{\Hs}_0$ are ${\ket{\Psi_n}}$ and the eigenenergies are ${E_n}$. 
By substituting the ansatz above into \eqref{eq:odeU} and multiplying with $\bra{\Psi_m}$, the coefficients $c_n(t)$ are obtained:
\begin{equation*}
    i  \dot{c}_m(t) = \sum_n c_n(t) e^{-i \omega_{nm} t} \braket{\Psi_m|\hat{\Hs}_{\mathrm{I}}(t)|\Psi_n}  \label{eq:ode}
\end{equation*}
with $\omega_{nm} = E_n - E_m$.

\medskip
Several challenges arise when implementing this approach.
First, while $\sum_n$ theoretically extends from zero to infinity, numerical simulations require a finite number of states.
The justification for the chosen number of states must be considered, along with ensuring that truncating the sum does not introduce numerical instabilities.

Additionally, increasing the number of coefficients affects previously calculated values.
For example, solving only for the ground state $c_0(t)$ results in $|c_0(t)|^2 = 1$.
However, the ground-state occupation decreases as more states are included in the sum, illustrating the \emph{coupled} nature of the system of ODEs.

As the number of states increases, the results (e.g., the rates) should converge.
However, since the electron is constrained to the subspace, including highly excited states may introduce numerical instabilities.
For intensities around $10^{14}\frac{\mathrm{W}}{\mathrm{cm}^2}$ and longer wavelengths ($800\mathrm{nm}-1200\mathrm{nm}$), allowing occupation of higher states led to unphysical oscillations in the rates.
Under typical conditions, the electron would ionize, but since ionization is suppressed in this approach, excitation to high-energy states can cause the model to fail.
Thus, a balance must be made between accuracy and numerical stability.
In the simulations presented later, a maximum of $n=3$ was used for the bound state calculations.\footnote{Higher values of $n$ were tested for lower wavelengths and intensities, but no significant changes in the results were observed.}
This choice is justified by analyzing the population dynamics of the hydrogen atom over time, where most pre-ionization dynamics are governed by the first few bound states—particularly $1s$, $2s$, $2p$, and $3p$—meaning that including additional states has minimal impact on the lower-state coefficients.

Note: This discussion pertains not to the number of bound states included in the final simulations, but rather to the number considered when solving the ODE.
While only the ground state could have been used in the SFA rate calculations, the influence of higher bound states on the ground state must still be assessed.
This justification is provided here.




\medskip
Further, equation \eqref{eq:ode} is gauge dependent because of $\braket{m|\hat{\Hs}_{\mathrm{I}}(t)|n}$.
Either the length gauge \eqref{eq:dipoleApprox} or the velocity gauge \eqref{eq:dipoleApprox_velocity} can be chosen.
Which gauge should be used to obtain the most meaningful results?

To address this question, the scenario can be reconsidered.
An electron initially resides in the ground state, and before ionization, its behavior is primarily governed by the Coulomb potential.
After ionization, according to the SFA, it is described as a plane wave oscillating in the laser field.

From this perspective, both gauges can be useful.
During the first part of the process—when the electron remains bound to the hydrogen atom—the length gauge is more appropriate, as the electron's behavior is better described in terms of its position rather than its momentum.
This is why, in most elementary introductions to light-matter interaction (such as discussions of Rabi oscillations), the length gauge is preferred for systems where ionization does not occur.
Conversely, after ionization, the velocity gauge becomes more suitable, as the electron is no longer influenced by the potential (due to the SFA) and is fully characterized by its momentum.

Thus, the length gauge is a fitting choice for the coefficients used in the ODE ansatz.





% \bigskip
% In my implementation I neglected transitions to states that are forbidden via the dipole selection rules. 
% However this is an approximation, since in reality two-photon processes can occur, effectively allowing transitions between $1s$ and $2s$ for instance. 
% I numerically solved the schroedinger equation with tRecX and modyfied the code to print out the coefficients $c_n(t)$ allowing me to get insight in the `real' dynamics of the electron.
% the dont differ???? why??? i thoguht im negecting that in ODE c_ns but it seems not, |2,0,0> is still there, why?
% i think im not neglecting these transitions, its the normal dynamic of the electron without neglecting 1s->2s transitions
% its jsut that the angular integral is 0 but thats just mathematics




\subsection{Full Hilbertspace}
The coefficients from the subspace are not the only ones used. A numerical solver (in this case tRecX) was employed to solve the entire TDSE, and the coefficients were extracted from the wavefunction.

As mentioned earlier, there are two ways to consider $\hat{\U}(t',t_0)\ket{\Psi}$.
First, the TDSE can be solved in the subspace of the bound states (as done with the ODE), or the TDSE can be solved in the full Hilbert space and then projected onto bound states.
tRecX implements the second approach, which is significantly more complex than the ODE method.
Since far more physical effects influence the time evolution of the coefficients—and consequently the ionization rate—it becomes more challenging to isolate individual effects and their consequences.
Nevertheless, having two independent methods that, in some sense, compute the same quantity is useful, allowing for interpretation based on their respective advantages and disadvantages.

For extracting the coefficients from tRecX—i.e., solving the full TDSE and projecting onto bound states—the source code was modified as follows:
The occupation probability of specified bound states was already implemented, meaning the code output \footnote{The notation used here follows the same convention as in the code for better reference.}  $\braket{\Psi(t)|\hat{P}_{\mathrm{Occ\{H0:n\}}}|\Psi(t)}=|\braket{\Psi(t)|n}|^2$.
The main modification involved implementing a new function that replaces the left bra $\bra{\Psi(t)}$ with the eigenstate used in $\hat{P}_{\mathrm{Occ\{H0:n\}}}$, while ensuring the eigenstates were normalized.
This results in:
\begin{equation*}
    \braket{\Psi(t)|\hat{P}_{\mathrm{Occ\{H0:n\}}}|\Psi(t)} \rightarrow \braket{n|\hat{P}_{\mathrm{Occ\{H0:n\}}}|\Psi(t)} = \braket{n|n}\times\braket{n|\Psi(t)} = c_n(t)
\end{equation*}
To achieve this, the eigenvalue problem had to be solved again, and the eigenstates were passed to the function calculating the expectation value.
A more elegant and efficient implementation could likely be devised, especially since the eigenstates are already computed elsewhere, but the current approach suffices for now.
A detailed description of the changes made to tRecX can be found in \cite{johannes_porsch_2025_16223179}.

\bigskip
In principle, this is all that is required.
However, simply using the coefficients without further scrutiny would be insufficient.
Several issues can arise, particularly when extracting values from code not originally designed for this purpose.
This is especially relevant when computing quantities that are not experimentally accessible or are gauge-dependent, as such results would typically be of limited utility.
Thus, caution is necessary.

\paragraph{Gauge dependence}
The coefficients are gauge-dependent, similar to the subspace approach using the ODE.
However, in tRecX, a `hybrid gauge' is employed.
This means that within a certain radius $R_g$ from the nucleus, length gauge is used, while outside this radius, velocity gauge is applied.
Length gauge is better suited for pre-ionization dynamics, as the electron's behavior is more accurately described by its position rather than its momentum.
The gauge radius must be chosen carefully to ensure that pre-ionization dynamics are not computed in velocity gauge.
To verify the gauge consistency of the coefficients, they were compared with the ODE approach, which operates strictly in length gauge.







% Also interesting?? amplitude also gauge dependent as long as the laser pulse is not over.

\paragraph{States used in the SFA rate}
A similar issue arises with the tRecX coefficients as with the ODE coefficients: determining how many should be included in the calculation.
Unlike the subspace approach, ionization is permitted here, so including higher bound states is not expected to cause the same numerical difficulties as before.
However, the gauge radius should not be exceeded, as beyond this point, the coefficients correspond to the velocity gauge, which may produce incorrect or unphysical results.
To minimize numerical issues, only the first few bound states are typically used for the SFA rate.
This choice can be justified by examining the populations of the hydrogen atom, where most dynamics occur within the $1s$, $2s$, $2p$, and $3p$ states.
% Getting to the actual coefficients with real and imaginary part is much more of an effort than just determining the amplitude $|c_n(t)|^2$.
% They store much more information about the wavefunction than just the amplitude.








%%%%%%%%%%%%%%%%

% When field-free states are used as part of the basis, they retain their intended physical meaning only when
% length gauge is used. On the other hand, for computational efficiency we want velocity gauge where the electron
% is essentially moving freely. How to combine the two is described in Ref. [?]. In the transition region rather ugly,
% quadrupole-type operators appear. These are pre-defined for polar coordinates as <<MixedGaugeDipole:Rg=20>>.
% In this example length gauge will be used up to the “gauge radius” Rg = 20. The radius must coincide with an
% element boundary. This will be checked and the code terminates, if it is violated.
% Surfaces will be transformed to velocity gauge before saving, such that spectral analysis works exactly as in
% velocity gauge. At present, this transformation is only implemented from length to velocity gauge, therefore we
% need the surface radius Rc ≤ Rg (not a deep limitation, can and will be removed).
% Mixed gauge is computationally slightly less efficient than velocity gauge, see figure


% # note: there is a frequent desire to see the populations
% # BUT physical meaning can only be attached to it after the pulse is over
% # during the pulse, the values are gauge-dependent
%% this is from trecx manual



% in E4 it was easy because laser had cosine shape. attosecond physics not the case, more a lase pulse, cos8 envelope so it doesnt make much sense to speak of rabi oszillations.\\
% I also need to check how often I have to write the coefficients to the expec file. That depends on the characterisitc time of the state and on the frequency of the laser of course.\\
% Naiv: as much as possible to be precise as possible. 
% But: we dont want to exceed the length gauge regime, are best described in length not velocity gauge.
% Further: more phenomenological but most of the dynamics is int the first few bound states, like 2s, 2p, 3p thats mostly it. 
% For that just look at the $|c_n(t)|^2$ which has the highest amplitudes. %ask vlad if this is correct????



\section{Dipole Matrix Elements}
The dipole transition matrix elements $\vec{d}_{nlm}(\vec{p})$ from a certain bound state to the continuum states are somewhat less ambiguous than the coefficients. 
Calculating them in general is difficult, but in the case of hydrogen, an analytical solution is possible. A detailed derivation can be found in Appendix \ref{sec:dipolematrixelements}. 
The actual calculation of the formula for implementation in the code was performed in a Mathematica notebook, which can be found in the electronic appendix \cite{johannes_porsch_2025_16223179}.

As mentioned in the theory section, this simplicity in calculation is only possible due to the strong approximation made with the SFA regarding the dipole matrix elements. 
The final state after ionization is not truly a plane wave but is approximated as such under the SFA.

As discussed in the coefficients section, not all states contribute equally to the ionization dynamics. 
Besides the ground state $1s$, this thesis restricts calculations to the states $2p$ and $3p$. Most of the pre-ionization dynamics is determined by these states; including additional states would primarily refine the ionization rate rather than introduce significant new physics.

Simplifying the matrix elements is also crucial to avoid numerical instabilities. 
Particularly when integrating over the azimuthal angle $\phi_p$ in the final rate, where exact cancellations are expected, limited numerical precision can lead to incorrect results.

The dipole matrix elements can be simplified by noting that the ground state is $1s$ and the light wave is linearly polarized, ensuring that all other states also have $m=0$. Since the derivation is lengthy, it is provided in Appendix \ref{sec:dipolematrixelements} as well.

%%%%%%%%%%%%%%%%
\section{TIPTOE Simulations}
As discussed in Chapter 3, TIPTOE is used in this thesis to compare the ionization yield obtained from the SFA rate and the numerical solution of the TDSE with tRecX.

In implementation, both approaches can be treated as a type of ``machine'', where an arbitrary laser field is input and an ionization probability is output. The laser field in this case consists of two pulses in a \texttt{cos8} envelope: a strong ``fundamental'' pulse and a weaker ``signal'' pulse shifted in time by $\tau$. 
In tRecX, two laser pulses can be implemented directly. In the SFA rate implemented in Python, the laser field is constructed as an object of the class \texttt{LaserField} from the file \texttt{field\_functions.py}.

Particularly in the SFA rate, numerical challenges arise because the signal pulse acts only as a small perturbation to the fundamental pulse. Care must be taken to ensure that numerical methods accurately capture this perturbation; otherwise, TIPTOE will fail. This challenge is addressed in Chapter 5.