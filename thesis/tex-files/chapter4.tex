\section{tRecX}
Difference between length gauge and velocity gauge in numerics\\
Why is it so diffficult? Solving time dependent Schroedinger euation numerically is not that hard. But in attosecond regime electron in Hydrogen likes length gauge because everything can be defined by giving the position of the electron. 
But a free electron in a strong laser pulse really likes velocity gauge because everything can be described by the kinetic energy of the electron.
\subsection{irECS}
Lorem ipsum dolor sit amet, consetetur sadipscing elitr, sed diam nonumy eirmod tempor invidunt ut labore et dolore magna aliquyam erat, sed diam voluptua. At vero eos et accusam et justo duo dolores et ea rebum. Stet clita kasd gubergren, no sea takimata sanctus est Lorem ipsum dolor sit amet. Lorem ipsum dolor sit amet, consetetur sadipscing elitr, sed diam nonumy eirmod tempor invidunt ut labore et dolore magna aliquyam erat, sed diam voluptua. At vero eos et accusam et justo duo dolores et ea rebum. Stet clita kasd gubergren, no sea takimata sanctus est Lorem ipsum dolor sit amet.
\subsection{tSURFF}
Lorem ipsum dolor sit amet, consetetur sadipscing elitr, sed diam nonumy eirmod tempor invidunt ut labore et dolore magna aliquyam erat, sed diam voluptua. At vero eos et accusam et justo duo dolores et ea rebum. Stet clita kasd gubergren, no sea takimata sanctus est Lorem ipsum dolor sit amet. Lorem ipsum dolor sit amet, consetetur sadipscing elitr, sed diam nonumy eirmod tempor invidunt ut labore et dolore magna aliquyam erat, sed diam voluptua. At vero eos et accusam et justo duo dolores et ea rebum. Stet clita kasd gubergren, no sea takimata sanctus est Lorem ipsum dolor sit amet.
\subsection{Challenges}
dangling pointer: interesting problem actually, how to solve it, how to find it, etc

\newpage
\section{Python Implementation of Ionization Model}
it uses a kernel, quasistatic
