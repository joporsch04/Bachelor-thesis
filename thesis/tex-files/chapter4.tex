This chapter describes the implementation of the formula \eqref{eq:sfa_rate_improved} and the associated challenges.
The majority of the code was originally developed by the authors of \cite{Theory_NPS}.
Detailed modifications made to both the tRecX source code and the SFA rate are documented in the electronic appendix [github zenodo].
The two main new components in this implementation are the coefficients $c_n(t)$ and the dipole matrix elements $d_{z,n}(\uvec{p})$, while the remaining parts were straightforward.




%%%%%%%%%%%%%%%%
\section{Coefficients}
As mentioned in chapter 2, the Dyson equation can be written in two ways, resulting in two different expressions for the S-matrix.
The difference between the two approaches lies only in the coefficients.
The ODE coefficients are defined with respect to the subspace spanned by the bound states, whereas the tRecX coefficients are defined with respect to the full Hilbert space.



\subsection{ODE}
To see how to solve for the coefficients, lets look at the defining equation again:
\begin{equation*}
    \hat{\U}^{\mathrm{ODE}}_0(t',t_0)\ket{\Psi_0(t_0)} = \sum_{n}c_n(t')e^{-iE_nt'}\ket{\Psi_n}
\end{equation*}
with
\begin{equation}
    i\partial_{t'} \, \hat{\U}^{\mathrm{ODE}}_0(t',t_0) = \hat{X}\hat{\Hs}(t')\hat{X}\,\hat{\U}^{\mathrm{ODE}}_0(t',t_0)    \label{eq:odeU}
\end{equation}
One can think of it as the wavefunction is not "allowed" to leave the subspace. 
Lets do the derivation for the coefficients real quick.

We start by splitting the $\hat{\Hs}(t)$ into two parts $\hat{\Hs}_0$ and $\hat{\Hs}_{\mathrm{I}}(t)$ where the eigenstates of $\hat{\Hs}_0$ are $\{\ket{\Psi_n}\}$ and the eigenenergies $\hat{\Hs}_0$ are $\{E_n\}$. 
The next step is to plug the ansatz above into \eqref{eq:odeU} and multiply with $\bra{\Psi_m}$ to get the coefficients $c_n(t)$.
\begin{equation*}
    i  \dot{c}_m(t) = \sum_n c_n(t) e^{-i \omega_{nm} t} \braket{\Psi_m|\hat{\Hs}_{\mathrm{I}}(t)|\Psi_n}  \label{eq:ode}
\end{equation*}
with $\omega_{nm} = E_n - E_m$.

\medskip
Now implementing this leads to some challenges.
First, $\sum_n$ goes from $0$ to $\infty$, but in numerical simulations we have to limit ourselves to a finite number of states.
How can we justify the number choosen?
How can we make shure that the abrubt end of the sum does not cause any numerical problems?

Of course, solving for more coefficients will also affect the previos calculated coefficients.
For instance, just solving for the ground state $c_0(t)$ will lead to $|c_0(t)|^2$ being $1$.
However, the occupation for the ground state will decrease, the more states are being included in the sum.
Thats why its called a \emph{coupled} system of ODEs.

By increasing the number of states used in the sum, the results (i.e. the rates) should converge to a certain rate.
However, since we forced the electron to stay inside the subspace, going to higher and higher excited states can cause numerical problems as well.
Especially for intensities around $10^{14}\frac{\mathrm{W}}{\mathrm{cm}^2}$ and longer wavelengths ($800\mathrm{nm}-1200\mathrm{nm}$), allowing the electron to occupy higher states caused unphysical oscillations in the rates. 
"Normally" the electron would be ionized but since the ionisation is "turned off" in approach, the electron gets excited to a high excited state, causing the model to break apart.
Eventually one has to find a certain number of states that is precise enough but does not cause numerical problems.
For my simulations later, I used a maximum of $n=3$ for the calculation of the bound states.
One could justify that by looking at the population of the hydrogen atom over time; most of the dynamics before ionization is determined by the first few bound states, especially $1s$, $2s$, $2p$ and $3p$ so solving for more doesnt affect the previous coefficients too much.

Note: This was not about how many bound states I include in the final simulations, but how many I even considered when solving the ODE.
I could have used only the ground state in the SFA rate later on, but there is still the question how many bound states have an effect on the ground state. 
This number was justified here.


\medskip
Further, equation \eqref{eq:ode} is gauge dependent because of $\braket{m|\hat{\Hs}_{\mathrm{I}}(t)|n}$. 
We can either choose length gauge \eqref{eq:dipoleApprox} or velocity gauge \eqref{eq:dipoleApprox_velocity}.
Which gauge should we use to get the most meaningful results?

To answer this question, lets think about the scenario again.
An electron sits in ground state and bevore ionization its behavior is mainly described by the coulomb potential.
After it gets ionized, SFA tells us it is now described as a plane wave oscillating in the laser pulse.

With this in mind, we can see that both gauges can be usefull.
During the first part of the process, i.e. the electron being still bounded to the hydrogen atom, the length gauge is more appropiate since most of the behavior of the electron can be described better when knowing more about the position of the electron rather then its momentum.
That is why in most elementary introductions on light matter interaction (for instance with rabi socillations and so on) length gauge because the system can not ionize.
On the other hand, after ionization it is better to use velocity gauge since there is no potential anymore (because of SFA) and the elcetron is fully described by its momentum.

This makes length gauge a fitting candidate for the coefficients used in the ODE ansatz.



\bigskip
In my implementation I neglected transitions to states that are forbidden via the dipole selection rules. 
However this is an approximation, since in reality two-photon processes can occur, effectively allowing transitions between $1s$ and $2s$ for instance. 
I numerically solved the schroedinger equation with tRecX and modyfied the code to print out the coefficients $c_n(t)$ allowing me to get insight in the "real" dynamics of the electron.
% the dont differ???? why??? i thoguht im negecting that in ODE c_ns but it seems not, |2,0,0> is still there, why?
% i think im not neglecting these transitions, its the normal dynamic of the electron without neglecting 1s->2s transitions
% its jsut that the angular integral is 0 but thats just mathematics







\subsection{tRecX}
The coefficients from the ODE are not the only one im using. I used a numerical solver (in this case tRecX) to solve the entire TDSE and extract the coefficients from the wavefunction.
As mentioned earlier, there are two ways to think about $\hat{\U}(t',t_0)\ket{\Psi}$. 
First, solve the TDSE in the subspace of the bound states (we did that with the ODE) or solve the TDSE in the full Hilbert space and the project it onto bound states. 
tRecX does the second one, which is far more complicated than what the ODE is doing.
However, tRecX results are also gauge dependent but the problem with the sum of all the bound states we faced earlier in the ODE approach is not relevant here.
That is a problem of the developers of the numerical solver.
Also it is far more difficult to interpret the results from tRecX since many effects can determine the time evolution of the coefficients and therefore the ionization rate.
Nonetheless its of course helpfull to have two independent sources of in some sense the same thing.

For extracting the coefficients from tRecX so solving the full TDSE and then projecting onto bound states I modifyed the source code as following:
Already implemented was the Occupation propability of specified bound states, so the code prints out $\braket{\Psi(t)|\hat{P}_{\mathrm{Occ\{H0:n\}}}\footnote{The notation here is the same as in the code for better reference}|\Psi(t)}=|\braket{\Psi(t)|n}|^2$
In principle all I did was changing implementing a new function that changes the left bra $\bra{\Psi(t)}$ to the eigenstate used in $\hat{P}_{\mathrm{Occ\{H0:n\}}}$ and verifying that the eigenstates are normalized.
This gives us 
\begin{equation*}
    \braket{\Psi(t)|\hat{P}_{\mathrm{Occ\{H0:n\}}}|\Psi(t)} \rightarrow \braket{n|\hat{P}_{\mathrm{Occ\{H0:n\}}}|\Psi(t)} = \braket{n|n}\times\braket{n|\Psi(t)} = c_n(t)
\end{equation*}
For that I needed to solve the eigenvalue problem again, and pass the eigenstates down to the function calculating the expectation value.
There is defenitely a more elegant way to do that, especially efficiency wise, since the eigenstates are already calculated elsewhere, but for now this works.
A detailed description of the changes made to tRecX can be found in the appendix [github zenodo].


\bigskip



%%%%%%%%%%%%%%%%

% When field-free states are used as part of the basis, they retain their intended physical meaning only when
% length gauge is used. On the other hand, for computational efficiency we want velocity gauge where the electron
% is essentially moving freely. How to combine the two is described in Ref. [?]. In the transition region rather ugly,
% quadrupole-type operators appear. These are pre-defined for polar coordinates as <<MixedGaugeDipole:Rg=20>>.
% In this example length gauge will be used up to the “gauge radius” Rg = 20. The radius must coincide with an
% element boundary. This will be checked and the code terminates, if it is violated.
% Surfaces will be transformed to velocity gauge before saving, such that spectral analysis works exactly as in
% velocity gauge. At present, this transformation is only implemented from length to velocity gauge, therefore we
% need the surface radius Rc ≤ Rg (not a deep limitation, can and will be removed).
% Mixed gauge is computationally slightly less efficient than velocity gauge, see figure


% # note: there is a frequent desire to see the populations
% # BUT physical meaning can only be attached to it after the pulse is over
% # during the pulse, the values are gauge-dependent
%% this is from trecx manual

% The gauge is a very fundamental problem. Not so fundamental is our ansatz, as mentioned erlier we artificially force the electron to stay in the part of the Hilbert space covered by the bound state.
% This restricts us to small ionization propabilities, and may also cause some numerical problems as will discussed later.\\ %for 850nm weird sozillations for max_n > 3
% Also the numerical issue that we need to have infinitely many bound states for covering the aprt of the Hilbert space completely is not possible, so we have to limit ourselves.
% First this seems like a big problem, but we will see that most of the dynamics inside the electron before ionisation is in my simulations only determined by a few bound states.
% But one still need to be carefull with the number of used bound states, but because of numerical reasons.


% in E4 it was easy because laser had cosine shape. attosecond physics not the case, more a lase pulse, cos8 envelope so it doesnt make much sense to speak of rabi oszillations.\\
% I also need to check how often I have to write the coefficients to the expec file. That depends on the characterisitc time of the state and on the frequency of the laser of course.\\
% Naiv: as much as possible to be precise as possible. 
% But: we dont want to exceed the length gauge regime, are best described in length not velocity gauge.
% Further: more phenomenological but most of the dynamics is int the first few bound states, like 2s, 2p, 3p thats mostly it. 
% For that just look at the $|c_n(t)|^2$ which has the highest amplitudes. %ask vlad if this is correct????


\bigskip
How many coefficients will i be using later? only a few? actually only 1s 2p 3p not more


\section{Dipole matrix Elements}
The dipole transition matrix elements $\vec{d}_{nlm}(\uvec{p})$ from a certain bound state to the volkov states are a bit less ambigous than the coefficients.
Calculating them generall is difficult, but in the case of hydrogen its even possible to do it analytically.
A detailed derivation can be found in the appendix \ref{sec:dipolematrixelements}.


As mentioned in the theory section, its only that easy to calculate because with SFA we made a very coarse approximation about the dipole matrix elements.
The final state after ionisation is not in reality a plane wave, but we approximated it with that by using SFA.

As mentioned in the coefficients section, not all states have equal amount of influence on the ionization dynamics.
Besides the ground state $1s$, this thesis restricts its calculation to the states $2p$ and $3p$.
Most of the dynamics before ionization is determined by these states, including more and more states will just make the final ionization rate more precise, but not include much "new" physics.

It is also essential to simplify the matrix elements as much as possible to avoid numerical problems.
Especially when integrating over the azimutal angle $\phi$ in the final rate and things are supposed to cancel out, this can lead to worng results since of course values are only until certain accuracy.

To simplify the dipole matrix elements, one can use the fact that the ground state is the $1s$ state, and the light wave is linearly polarized.
$m=0$ because light linear polarized!!!!



%%%%%%%%%%%%%%%%
\section{\textcolor{red}{tRecX TIPTOE Simulations}}
trecx for reference to compare SFA and excited SFA
How to implement trecx tiptoe scan?

