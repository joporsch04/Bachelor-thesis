In this chapter I will explain how I implemented formula \eqref{eq:smatrix} in the code.
Most of the code was already written by the authors of \cite{Theory_NPS}.
The exact changes I made can be found in the electronic appendix.
Since I am using git, you can also see the changes in the commit history.
The main two parts that where new in this implementation now are the coefficients $c_n(t)$ and the dipole matrix elements $\vec{d}_n(\uvec{p})$.






%%%%%%%%%%%%%%%%
\section{Coefficients}
As mentioned in chapter 2, there are two ways to write the dyson equation which results in two different expressions for the s-matrix.
The only difference between both approaches are the coefficients. 
The ODE coefficients are with respect to the subspace spanned by the bound states, while tRecX coefficients are with respect to the full Hilbert space.







\subsection{ODE}
To see how to solve for the coefficients,, lets look at the defining equation again:
\begin{equation*}
    \hat{\U}^{\mathrm{ODE}}_0(t',t_0)\ket{\Psi_0(t_0)} = \sum_{n}c_n(t')e^{-iE_nt'}\ket{\Psi_n}
\end{equation*}
with
\begin{equation}
    i\partial_{t'} \, \hat{\U}^{\mathrm{ODE}}_0(t',t_0) = \hat{X}\hat{\Hs}(t')\hat{X}\,\hat{\U}^{\mathrm{ODE}}_0(t',t_0)    \label{eq:odeU}
\end{equation}
One can think of it as the wavefunction is not "allowed" to leave the subspace. 
Lets do the derivation for the coefficients real quick.

We start by splitting the $\hat{\Hs}(t)$ into two parts $\hat{\Hs}_0$ and $\hat{\Hs}_{\mathrm{I}}(t)$ where the eigenstates of $\hat{\Hs}_0$ are $\{\ket{\Psi_n}\}$ and the eigenenergies $\hat{\Hs}_0$ are $\{E_n\}$. 
The next step is to plug the ansatz above into \eqref{eq:odeU} and multiply with $\bra{\Psi_m}$ to get the coefficients $c_n(t)$.
\begin{equation*}
    i  \dot{c}_m(t) = \sum_n c_n(t) e^{-i \omega_{nm} t} \braket{\Psi_m|\hat{\Hs}_{\mathrm{I}}(t)|\Psi_n}  \label{eq:ode}
\end{equation*}
with $\omega_{nm} = E_n - E_m$.

\bigskip
Now this has its problems.
First, $\sum_n$ goes from $0$ to $\infty$, but since we are in numerical simulations we have to limit ourselves to a finite number of states.
How can we justify the number choosen?
How can we make shure that the abrubt end of the sum does not cause any numerical problems?

One way to think about it is just testing.
By increasing the number of states used in the sum, the results (i.e. the rates) should converge.
However, since we forced the electron to stay inside the subspace, going to higher and higher excited states can cause numerical problems as well.
Especially for higher intensities and longer wavelengths, going to higher states caused unphysical oscillations in the rates. 
Eventually one has to find a certain number of states that is precise enough but does not cause numerical problems.
For my simulations later, I used a maximum of $n=3$ for the calculation of the bound states.
One could justify that by looking at the population; most of the dynamics before ionization is determined by the first few bound states, especially $1s$, $2s$, $2p$ and $3p$ so solving for more doesnt affect the previous coefficients not much.

Note: This was not about how many bound states I include in the final simulations, but how many I even considered when solving the ODE.
I could have used only the ground state in the SFA rate later on, but there is still the question how many bound states have an effect on the ground state. 
This number was justified here.


\bigskip
Further, equation \eqref{eq:ode} is gauge dependent because of $\braket{m|\hat{\Hs}_{\mathrm{I}}(t)|n}$. 
We can either choose length gauge \eqref{eq:dipoleApprox} or velocity gauge \eqref{eq:dipoleApprox_velocity}.
Which gauge should we use to get the most meaningful results?

To answer this question, lets think about the scenario again.
An electron sits in ground state and bevore ionization its behavior is mainly described by the coulomb potential.
After it gets ionized, SFA tells us it is now described as a plane wave oscillating in the laser pulse.

With this in mind, we can see that both gauges can be usefull.
During the first part of the process, i.e. the electron being still bounded to the hydrogen atom, the length gauge is more appropiate since most of the behavior of the electron can be described better when knowing more about the position of the electron rather then its momentum.
That is why in most elementary introductions on light matter interaction (for instance with rabi socillations and so on) length gauge because the system can not ionize.
On the other hand, after ionization it is better to use velocity gauge since there is no potential anymore (because of SFA) and the elcetron is fully described by its momentum.

This makes length gauge a fitting candidate for the coefficients used in the ODE ansatz.




\subsection{tRecX}






The coefficients from the ODE are not the only one im using. I used tRecX to calculate solve the entire TDSE and extract the coefficients from the whole wavefunction.
As mentioned earlier, there are two ways to think about $\hat{\U}(t',t_0)\ket{\Psi}$. 
First, solve the TDSE in the subspace of the bound states (we did that with the ODE) or solve the TDSE in the full Hilbert space and the project it onto bound states. 
tRecX does the second one, which is far more complicated than what I did using the ODE.
However, tRecX results are also gauge dependent, so we need to be careful with the gauge.
Also it is far more difficult to interpret the rsults from tRecX since many effects can determine the time evolution of the coefficients and therefore the ionization rate but its of course helpfull to have two independent sources of in some sense the same thing.







%%%%%%%%%%%%%%%%
\subsection{Coefficients}
In theory, $|c_n(t)|^2$ or the amplitude of the coefficients is observable quantity and can be measured in experiment.
It tells us how propable the system is to be in a certain state $n$ at time $t$.
Furthermore, and most importantly, they are gauge independent.
But since SFA is gauge dependent theory, and now we are not even dealing with the amplitude instead the complex coefficients, we need to be extremely careful with the gauge.\\
In my thesis there are mostly two gauges used, the length gauge and the velocity gauge. 
In the following plot you see that the amplitudes are indeed not gauge independent. But why is that the case?\\\\

% When field-free states are used as part of the basis, they retain their intended physical meaning only when
% length gauge is used. On the other hand, for computational efficiency we want velocity gauge where the electron
% is essentially moving freely. How to combine the two is described in Ref. [?]. In the transition region rather ugly,
% quadrupole-type operators appear. These are pre-defined for polar coordinates as <<MixedGaugeDipole:Rg=20>>.
% In this example length gauge will be used up to the “gauge radius” Rg = 20. The radius must coincide with an
% element boundary. This will be checked and the code terminates, if it is violated.
% Surfaces will be transformed to velocity gauge before saving, such that spectral analysis works exactly as in
% velocity gauge. At present, this transformation is only implemented from length to velocity gauge, therefore we
% need the surface radius Rc ≤ Rg (not a deep limitation, can and will be removed).
% Mixed gauge is computationally slightly less efficient than velocity gauge, see figure


% # note: there is a frequent desire to see the populations
% # BUT physical meaning can only be attached to it after the pulse is over
% # during the pulse, the values are gauge-dependent


The gauge is a very fundamental problem. Not so fundamental is our ansatz, as mentioned erlier we artificially force the electron to stay in the part of the Hilbert space covered by the bound state.
This restricts us to small ionization propabilities, and may also cause some numerical problems as will discussed later.\\ %for 850nm weird sozillations for max_n > 3
Also the numerical issue that we need to have infinitely many bound states for covering the aprt of the Hilbert space completely is not possible, so we have to limit ourselves.
First this seems like a big problem, but we will see that most of the dynamics inside the electron before ionisation is in my simulations only determined by a few bound states.
But one still need to be carefull with the number of used bound states, but because of numerical reasons.\\\\

For my numerica solution I used a system of ordinary differential equations (ODEs) to calculate the coefficients $c_n(t)$ using the interaction picture.
In my implementation I neglected transitions to states that are forbidden via the dipole selection rules. 
However this is an approximation, since in reality two-photon processes can occur, effectively allowing transitions between $1s$ and $2s$ for instance. 
I numerically solved the schroedinger equation with tRecX and modyfied the code to print out the coefficients $c_n(t)$ allowing me to get insight in the "real" dynamics of the electron.
% the dont differ???? why??? i thoguht im negecting that in ODE c_ns but it seems not, |2,0,0> is still there, why?
% i think im not neglecting these transitions, its the normal dynamic of the electron without neglecting 1s->2s transitions
% its jsut that the angular integral is 0 but thats just mathematics
\\\\
Question: how many cound states do we nodd (mentioned earlier)?\\

in E4 it was easy because laser had cosine shape. attosecond physics not the case, more a lase pulse, cos8 envelope so it doesnt make much sense to speak of rabi oszillations.\\
I also need to check how often I have to write the coefficients to the expec file. That depends on the characterisitc time of the state and on the frequency of the laser of course.\\
Naiv: as much as possible to be precise as possible. 
But: we dont want to exceed the length gauge regime, are best described in length not velocity gauge.
Further: more phenomenological but most of the dynamics is int the first few bound states, like 2s, 2p, 3p thats mostly it. 
For that just look at the $|c_n(t)|^2$ which has the highest amplitudes.\\\\ %ask vlad if this is correct????
For extracting the coefficients from tRecX so solving the full TDSE and then projecting onto bound states I modifyed the code as following:
Already implemented was the Occupation propability of specified bound states, so the code prints out $\braket{\Psi(t)|\hat{P}_{\mathrm{Occ\{H0:n\}}}\footnote{The notation here is the same as in the code for better reference}|\Psi(t)}=|\braket{\Psi(t)|n}|^2$
In principle all I did was changing implementing a new function that changes the left bra $\bra{\Psi(t)}$ to the eigenstate used in $\hat{P}_{\mathrm{Occ\{H0:n\}}}$ and verifying that the eigenstates are normalized.
This gives us 
\begin{equation*}
    \braket{\Psi(t)|\hat{P}_{\mathrm{Occ\{H0:n\}}}|\Psi(t)} \rightarrow \braket{n|\hat{P}_{\mathrm{Occ\{H0:n\}}}|\Psi(t)} = \braket{n|n}\times\braket{n|\Psi(t)} = c_n(t)
\end{equation*}
For that I needed to solve the eigenvalueporblem again, and pass the eigenstates down to the function calculating the expectation value.
There is defenitely a more elegant way to do that, especially efficiency wise, since the eigenstates are already calculated elsewhere, but for now this works.





\subsection{Dipole matrix Elements}
The dipole matrix elements are an essential part of the formula in \eqref{eq:smatrix}. 
Calculating it generall is difficult, but in the case of hydrogen its even possible to do it analytically \ref{sec:dipolematrixelements}.
However, its that easy to calculate because we made a very coarse approximation, since the final state after ionisation is not in realtiy a plane wave, but we approximated it with that by using SFA.

Furthermore, our problem is \\\\

For the dipole matrix elements its alsow needed to simplify them as much as possible. 
Sometimes certain parts cancel out later in the integration (phi integral) but the computer is not good with that, even some small numerical errors will cause it to be not exactl zero and messes up the calculation.




%%%%%%%%%%%%%%%%
\section{\textcolor{red}{tRecX TIPTOE Simulations}}
\subsection{TIPTOE}
TIPTOE \cite{Park:18} is a sampling method used for sub femtosecond processes. It is relevant for this thesis because it was used to verify the results from the Ionization model. 
TIPTOE is great because its fundamentals are very simple but it can tell you a lot about the dynamics in attosecond regime. \\\\

Now i used length gauge for the coefficients but velocity gauge for the ionization propabilities





