% Solving TDSE for Hydrogen atom because it will be important later. also note the structure of wavefunction:
% \begin{equation*}
%     \psi_{nlm}(\uvec{x}) = \psi_{nlm}(r,\theta,\phi) = R_{nl}(r)Y_{lm}(\theta, \phi)
% \end{equation*}
% $E_n=\frac{Z^2}{2n^2}$

\label{sec:dipolematrixelements}


Here, the general transition dipole matrix elements into the continuum for a hydrogen-like atom are derived.     %hydrogenlike???
The general matrix element in this case is given by:
\begin{equation*}
    \uvec{d}_{nlm}(\uvec{p}) = \braket{\Pi|\vec{\hat{d}}|\Psi_{nlm}} \stackrel{\mathrm{a.u.}}{=} \braket{p|\vec{\hat{r}}|\Psi_{nlm}}
\end{equation*}
where $\ket{p}$ is a plane wave.
The wave function for the hydrogen atom is well known:
\begin{equation}
    \Psi_{nlm}(\uvec{x}) = \braket{\uvec{x}|\Psi_{nlm}} = R_{nl}(r)Y_{lm}(\theta, \phi) \label{eq:psi_position_hydrogen}
\end{equation}
with $R_{nl}(r)$ being the radial part of the wavefunction and $Y_{lm}(\theta, \phi)$ being the spherical harmonics.

By partitioning the $\hat{\vec{1}}$ operator and using the fact that $\vec{\hat{r}} \rightarrow i\nabla_{\uvec{p}}$ %Interesting because by choosing a certain way to display the operator we also choose a natural basis for the representation of the operator. Therefore I write \nabla_{\uvec{p}} instead of \nabla_{\vec{p}}.
in momentum representation, a general formula for the transition is obtained:
\begin{equation*}
    \uvec{d}_{nlm}(\uvec{p}) = i\nabla_{\uvec{p}}\int \dd^3\uvec{x}\,\psi_{nlm}(\uvec{x}) e^{-i\uvec{p}\cdot\uvec{x}} = i\nabla_{\uvec{p}}\phi_{nlm}(\uvec{p})
\end{equation*}
In principle, this integral (i.e the Fourier transform) is all that is required.
Due to the structure of $\psi_{nlm}$, a result similar to \eqref{eq:psi_position_hydrogen} can be expected.
A posteriori, it can be shown that:
\begin{equation*}
    \F\{\psi_{nlm}(\uvec{x})\} = \phi_{nlm}(\uvec{p}) = F_{nl}(p)Y_{lm}(\theta_p, \phi_p)
\end{equation*}
where $F_{nl}(p)$ is the Fourier transform of the radial part of the wavefunction and $Y_{lm}(\theta_p, \phi_p)$ are the spherical harmonics in momentum space, analogous to the hydrogen atom in position space.






%%%%%%%%%%%%%%%%%%%%%%
\subsection*{Momentum space}
The so-called plane wave expansion \cite{Jackson:1998nia} of the exponential part of the integral is given by:
\begin{equation*}
    e^{i\uvec{p}\cdot\uvec{x}} = \sum_{l'=0}^\infty (2l'+1)i^{l'} j_{l'}(pr) P_{l'}(\uvec{p}\cdot\uvec{x}) = 4\pi\sum_{l'=0}^\infty \sum_{m'=-l'}^{l'} i^{l'} j_{l'}(pr) Y_{l'm'}(\theta_p, \phi_p) Y_{l'm'}^*(\theta_x, \phi_x)
\end{equation*}
Here, $j_l(pr)$ represents the spherical Bessel functions, and the integration is performed over spherical coordinates. 
While the expression initially appears complex, the orthogonality of the spherical harmonics can be used to simplify the integral to:
\begin{equation*}
    \phi_{nlm}(\uvec{p}) = 4\pi \sum_{m=-l}^{l}Y_{lm}(\theta_p,\phi_p)i^l\underbrace{\int_{0}^{\infty}\dd r\,r^2j_l(pr)R_{nl}(r)}_{\tilde{R}_{nl}(p)}
\end{equation*}
This structure is the desired form. The focus now shifts to the radial part $\tilde{R}_{nl}(p)$ of the integral.
The term $R_{nl}(r)$ corresponds to the radial function of the hydrogen atom in position space and is independent of the magnetic quantum number $m$.
It consists of an exponential term dependent on $r$, a polynomial term dependent on $r$, the generalized Laguerre polynomials, and the normalization constant.
A closed expression for the generalized Laguerre polynomials would be convenient. They are represented as:
\begin{equation*}
    L_n^l(r) = \sum_{\iota=0}^{n} \frac{(-1)^{\iota}}{\iota!}\binom{n+l}{n-\iota}r^{\iota}
\end{equation*}
Thus, the Laguerre polynomials depend only on an exponential term and finitely many polynomial terms.
$\tilde{R}_{nl}(p)$ can be expressed (without prefactors and summation over $\iota$) as:
\begin{equation*}
    \int_{0}^{\infty}\dd r\,r^{2+l+\iota} e^{-\frac{Zr}{n}} j_l(pr)
\end{equation*}
Before solving the integral using computational methods, the spherical Bessel function must be transformed into ordinary Bessel functions:
\begin{equation*}
    j_l(pr) = \sqrt{\frac{\pi}{2pr}}J_{l+\frac{1}{2}}(pr)
\end{equation*}
At this stage, it is useful to combine all prefactors and summations into a single expression and examine the integral as a whole:
\begin{equation}
\label{eq:phi_nlm_momentum}
\begin{aligned}
    \phi_{nlm}(\uvec{p}) =\ & \frac{\pi^{3/2}}{\sqrt{2p}}\sqrt{\left(\frac{2}{n}\right)^3\frac{(n-l-1)!}{n(n+1)!}}\\
    & \times \sum_{m=-l}^{l}\sum_{\iota=0}^{n-l-1}i^l\frac{(-1)^{\iota}}{\iota!}\left(\frac{2}{n}\right)^{l+\iota}\binom{n+l}{n-l-1}\underbrace{\int_{0}^{\infty}\dd r\,r^{l+\iota+\frac{3}{2}}e^{-\frac{Zr}{n}}J_{l+\frac{1}{2}}(pr)}_{(*)} Y_{lm}(\theta_p,\phi_p)
\end{aligned}
\end{equation}
The remaining integral was computed using Mathematica, so a detailed derivation is not provided. Interestingly, an analytical solution exists.
The result for $(*)$ is:
\begin{equation*}
    (*) = {}_2\tilde{F}_1\left(2 + l + \frac{\iota}{2}, \frac{1}{2}(5 + 2l + \iota); \frac{3}{2} + l; -\frac{n^2 p^2}{Z^2}\right)
\end{equation*}
Here, ${}_2\tilde{F}_1$ denotes the regularized hypergeometric function, defined as:
\begin{equation*}
    {}_2\tilde{F}_1(a,b;c;z) = \frac{{}_2F_1(a,b;c;z)}{\Gamma(c)} = \frac{1}{\Gamma(a)\,\Gamma(b)} \sum_{n=0}^{\infty} \frac{\Gamma(a+n)\,\Gamma(b+n)}{\Gamma(c+n)} \frac{z^n}{n!}
\end{equation*}
The final expression for $\phi_{nlm}(\uvec{p})$, which can also be found in \cite{Bransdenatomsmolecules} in a slightly different form, is given by:
\begin{align}
    \label{eq:phi_nlm}
    \phi_{nlm}(\uvec{p}) = \sum_{\iota=0}^{2l+1} \;
        & \frac{(-1)^{\iota} \; 2^{\iota + \frac{1}{2}} \; n \; (i n)^l \; (p^2)^{l/2} \; Z^{-l-3} \; \Gamma(2l+\iota+3)}{\iota!} \nonumber \\
        & \times \binom{l+n}{-l+n-\iota-1} 
        \sqrt{\frac{Z^3 \Gamma(n-l)}{\Gamma(l+n+1)}} \nonumber \\
        & \times Y_l^m(\theta_p, \phi_p) \,
        {}_2\tilde{F}_1\left(l+\frac{\iota}{2}+2, \frac{1}{2}(2l+\iota+3); l+\frac{3}{2}; -\frac{n^2 p^2}{Z^2}\right)
\end{align}






%%%%%%%%%%%%%%%%%%%%%%%
\subsection*{Improved rate}
Usually all that remains is to differentiate \eqref{eq:phi_nlm} with respect to $\uvec{p}$.
However, before that, the SFA rate can be improved directly by performing the integration over $\phi_p$ analytically.
The rate according to chapter 2 is given by:
\begin{align*}
    \Gamma_{\mathrm{SFA}}(t) &= \sum_{n_1}\sum_{n_2} \int_0^{\infty} \dd p\,p^2\int_0^{\pi} \dd\theta_p\,\sin\theta_p \int_0^{2\pi}\dd \phi_p\int_{-\infty}^{\infty} \dd T\\
    &\times \exp\left(i\vec{p}^2T + ip\cos\theta_p\bar{\Delta}_z^k +  \frac{i}{2}\bar{\Delta}_z^2  + i(t-T)E_{n_1}-i(t+T)E_{n_2}\right)\\
    &\times E_z(t-T) E_z(t+T)c_{n_1}^*(t-T)c_{n_2}(t+T) d_{z,n_1}^*(\uvec{p}+\Delta_z(-T))d_{z,n_2}(\uvec{p}+\Delta_z(T))
\end{align*}
As mentioned before, the dipole transition matrix element can be written in the following convenient form:
\begin{equation*}
    d_{z,nlm}(\uvec{p}) = \left[i\nabla_{\uvec{p}}\phi_{nlm}(\uvec{p})\right]_z = iY_{lm}(\theta_p, \phi_p)\left(\cos\theta_p\frac{\partial F_{nl}}{\partial p}-\frac{\sin\theta_p}{p}\frac{\partial F_{nl}}{\partial p}\right)
\end{equation*}
Note that here the $z$ component of a gradient in spherical coordinates was taken. The exact formula can be easily computed using the transformation between Cartesian and spherical coordinates.

The integration over $\theta_p$ is difficult and might not be possible to perform analytically.
However, for the integral over $\phi_p$, only the following needs to be solved:
\begin{align*}
    \int_{0}^{2\pi}\dd \phi_p\, Y_{l'm'}^*(\theta_p, \phi_p) Y_{lm}(\theta_p, \phi_p) &= \frac{1}{4\pi}\sqrt{(2l+1)(2l'+1)\frac{(l-m)!(l'-m')}{(l+m)!(l'+m')!}}\int_{0}^{2\pi}\dd \phi_p\, \e^{i(m-m')\phi_p}\\
    &= \frac{1}{2}\sqrt{(2l+1)(2l'+1)\frac{(l-m)!(l'-m)}{(l+m)!(l'+m)!}}
\end{align*}
This means that the magnetic quantum number is conserved during the transition between two arbitrary states.
Since the initial state is the ground state $1s$, $m$ is zero here and therefore everywhere.
This significantly simplifies the expression.






% Further since only the $z$ component of the gradient is important only the partial derivative of $\phi_{nlm}(\uvec{p})$ with respect to $p$ is needed.
% This can be also done analytically.
% \begin{align*}
%     \frac{\partial F_{nl}(p)}{\partial p} = \sum_{\iota=0}^{2l+1} \;
%         & \frac{(-1)^{\iota} \; 2^{\iota + \frac{1}{2}} \; n \; (i n)^l \; \; Z^{-l-3} \; \Gamma(2l+\iota+3)}{\iota!} \nonumber \\
%         & \times \binom{l+n}{-l+n-\iota-1} 
%         \sqrt{\frac{Z^3 \Gamma(n-l)}{\Gamma(l+n+1)}} \nonumber \\
%         & \times  \,
%         (lp^{l-1}{}_2\tilde{F}_1\left(l+\frac{\iota}{2}+2, \frac{1}{2}(2l+\iota+3); l+\frac{3}{2}; -\frac{n^2 p^2}{Z^2}\right)\\
%         &\times\frac{p^ln^2}{Z^2}{}_2\tilde{F}_1\left(l+\frac{\iota}{2}+1, \frac{1}{2}(2l+\iota+1); l+\frac{1}{2}; -\frac{n^2 p^2}{Z^2}\right))
% \end{align*}

% Now lets see what the integral over $\theta$ and $\phi$ will look like:
% \begin{equation*}
%     \frac{\partial F_{n'l'}^*}{\partial p}\frac{\partial F_{nl}}{\partial p}\int_0^{\pi}\int_0^{2\pi}\sin\theta_p \e^{ip\cos\theta_p\bar{\Delta}_z^k} Y_{l'm'}^* Y_{lm}(\theta_p, \phi_p)\left(\cos\theta_p - \frac{\sin\theta_p}{p}\right)^2\,\dd\theta_p\dd \phi_p
% \end{equation*}
% The rest is independent of $\theta$ and $\phi$.


% %%%%%%%%%%%%%%%%%%%%%%%
% \subsection*{\textcolor{red}{Transition Element}}
% Now all thats left is to differentiate \eqref{eq:phi_nlm} with respect to $\uvec{p}$. 


% new eq
% \begin{align}
%     \sum _{\iota =0}^{2 l+1} \left(-\frac{(-1)^{\iota } n \text{Ip}^{-l-3} 2^{\iota -l-1} (i n)^l \left(p^2\right)^{l/2}
%    \left(\frac{1}{\sqrt{p^2}}-\frac{\text{pz}^2}{\left(p^2\right)^{3/2}}\right) \Gamma (2 l+\iota +3) \binom{l+n}{-l+n-\iota -1} \sqrt{\frac{\text{Ip}^3 \Gamma
%    (n-l)}{\Gamma (l+n+1)}} \, _2\tilde{F}_1\left(l+\frac{\iota }{2}+2,\frac{1}{2} (2 l+\iota +3);l+\frac{3}{2};-\frac{n^2 p^2}{4 \text{Ip}^2}\right) \left(\frac{m
%    \text{pz} Y_l^m\left(\cos ^{-1}\left(\frac{\text{pz}}{\sqrt{p^2}}\right),\tan ^{-1}(\text{px},\text{py})\right)}{\sqrt{p^2}
%    \sqrt{1-\frac{\text{pz}^2}{p^2}}}+\frac{\sqrt{\Gamma (l-m+1)} \sqrt{\Gamma (l+m+2)} e^{-i \tan ^{-1}(\text{px},\text{py})} Y_l^{m+1}\left(\cos
%    ^{-1}\left(\frac{\text{pz}}{\sqrt{p^2}}\right),\tan ^{-1}(\text{px},\text{py})\right)}{\sqrt{\Gamma (l-m)} \sqrt{\Gamma (l+m+1)}}\right)}{\iota !
%    \sqrt{1-\frac{\text{pz}^2}{p^2}}}+\frac{(-1)^{\iota } l n \text{pz} \text{Ip}^{-l-3} 2^{\iota -l-1} (i n)^l \left(p^2\right)^{\frac{l}{2}-1} \Gamma (2 l+\iota +3)
%    \binom{l+n}{-l+n-\iota -1} \sqrt{\frac{\text{Ip}^3 \Gamma (n-l)}{\Gamma (l+n+1)}} \, _2\tilde{F}_1\left(l+\frac{\iota }{2}+2,\frac{1}{2} (2 l+\iota
%    +3);l+\frac{3}{2};-\frac{n^2 p^2}{4 \text{Ip}^2}\right) Y_l^m\left(\cos ^{-1}\left(\frac{\text{pz}}{\sqrt{p^2}}\right),\tan
%    ^{-1}(\text{px},\text{py})\right)}{\iota !}-\frac{(-1)^{\iota } n^3 \text{pz} \text{Ip}^{-l-5} 2^{\iota -l-3} \left(\frac{\iota }{2}+l+2\right) (\iota +2 l+3) (i
%    n)^l \left(p^2\right)^{l/2} \Gamma (2 l+\iota +3) \binom{l+n}{-l+n-\iota -1} \sqrt{\frac{\text{Ip}^3 \Gamma (n-l)}{\Gamma (l+n+1)}} \,
%    _2\tilde{F}_1\left(l+\frac{\iota }{2}+3,\frac{1}{2} (2 l+\iota +3)+1;l+\frac{5}{2};-\frac{n^2 p^2}{4 \text{Ip}^2}\right) Y_l^m\left(\cos
%    ^{-1}\left(\frac{\text{pz}}{\sqrt{p^2}}\right),\tan ^{-1}(\text{px},\text{py})\right)}{\iota !}\right)
% \end{align}
% \begin{align}
%     \sum_{\iota=0}^{2l+1} \Bigg(
%         & -\frac{(-1)^{\iota} n\, \text{Ip}^{-l-3} 2^{\iota-l-1} (i n)^l (p^2)^{l/2}
%         \left(\frac{1}{\sqrt{p^2}} - \frac{\text{pz}^2}{(p^2)^{3/2}}\right) \Gamma(2l+\iota+3)
%         \binom{l+n}{-l+n-\iota-1}
%         \sqrt{\frac{\text{Ip}^3 \Gamma(n-l)}{\Gamma(l+n+1)}} }{\iota! \sqrt{1-\frac{\text{pz}^2}{p^2}}} \nonumber \\
%         & \times {}_2\tilde{F}_1\left(l+\frac{\iota}{2}+2, \frac{1}{2}(2l+\iota+3); l+\frac{3}{2}; -\frac{n^2 p^2}{4 \text{Ip}^2}\right)
%         \left(
%             \frac{m\, \text{pz}\, Y_l^m\left(\cos^{-1}\left(\frac{\text{pz}}{\sqrt{p^2}}\right), \tan^{-1}(\text{px},\text{py})\right)}
%             {\sqrt{p^2} \sqrt{1-\frac{\text{pz}^2}{p^2}}}
%         \right. \nonumber \\
%         & \qquad \left.
%             + \frac{
%                 \sqrt{\Gamma(l-m+1)} \sqrt{\Gamma(l+m+2)} e^{-i \tan^{-1}(\text{px},\text{py})}
%                 Y_l^{m+1}\left(\cos^{-1}\left(\frac{\text{pz}}{\sqrt{p^2}}\right), \tan^{-1}(\text{px},\text{py})\right)
%             }{
%                 \sqrt{\Gamma(l-m)} \sqrt{\Gamma(l+m+1)}
%             }
%         \right) \nonumber \\
%         & + \frac{(-1)^{\iota} l n\, \text{pz}\, \text{Ip}^{-l-3} 2^{\iota-l-1} (i n)^l (p^2)^{\frac{l}{2}-1}
%             \Gamma(2l+\iota+3) \binom{l+n}{-l+n-\iota-1}
%             \sqrt{\frac{\text{Ip}^3 \Gamma(n-l)}{\Gamma(l+n+1)}} }{\iota!}
%         {}_2\tilde{F}_1\left(l+\frac{\iota}{2}+2, \frac{1}{2}(2l+\iota+3); l+\frac{3}{2}; -\frac{n^2 p^2}{4 \text{Ip}^2}\right)
%         Y_l^m\left(\cos^{-1}\left(\frac{\text{pz}}{\sqrt{p^2}}\right), \tan^{-1}(\text{px},\text{py})\right) \nonumber \\
%         & - \frac{(-1)^{\iota} n^3 \text{pz} \text{Ip}^{-l-5} 2^{\iota-l-3} \left(\frac{\iota}{2}+l+2\right) (\iota+2l+3) (i n)^l (p^2)^{l/2}
%             \Gamma(2l+\iota+3) \binom{l+n}{-l+n-\iota-1}
%             \sqrt{\frac{\text{Ip}^3 \Gamma(n-l)}{\Gamma(l+n+1)}} }{\iota!}
%         {}_2\tilde{F}_1\left(l+\frac{\iota}{2}+3, \frac{1}{2}(2l+\iota+3)+1; l+\frac{5}{2}; -\frac{n^2 p^2}{4 \text{Ip}^2}\right)
%         Y_l^m\left(\cos^{-1}\left(\frac{\text{pz}}{\sqrt{p^2}}\right), \tan^{-1}(\text{px},\text{py})\right)
%     \Bigg)
% \end{align}


% \phi_{nlm}(\uvec{p}) =\ & \sqrt{2} \left(\frac{1}{n}\right)^{-2 l-3} 
% \sqrt{\frac{(-l+n-1)!}{n^4 (l+n)!}} \\
% & \times \sum _{\iota =0}^{-l+n-1} 
% \frac{(-2)^{\iota } i^l p^l \Gamma (2 l+\iota +3)
% \binom{l+n}{-l+n-\iota -1} \, 
% {}_2\tilde{F}_1\left(l+\frac{\iota }{2}+2,\frac{1}{2} (2 l+\iota +3);l+\frac{3}{2};-n^2 p^2\right)}
% {\iota !} Y_{lm}(\theta_p,\phi_p)




% Some dipole matrix elements:\\
% First start with transforming the Schroedinger equation into momentum space %https://physics.stackexchange.com/questions/249400/schr%C3%B6dinger-equation-in-momentum-space
% Note that the prefactor does NOT depend on magentic quantumnumber m.\\
% Spherical harmonic: Instead of $Y_{lm}(\theta, \phi)$ you can write $Y_{lm}(\uvec{r})$ since $\uvec{r}=\hat{e}_x \sin(\theta)\cos(\phi)+\hat{e}_y\sin(\theta)\sin(\phi)+\hat{e}_z\cos(\theta)$\\
% The transition dipole matrix element is given by
% \begin{equation*}
%     % Analytical result for the integral
%     [ \int_0^\infty x^{\mu} e^{-\alpha x} J_\nu(\beta x) dx = \frac{\beta^\nu \Gamma(\mu+\nu+1)}{2^\nu \alpha^{\mu+\nu+1}} , {}_2F_1\left(\frac{\mu+\nu+1}{2}, \frac{\mu+\nu+2}{2}; \nu+1; -\frac{\beta^2}{\alpha^2}\right) ]
% \end{equation*}



























% \begin{equation*}
%     \vec{d}(\vec{p}) = \braket{\vec{p}|\vec{\hat{d}}|\Psi_{nlm}} = \nabla_{\vec{p}}\Psi_{nlm}(\vec{p})
% \end{equation*}
% \begin{equation*}
%     \braket{\vec{p}|100} = \frac{8 \sqrt{\pi }}{\sqrt{\frac{1}{a^3}} \left(a^2 p^2+1\right)^2}\text{if}\Re\left(\frac{1}{a}\right)>0
% \end{equation*}
% \begin{equation*}
%     \braket{\vec{p}|200} = \frac{32 \sqrt{2 \pi } \left(4 a^2 p^2-1\right)}{\sqrt{\frac{1}{a^3}} \left(4 a^2 p^2+1\right)^3}\text{if}\Re\left(\frac{1}{a}\right)>0
% \end{equation*}
% \begin{equation*}
%     \braket{\vec{p}|210} = -\frac{128 i \sqrt{2 \pi } \sqrt{\frac{1}{a^3}} a^4 p}{\left(4 a^2 p^2+1\right)^3}\text{if}\Re\left(\frac{1}{a}\right)>0
% \end{equation*}
% \begin{equation*}
%     \braket{\vec{p}|300} = \frac{72 \sqrt{3 \pi } \left(81 a^4 p^4-30 a^2 p^2+1\right)}{\sqrt{\frac{1}{a^3}} \left(9 a^2p^2+1\right)^4}\text{ if }\Re\left(\frac{1}{a}\right)>0
% \end{equation*}
% \begin{equation*}
%     \braket{\vec{p}|310} = -\frac{864 i \sqrt{2 \pi } \sqrt{\frac{1}{a^3}} a^4 p \left(9 a^2 p^2-1\right)}{\left(9 a^2 p^2+1\right)^4}
% \end{equation*}
% \begin{equation*}
%     \braket{\vec{p}|320} = -\frac{1728 \sqrt{6 \pi } \sqrt{\frac{1}{a^3}} a^5 p^2}{\left(9 a^2 p^2+1\right)^4}
% \end{equation*}
% \begin{equation*}
%     \braket{\vec{p}|400} = \frac{256 \sqrt{\pi } \left(4096 a^6 p^6-1792 a^4 p^4+112 a^2 p^2-1\right)}{\sqrt{\frac{1}{a^3}}\left(16 a^2 p^2+1\right)^5}\text{ if }\Re\left(\frac{1}{a}\right)>0
% \end{equation*}
% \begin{equation*}
%     \braket{\vec{p}|410} = -\frac{2048 i \sqrt{\frac{\pi }{5}} \left(\frac{1}{a^3}\right)^{3/2} a^7 p \left(32 a^2 p^2 \left(40 a^2p^2-7\right)+5\right)}{\left(16 a^2 p^2+1\right)^5}\text{ if }\Re\left(\frac{1}{a}\right)>0
% \end{equation*}
% \begin{equation*}
%     \braket{\vec{p}|420} = -\frac{32768 \sqrt{\pi } \sqrt{\frac{1}{a^3}} a^5 p^2 \left(16 a^2 p^2-1\right)}{\left(16 a^2 p^2+1\right)^5}
% \end{equation*}
% \begin{equation*}
%     \braket{\vec{p}|430} = \frac{262144 i \sqrt{\frac{\pi }{5}} \sqrt{\frac{1}{a^3}} a^6 p^3}{\left(16 a^2 p^2+1\right)^5}
% \end{equation*}
% \begin{equation*}
%     \braket{\vec{p}|900} = \frac{1944 \sqrt{\pi } \left(27 a^2 p^2-1\right) \left(243 a^2 p^2-1\right) \left(243 a^2 p^2 \left(6561
%     a^4 p^4-729 a^2 p^2+11\right)-1\right) \left(729 \left(243 a^6 p^6-99 a^4 p^4+a^2
%     p^2\right)-1\right)}{\sqrt{\frac{1}{a^3}} \left(81 a^2 p^2+1\right)^{10}}\text{ if
%     }\Re\left(\frac{1}{a}\right)>0
% \end{equation*}
% \begin{equation*}
%     \braket{\vec{p}|510} = -\frac{4000 i \sqrt{10 \pi } \sqrt{\frac{1}{a^3}} a^4 p \left(15625 a^6 p^6-3375 a^4 p^4+135 a^2
%     p^2-1\right)}{\left(25 a^2 p^2+1\right)^6}\text{ if }\Re\left(\frac{1}{a}\right)>0
% \end{equation*}