Convention: $\Psi$ for an abstract state, $\psi$ for a wavefunction in position space, $\phi$  for a wavefunction in momentum space, $\vec{A}$ for abstract vecotr as element in vector space and $\underline{\vec{x}}$ for vector in coordinate space, $\ket{\Psi}$ an abstract element in Hilbert space, $\ket{\Psi_0}$ the ground state of Hydrogen and $\ket{\Psi_n}$ the Eigenstates of the field free Hamiltonian.
This chapter mainly follows \cite{Ivanov20012005} 



%%%%%%%%%%%%%%%%%%%%%
\section{Basic Formalism}
Our goal is to come up with a expression were we can 


\subsection{Schrödinger Equation}

Basic Definitions of schröfinger qe, light dyson series, and strong field s matrix

We want the time evolution of a quantum system in the presence of an external time dependent field in order to describe the strong field ionization later on.
The time evolution of a quantum system is given by the time dependent Schrödinger equation and a general hamiltonian
\begin{equation}
    i \hbar \frac{\partial}{\partial t} \ket{\Psi(t)} = \hat{H}(t) \ket{\Psi(t)}. \label{eq:schroedinger}
\end{equation}
The formal solution depends on the time dependence of the hamiltonian and the physical setting. 
In the following we assume \footnote{How? Later. No physical setting bzw no approximations yet. Its better to juistify it later but have a working formalism instead of the other way around.} (IMPORTANT) that $[\hat{H}(t), \hat{H}(t')] = 0$ so we assume some sort of quasi static approximation to the Hamiltons time evolution. 
The solution is then given by 
\begin{equation}
    \ket{\Psi(t)} = e^{-\frac{i}{\hbar} \int_{t_0}^{t} \hat{H}(t') dt'} \ket{\Psi(t=0)} = \hat{\U}(t)\ket{\Psi(t=0)}
\end{equation}
Now its time so establish a physical setting. We have Hydrogen Atom with nucleus and electron described by time indepentent Hamilton $\hat{H}_0$. 
The external laser Field is descirbed by an time dependent part $\hat{V}(t)$. To describe the interaction of the atom with the laser field we use in the following the dipole approximation.




%%%%%%%%%%%%%%%%%%%%%
\newpage
\subsection{Light-Matter Interaction}
A light wave is defined by the Maxwell equations
\begin{equation*}
    \begin{aligned}
        \nabla \cdot \vec{E} &= \rho \quad & \nabla \times \vec{E} &= -\frac{\partial \vec{B}}{\partial t} \\
        \nabla \cdot \vec{B} &= 0 \quad & \nabla \times \vec{B} &= \vec{J} + \frac{\partial \vec{E}}{\partial t}
    \end{aligned}
\end{equation*}
The Maxwell equations are being solved by
\begin{equation}
    \begin{aligned}
        \vec{E} &= -\nabla \varphi - \frac{\partial \vec{A}}{\partial t}\\ \label{eq:potentials}
        \vec{B} &= \nabla \times \vec{A}
    \end{aligned}
\end{equation}
For these solutions we introduced the vector potential $\vec{A}(\uvec{x}, t)$ and the scalar potential $\varphi(t)$. 
These are not unique such that different choices can result in the same physical setting. In general
\begin{equation*}
    \begin{aligned}
        \vec{A} &\to \vec{A} + \nabla \chi \\
        \varphi &\to \varphi - \frac{\partial \chi }{\partial t}   
    \end{aligned}
\end{equation*}
also fulfill the Maxwell equations while $\chi(t)$ is an arbitrary smooth scalar function. The arbitrariness of $\chi$ is known as gauge freedom and a direct consequence of the Maxwell equations.
Choosing a gauge (i.e., a specific $\chi$) is a matter of convenience and can be used to simplify the calculations as presented in the following.




\subsection{Dipole Approximation}
Very important approximation. 
The dipole approximation is valid when the wavelength of the optical field is much larger than both the size of the relevant bound electron states and the maximum displacement of a free electron during the light-matter interaction. 
Additionally, it assumes that the magnetic field of the light has a negligible effect on the electron's motion, meaning the velocities of the charged particles must be nonrelativistic. \\
To see where exaclty one makes this assumption, first we rewrite the Maxwell equations in the dependence of the vector potential and the scalar potential as defined in \eqref{eq:potentials}. 
This will result in two coupled differential equations, what does not bring us any further. 
However we are interestet in making a simple expression for the vector potential $\vec{A}$.
We achieve this by choosing a certain gauge, the so called Lorentz gauge
\begin{equation*}
    \partial_{\mu} \vec{A}^{\mu} = 0 \quad \text{or} \quad \nabla \cdot \vec{A} + \frac{\partial \varphi}{\partial t} = 0
\end{equation*}
This can be achieved by solving the inhomogenous wave equation for $\chi$ that comes up when doing this calculation explicitly and is possible when $\vec{A}$ and $\varphi$ are know.
Now the Maxwell equations are uncoupled and can be written as
\begin{equation*}
    \begin{aligned}
        \nabla^2 \varphi - \frac{\partial^2 \varphi}{\partial t^2} &= \rho \\
        \nabla^2 \vec{A} - \frac{\partial^2 \vec{A}}{\partial t^2} &= \vec{J} 
    \end{aligned}
\end{equation*}
We are mainly interested in the second equation. The equation is known as the wave equation therefore $\vec{A}$ describes plane waves
\begin{equation*}
    \vec{A}(\uvec{x}, t) = \vec{A}_0 \e^{i (\uvec{k} \cdot \uvec{x} - \omega t)}
\end{equation*}
The dipole approximation is mathemaically speaking just the leading term in Taylor expansion of $\e^{i\uvec{k} \cdot \uvec{x}}$. The vector potential is therefore independent of the spatial coordinates and can be written as
\begin{equation*}
    \vec{A}(\uvec{x}, t) \approx \vec{A}_0 \e^{- i \omega t} = \vec{A}(t)
\end{equation*}
In other words
\begin{equation*}
    \vec{B} = \nabla \times \vec{A} \approx 0
\end{equation*}
Therefore we can rewirte the time dependent part $\hat{V}$ of our Hamiltonian (HOW??) as
\begin{equation*}
    \hat{V}(t) = -\hat{\vec{d}} \cdot \vec{E}(t)
\end{equation*}
where $\hat{\vec{d}}$ is the dipole operator and $\vec{E}(t)$ is the electric field.




%%%%%%%%%%%%%%%%%%%
\newpage
\section{Strong Field Approximation}
%clear defintition of the strong field approximation, and the assumptions that are made.

For making the strong field approximation we first have to obtain a poitn where is its good to use. 
When we treat $\hat{V}(t)$ as the interaction term, we can write an exakt solution to \eqref{eq:schroedinger}
\begin{equation}
    \ket{\Psi(t)} = -i \int_{t_0}^{t} dt' e^{-\frac{i}{\hbar} \int_{t'}^{t} \hat{H}(t'') dt''} \hat{V}(t') e^{-\frac{i}{\hbar} \int_{t_0}^{t'} \hat{H}(t'') dt''} \ket{\Psi(t=0)} + e^{-\frac{i}{\hbar} \int_{t_0}^{t} \hat{H}(t') dt'} \ket{\Psi(t=0)}
\end{equation}
as can be checked by inserting the solution into the Schrödinger equation using the parameter Integral trick.





%%%%%%%%%%%%%%%%%%%%%
\newpage
\section{Strong Field Ionization}

Derivation of 

\begin{equation}
    \lim_{t \to \infty} \ket{\Psi(t)}  = -i \int d^3 p\, \ket{\vec{p}} \int_{-\infty}^{\infty} dt'\, e^{-\frac{i}{2}\int_{t'}^{\infty} [\vec{p}+\vec{A}(t')]^2 \, dt'} e^{i I_\mathrm{p} t'} \langle \vec{p} + \vec{A}(t') | \hat{\vec{d}} \cdot \vec{E}(t') | \Psi_0 \rangle
\end{equation}





%%%%%%%%%%%%%%%%%%%%
\newpage
\section{Multiphoton Ionization}

Different types of Ionization, tunneling Ionization, multiphoton