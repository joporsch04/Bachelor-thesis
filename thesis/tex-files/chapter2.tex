Convention: \\
$\Psi$ wavefunction for the whole system  \\
$\psi(\uvec{x})$ for a wavefunction in position space without choosing explicit coordinates, \\
$\phi(\uvec{p})$  for a wavefunction in momentum space, \\
$\vec{A}$ for abstract vector as element in vector space, \\
$\uvec{x}$ for vector in $\R^n$ \\
$\ket{\Psi}$ an abstract element in Hilbert space $\Hc$, \\
$\ket{\Phi}$ for the abstract Eigenstates of the whole Hamiltonian, \\
$Y_{l, m}(\theta, \phi) = \braket{\theta, \phi | l, m}$ definition of spherical harmonics, \\
$\psi_{n, l, m}(r, \theta, \phi)$ for the wavefunction of hydrogen in spherical coordinates, with $\uvec{x}=(r, \theta, \phi)$ \\
I use underlined vectors when they are the coordinates and bold vectors when they are abstract elements in a vector space. \\
The canonical momentum $\uvec{P}$ parametrises the phase space but the kinetic momentum does not so kinetic momentum $\hat{=} \vec{p}$.\\
The position and momentum operator are in boldface $\hat{\vec{x}}$ because they do not choose any kind of basis not even a representation in which they are displayed.\\
When i use $\ket{\vec{k}}$ I mean a plane wave solution so "special" a continuum state.


This chapter mainly follows \cite{Ivanov20012005} with some modifications.



%%%%%%%%%%%%%%%%%%%%%
\section{Basic Formalism}
Our goal is to come up with a expression were we can 


\subsection{Schrödinger Equation}

Basic Definitions of schröfinger qe, light dyson series, and strong field s matrix

We want the time evolution of a quantum system in the presence of an external time dependent field in order to describe the strong field ionization later on.
The time evolution of a quantum system is given by the time dependent Schrödinger equation and a general hamiltonian
\begin{equation}
    i \frac{\partial}{\partial t} \ket{\Psi(t)} = \hat{\Hs}(t) \ket{\Psi(t)}. \label{eq:schroedinger}
\end{equation}
The formal solution depends on the time dependence of the hamiltonian and the physical setting. 
In the following we assume \footnote{How? Later. No physical setting bzw no approximations yet. Its better to juistify it later but have a working formalism instead of the other way around.} (IMPORTANT) that $[\hat{\Hs}(t), \hat{\Hs}(t')] = 0$ so we assume some sort of quasi static approximation to the Hamiltons time evolution. 
The solution is then given by 
\begin{equation}
    \ket{\Psi(t)} = e^{-i \int_{t_0}^{t} \hat{\Hs}(t') dt'} \ket{\Psi(t=0)} = \hat{\U}(t)\ket{\Psi(0)}
\end{equation}
Now its time so establish a physical setting. We have Hydrogen Atom with nucleus and electron described by time indepentent Hamilton $\hat{\Hs}_0$. 
The external laser Field is described by an time dependent part $\hat{V}_I(t)$. To describe the interaction of the atom with the laser field we use in the following the dipole approximation.\\
There are two ways of representing $\ket{\Psi(t)}$, schroedinger and interaction picture. Coefficients.





%%%%%%%%%%%%%%%%%%%%%
\newpage
\subsection{Light-Matter Interaction}
A light wave is defined by the Maxwell equations
\begin{equation*}
    \begin{aligned}
        \nabla \cdot \vec{E} &= \rho \quad & \nabla \times \vec{E} &= -\frac{\partial \vec{B}}{\partial t} \\
        \nabla \cdot \vec{B} &= 0 \quad & \nabla \times \vec{B} &= \vec{J} + \frac{\partial \vec{E}}{\partial t}
    \end{aligned}
\end{equation*}
The Maxwell equations are being solved by
\begin{equation}
    \begin{aligned}
        \vec{E} &= -\nabla \varphi - \frac{\partial \vec{A}}{\partial t}\\ \label{eq:potentials}
        \vec{B} &= \nabla \times \vec{A}
    \end{aligned}
\end{equation}
For these solutions we introduced the vector potential $\vec{A}(\uvec{x}, t)$ and the scalar potential $\varphi(\uvec{x}, t)$. 
These are not unique such that different choices can result in the same physical setting. In general
\begin{equation*}
    \begin{aligned}
        \vec{A} &\to \vec{A} + \nabla \chi \\
        \varphi &\to \varphi - \frac{\partial \chi }{\partial t}   
    \end{aligned}
\end{equation*}
also fulfill the Maxwell equations while $\chi(t)$ is an arbitrary smooth scalar function. The arbitrariness of $\chi$ is known as gauge freedom and a direct consequence of the Maxwell equations.
Choosing a gauge (i.e., a specific $\chi$) is a matter of convenience and can be used to simplify the calculations as presented in the following.








%%%%%%%%%%%%%%%%%%%%%
\subsection{Dipole Approximation}
Very important approximation. 
The dipole approximation is valid when the wavelength of the optical field is much larger than both the size of the relevant bound electron states and the maximum displacement of a free electron during the light-matter interaction. 
Additionally, it assumes that the magnetic field of the light has a negligible effect on the electron's motion, meaning the velocities of the charged particles must be nonrelativistic. \\
To see where exaclty one makes this assumption, first we rewrite the Maxwell equations in the dependence of the vector potential and the scalar potential as defined in \eqref{eq:potentials}. 
This will result in two coupled differential equations, what does not bring us any further. 
However we are interestet in making a simple expression for the vector potential $\vec{A}$.
We achieve this by choosing a certain gauge, the so called Lorentz gauge
\begin{equation*}
    \partial_{\mu} \vec{A}^{\mu} = 0 \quad \text{or} \quad \nabla \cdot \vec{A} + \frac{\partial \varphi}{\partial t} = 0
\end{equation*}
This can be achieved by solving the inhomogenous wave equation for $\chi$ that comes up when doing this calculation explicitly and is possible when $\vec{A}$ and $\varphi$ are know.
Now the Maxwell equations are uncoupled and can be written as
\begin{equation*}
    \begin{aligned}
        \nabla^2 \varphi - \frac{\partial^2 \varphi}{\partial t^2} &= \rho \\
        \nabla^2 \vec{A} - \frac{\partial^2 \vec{A}}{\partial t^2} &= \vec{J} 
    \end{aligned}
\end{equation*}
We are mainly interested in the second equation. The equation is known as the wave equation therefore $\vec{A}$ describes plane waves
\begin{equation*}
    \vec{A}(\uvec{x}, t) = \vec{A}_0 \e^{\pm i (\uvec{k} \cdot \uvec{x} - \omega t)}
\end{equation*}
The dipole approximation is mathemaically speaking just the leading term in Taylor expansion of $\e^{i\uvec{k} \cdot \uvec{x}}$. 
The vector potential is therefore independent of the spatial coordinates and can be written as
\begin{equation*}
    \vec{A}(\uvec{x}, t) = \vec{A}_0\e^{\mp i \omega t}\mathrm{exp}\left\{\pm2\pi i\frac{|\uvec{x}|}{\lambda}\uvec{e}_k \cdot \uvec{e}_x\right\} \approx \vec{A}_0 \e^{\mp i \omega t}\left(1+\Os\left(\frac{|\uvec{x}|}{\lambda}\right)\right) = \vec{A}(t)
\end{equation*}
As long as the Wavelength is big enough this approximation is valid. It follows:
\begin{equation*}
    \vec{B} = \nabla \times \vec{A} \approx 0
\end{equation*}
Even tough we will later choose another gauge, the physics in our system remains the same. The dipole Approximation is not gauge dependent, so in another gauge $\vec{B}$ remains approximately zero.
Choosing the lorentz gauge here is just a matter of convenience, because just expanding the vector potential to the linear term is very intuitive. \\
This was the essence of the dipole approximation but we also want a intuitive expression for our Laser Fild in the Hamiltonian.
It will be helpfull to look at the semi classical Hamilton function of a free electron in an electric field\footnote{Because its interesting it will be derived in \ref{sec:semiclassichamilton}}: % what is with V(x), and the mass of the proton. is it in \vec{p}?
\begin{equation} %what are the arguments in H, is it even x? -> classical hamilton with H(P, x, t) but now quantum without because P is now nabla operator
    \hat{\Hs}(\uvec{x}, t) = \frac{1}{2m}(\vec{\hat{P}} - e \vec{A}(\uvec{x}, t))^2 + e \varphi(\uvec{x}, t)  \label{eq:semiClassicalHamilton}    %+ V(x)
\end{equation}
In the dipole Approximation, this can be simplified to:
\begin{equation*}
    \hat{\Hs}(\uvec{x}, t) = \frac{\vec{\hat{P}}^2}{2m} + \frac{e}{m}\vec{\hat{P}}\cdot \vec{A}(t) + \frac{e^2}{2m}\vec{A}^2(t) - e \varphi(\uvec{x}, t) %+ V(x)
\end{equation*}
Note that we could set $\varphi$ to zero because the source of the em wave are outside of our region of interest but the dipole approximation can be made without this assumption. 
We will however set $\varphi$ to zero later. 
Another general assumption one made when working with semi classical Hamiltonians is that only the vectorpotential causes the electron to change its state but not vice versa (Bosßmann). 
This is reasonable approximation because in our case the intensity of the Laser is sufficiently high, so we dont have to worry about that(is it really??).
Now we perform another gauge transformation to the so called length gauge via 
\begin{equation*}
    \chi = -\vec{A}(t)\cdot \uvec{x} %r=(x/3, y/3, z/3)
\end{equation*}
This gauge sets $\vec{A}$ to zero, and $\varphi$ will have the following form:
\begin{equation*}
    \nabla \varphi \to \nabla \cdot (\varphi + \vec{x} \cdot \frac{\partial \vec{A}}{\partial t}) = \nabla \varphi + \frac{\partial \vec{A}}{\partial t} = - \vec{E}
\end{equation*}
Integrating this equation from the origin to $\vec{x}$ gives us the electric potential in the length gauge. Furthermore, $\vec{r}$ is now quantized and our Hamilton therefore reads:
\begin{equation*}
    \hat{\Hs}(\uvec{x}, t) = \frac{\hat{\vec{P}}^2}{2m} - e \hat{\vec{x}} \cdot \vec{E} %+ V(x)
\end{equation*}
We can rewirte the time dependent part $\hat{V}$ of our quantum mechanical Hamiltonian as
\begin{equation}
    \hat{V}_I(t) = -\hat{\vec{d}} \cdot \vec{E}(t) \label{eq:dipoleApprox}
\end{equation}
where $\hat{\vec{d}}=e\vec{\hat{x}}$ is the dipole operator and $\vec{E}(t)$ is the electric field.

%breakdown of dipole approx-> for fast electrons also not only for small wavelengths!!!






%%%%%%%%%%%%%%%%%%%
\newpage
\section{Strong Field Approximation}
%clear defintition of the strong field approximation, and the assumptions that are made.

For making the strong field approximation we first have to obtain a point where is its good to use. 
When we treat $\hat{V}_I(t)$ as the interaction term, we can write an exakt solution to \eqref{eq:schroedinger}:
\begin{equation}
    \ket{\Psi(t)} = -i \int_{t_0}^{t} \dd t' e^{-i \int_{t'}^{t} \hat{\Hs}(t'') \dd t''} \hat{V}_I(t') e^{-i \int_{t_0}^{t'} \hat{\Hs_0}(t'') \dd t''} \ket{\Psi(0)} + e^{-i \int_{t_0}^{t} \hat{\Hs_0}(t') \dd t'} \ket{\Psi(0)} \label{eq:schroedingerSolPart}
\end{equation}
as can be checked by inserting the solution into the Schrödinger equation using the parameter Integral trick. 
To make this expression more appealing we can project it into an eigenstate in the continuum ie a electron state characterized by its velocity.
As can be seen in the following, the last term in \eqref{eq:schroedingerSolPart} is now gone because there can be no overlap between a continuum state and the initial state.
\begin{equation}
    \braket{\Pi(t_c)|\Psi(t)} = -i \int_{t_0}^{t} \dd t' \bra{\Pi(t=t_c)} e^{-i \int_{t'}^{t} \hat{\Hs}(t'') \dd t''}\hat{V}_I(t') e^{-i \int_{t_0}^{t'} \hat{\Hs_0} \dd t''} \ket{\Psi(0)}
\end{equation}
With $\ket{\Pi(t_c)}$ being a continuum state with momentum $\uvec{p}$ at time $t_c$ with $t_c$ being sufficiently big enough for the electron to be in the continuum ($t_c >> t'$). 
Equation \eqref{eq:schroedingerSolPart} is known as the strong field S-matrix amplitude. 
It is best to read this equation from right to left, starting with the initial state of our system and the field free time propagation of the system. 
At moment $t'$ the Laser starts to interact with the system and it transisions into a virtual state instantaniously. 
Note that $\ket{\Psi(0)}$ is not the ground state of the Hydrogen atom, just the initial state of the system at $t=0$.
From time $t'$ to the observed time $t$ the system is described by the full Hamiltonian including both Laser Field and the binding potential.
Now we have a good place to start with the strong field approximation. 
In principle, SFA is the neglecting of the Coulomb potential once the electron is in the continuum because the Laser Field is now the dominant force acting on the electron.
We can therefore write the term with the full Hamiltonian as
\begin{equation*}
    \begin{aligned}
        %\hat{\Hs}(t) &= \hat{\Hs}_0 + \hat{V}_I(t) = \hat{\Hs}_p + \hat{V}_C + \hat{V}_I(t) 
        e^{-i \int_{t'}^{t} \hat{\Hs}(t'') \dd t''} &\approx e^{-i \int_{t'}^{t} \hat{\Hs}_{SFA}(t'') \dd t''} \quad \text{with} \quad \hat{\Hs}_{SFA}(t) &= \hat{\Hs}(t) - \hat{V}_C
    \end{aligned}
\end{equation*}
This is very usefull because for $\hat{\Hs}_{SFA}$ we know an exact analytical solution. For that lets take another look at the semi classical Hamilton \eqref{eq:semiClassicalHamilton}. Classically, the momentum operator $\hat{\vec{P}}$ is known as the canonical momentum and is given by:
\begin{equation*}
    \frac{\partial \Ls}{\partial \dot{\uvec{x}}} = \uvec{P} = m\uvec{\dot{x}} + \frac{e}{c}\vec{A} \stackrel{\mathrm{a.u.}}{=} \vec{p} + \vec{A}
\end{equation*}
In our case the canonical momentum is conserved. To see this, lets finally set $\varphi=0$ so we have $\vec{E} = -\frac{\partial \vec{A}}{\partial t}$ as justified above and recall the equation of motion for a charged particle in an electromagnetic field \cite{LandauLifschitzBand2}:
\begin{equation*}
    \frac{\dd\vec{p}}{\dd t} = \vec{E} + \left[\uvec{\dot{x}}\times\vec{B}\right] \approx -\frac{\partial \vec{A}}{\partial t} = -\frac{\dd \vec{A}}{\dd t}   %why would i use uvec isntead of vec? doesnt make sense => with x its necessary because its natural variable of lagrandian
\end{equation*}
And also the energy of the system is clear:
\begin{equation*}
    E(t) = \dot{\uvec{x}}\frac{\partial \Ls}{\partial \dot{\uvec{x}}}-\Ls = \frac{\vec{p}^2}{2}
\end{equation*}
Note that the energy is not conserved because the argument we made bevor does not hold for the kinetic momentum only for the canonical momentum.
Now comes the interesting part. Clearly $\ket{\Pi(t_c)}$ is a solution of $e^{-i \int_{t'}^{t} \hat{\Hs}_{SFA}(t'') \dd t''}$ so that gives us:
\begin{equation*}
    e^{-i \int_{t'}^{t} \hat{\Hs}_{SFA}(t'') \dd t''}\ket{\Pi(t_c)} = e^{-i \int_{t'}^{t} (\uvec{P} - \vec{A}(t''))^2 \dd t''}\ket{\Pi(t_c)}
\end{equation*}
$\uvec{P}$ is of course independent of time, but $\vec{A}$ is not. 
Since dealing with canonical momentum in numerical simulations is not very convenient, we use the fact that its conserved and calculate the momentum at other times.
In particular we are interested in times where the laser field is long gone:
\begin{equation*}
    \uvec{P} = \vec{p}(t'') + \vec{A}(t'') = \vec{p}(t\rightarrow \infty) + \vec{A}(t\rightarrow \infty) = \vec{p}
\end{equation*}
Furthermore (how??)
\begin{equation*}
    \ket{\Pi} = \ket{p+A}
\end{equation*}
\begin{equation}
    \braket{\Xi(t_c)|\Psi(t)} = -i \int_{t_0}^{t} \dd t' e^{-i \int_{t'}^{t} \hat{\Hs}(t'') \dd t''}e^{-i \int_{t_0}^{t'} \hat{\Hs_0} \dd t''} \bra{\Xi(t=t_c)}|\vec{\hat{d}}\cdot\vec{E}(t')|\ket{\Psi(0)}
\end{equation}
This is the equation where most papers start with \cite{Theory_NPS}.\\\\
Note that SFA is not about strong laser pulses since here we are also dealing with small ionisation proabilities (<0.01) so SFA states that when ionisation does happen (regardless if its unlikely) the laser pulse will then be the dominant force.


I need to derive in paper from manoram 2023 the same thing as app A just instead of $\hat{p}$ i use $\hat{1}$ and use instead of $\ket{\Psi}_0$ i use the expansion in eigenstates from my project plan


%%%%%%%%%%%%%%%%%%%%
\newpage
\section{Derivation of SFA Rate}
For the derivation of the SFA rate we need to think more about the groudn state ...\\\\
The ground state is now a superosition $\sum_{n}c_n(t)\ket{\Psi_n}$.\\
We can write the SFA rate as:
\begin{equation*}
    \braket{\Psi(t)|\hat{\vec{1}}|\Psi(t)} = \int_{\R} \Gamma(t) \dd t = 
\end{equation*}




%%%%%%%%%%%%%%%%%%%%%
\newpage
\section{Strong Field Ionization}
Decied to do Phenomenology after theory.\\
Phenomenology of strong field ionization, Different types of Ionization, tunneling Ionization, multiphoton, stark effect.

