Convention: \\
$\Psi$ wavefunction for the whole system  \\
$\psi(\uvec{x})$ for a wavefunction in position space without choosing explicit coordinates, \\
$\phi(\uvec{p})$  for a wavefunction in momentum space, \\
$\vec{A}$ for abstract vector as element in vector space, \\
$\uvec{x}$ for vector in $\R^n$ \\
$\ket{\Psi}$ an abstract element in Hilbert space $\Hc$, \\
$\ket{\Phi}$ for the abstract Eigenstates of the whole Hamiltonian, \\
$Y_{l, m}(\theta, \phi) = \braket{\theta, \phi | l, m}$ definition of spherical harmonics, \\
$\psi_{n, l, m}(r, \theta, \phi)$ for the wavefunction of hydrogen in spherical coordinates, with $\uvec{x}=(r, \theta, \phi)$ \\
I use underlined vectors when they are the coordinates and bold vectors when they are abstract elements in a vector space. \\
The canonical momentum $\uvec{P}$ parametrises the phase space but the kinetic momentum does not so kinetic momentum $\hat{=} \vec{p}$.\\
The position and momentum operator are in boldface $\hat{\vec{x}}$ because they do not choose any kind of basis not even a representation in which they are displayed.\\
When i use $\ket{\vec{k}}$ I mean a plane wave solution so "special" a continuum state.\\
A $\cdot$ denotes a scalar product between two vectors, $\times$ is just normal multiplication. \\


The strucutre in this chapter mainly follows \cite{Ivanov20012005} with some modifications.



%%%%%%%%%%%%%%%%%%%%%
\newpage
\section{Basic Formalism}
Our goal is to come up with a expression were we can use the strong field approximation effectively.
We want the time evolution of a quantum system in the presence of an external time dependent field in order to describe the strong field ionization later on.
For that we first need to solve the schroedinger equation for a given Hamiltonian.\\
We will come across some difficulties like gauge dependence.

\subsection{Schrödinger Equation}
The time evolution of a quantum system is given by the time dependent Schrödinger equation and a general hamiltonian
\begin{equation}
    i \frac{\partial}{\partial t} \ket{\Psi(t)} = \hat{\Hs}(t) \ket{\Psi(t)}. \label{eq:schroedinger}
\end{equation}
The formal solution depends on the time dependence of the hamiltonian and the physical setting. 
With no further assumptions about our Hamiltonian becasue $[\hat{\Hs}(t), \hat{\Hs}(t')] \neq 0$ we can write the formal solution to \eqref{eq:schroedinger} as a Dyson series:
The solution is then given by 
\begin{equation}
    \ket{\Psi(t)} = \hat{\U}(t)\ket{\Psi(0)} = \hat{1} + \sum_{n=1}^{\infty} (-i)^n \int_{t_0}^{t} \dd t_1 \int_{t_0}^{t_1} \dd t_2 \cdots \int_{t_0}^{t_{n-1}} \dd t_n \hat{\Hs}(t_n) \hat{\Hs}(t_{n-1})\cdots\hat{\Hs}(t_1) \ket{\Psi(0)}. \label{eq:dyson}
\end{equation}
% Now its time so establish a physical setting. We have Hydrogen Atom with nucleus and electron described by time indepentent Hamilton $\hat{\Hs}_0$. 
% The external laser Field is described by an time dependent part $\hat{V}(t)$. To describe the interaction of the atom with the laser field we use in the following the dipole approximation.\\
% There are two ways of representing $\ket{\Psi(t)}$, schroedinger and interaction picture. Coefficients.
\subsection{Interaction picture and Projection operators}
Solving the schroedinger equation can be cumbersome, especially when we have to deal with a time dependent Hamiltonian that doesnt commutate with itself at different times.
To make the calculations easier we can use projection operators with a method called feshbach method within the interaction picture.




%%%%%%%%%%%%%%%%%%%%%
%\newpage
\subsection{Light-Matter Interaction}
A light wave is defined by the Maxwell equations
\begin{equation*}
    \begin{aligned}
        \nabla \cdot \vec{E} &= \rho \quad & \nabla \times \vec{E} &= -\frac{\partial \vec{B}}{\partial t} \\
        \nabla \cdot \vec{B} &= 0 \quad & \nabla \times \vec{B} &= \vec{J} + \frac{\partial \vec{E}}{\partial t}
    \end{aligned}
\end{equation*}
The Maxwell equations are being solved by
\begin{equation}
    \begin{aligned}
        \vec{E} &= -\nabla \varphi - \frac{\partial \vec{A}}{\partial t}\\ \label{eq:potentials}
        \vec{B} &= \nabla \times \vec{A}
    \end{aligned}
\end{equation}
For these solutions we introduced the vector potential $\vec{A}(\uvec{x}, t)$ and the scalar potential $\varphi(\uvec{x}, t)$. 
These are not unique such that different choices can result in the same physical setting. In general
\begin{equation*}
    \begin{aligned}
        \vec{A} &\to \vec{A} + \nabla \chi \\
        \varphi &\to \varphi - \frac{\partial \chi }{\partial t}   
    \end{aligned}
\end{equation*}
also fulfill the Maxwell equations while $\chi(t)$ is an arbitrary smooth scalar function. The arbitrariness of $\chi$ is known as gauge freedom and a direct consequence of the Maxwell equations.
Choosing a gauge (i.e., a specific $\chi$) is a matter of convenience and can be used to simplify the calculations as presented in the following.








%%%%%%%%%%%%%%%%%%%%%
\subsection{Dipole Approximation}
Very important approximation. 
The dipole approximation is valid when the wavelength of the optical field is much larger than both the size of the relevant bound electron states and the maximum displacement of a free electron during the light-matter interaction. 
Additionally, it assumes that the magnetic field of the light has a negligible effect on the electron's motion, meaning the velocities of the charged particles must be nonrelativistic. \\
To see where exaclty one makes this assumption, first we rewrite the Maxwell equations in the dependence of the vector potential and the scalar potential as defined in \eqref{eq:potentials}. 
This will result in two coupled differential equations, what does not bring us any further. 
However we are interestet in making a simple expression for the vector potential $\vec{A}$.
We achieve this by choosing a certain gauge, the so called Lorentz gauge
\begin{equation*}
    \partial_{\mu} \vec{A}^{\mu} = 0 \quad \text{or} \quad \nabla \cdot \vec{A} + \frac{\partial \varphi}{\partial t} = 0
\end{equation*}
This can be achieved by solving the inhomogenous wave equation for $\chi$ that comes up when doing this calculation explicitly and is possible when $\vec{A}$ and $\varphi$ are know.
Now the Maxwell equations are uncoupled and can be written as
\begin{equation*}
    \begin{aligned}
        \nabla^2 \varphi - \frac{\partial^2 \varphi}{\partial t^2} &= \rho \\
        \nabla^2 \vec{A} - \frac{\partial^2 \vec{A}}{\partial t^2} &= \vec{J} 
    \end{aligned}
\end{equation*}
We are mainly interested in the second equation. The equation is known as the wave equation therefore $\vec{A}$ describes plane waves
\begin{equation*}
    \vec{A}(\uvec{x}, t) = \vec{A}_0 \e^{\pm i (\uvec{k} \cdot \uvec{x} - \omega t)}
\end{equation*}
The dipole approximation is mathemaically speaking just the leading term in Taylor expansion of $\e^{i\uvec{k} \cdot \uvec{x}}$. 
The vector potential is therefore independent of the spatial coordinates and can be written as
\begin{equation*}
    \vec{A}(\uvec{x}, t) = \vec{A}_0\e^{\mp i \omega t}\mathrm{exp}\left\{\pm2\pi i\frac{|\uvec{x}|}{\lambda}\uvec{e}_k \cdot \uvec{e}_x\right\} \approx \vec{A}_0 \e^{\mp i \omega t}\left(1+\Os\left(\frac{|\uvec{x}|}{\lambda}\right)\right) = \vec{A}(t)
\end{equation*}
As long as the Wavelength is big enough this approximation is valid. It follows:
\begin{equation*}
    \vec{B} = \nabla \times \vec{A} \approx 0
\end{equation*}
Even tough we will later choose another gauge, the physics in our system remains the same. The dipole Approximation is not gauge dependent, so in another gauge $\vec{B}$ remains approximately zero.
Choosing the lorentz gauge here is just a matter of convenience, because just expanding the vector potential to the linear term is very intuitive. \\
This was the essence of the dipole approximation but we also want a intuitive expression for our Laser Fild in the Hamiltonian. 
For that we need to think more about the gauge of our system.






%%%%%%%%%%%%%%%%%%%%%
\subsection{Gauges}
What makes calculating ionisation rates in strong field physiccs so difficult is the gauge, ie deciding which one you want to choose and when. \\
First, I will derive two basic expressions for the Hamiltonian in the so called velocity gauge and length gauge using the dipole approximation.
It will be helpfull to look at the semi classical Hamilton function of a free electron in an electric field\footnote{Because its interesting it will be derived in \ref{sec:semiclassichamilton}}: % what is with V(x), and the mass of the proton. is it in \vec{p}?
\begin{equation} %what are the arguments in H, is it even x? -> classical hamilton with H(P, x, t) but now quantum without because P is now nabla operator
    \hat{\Hs}(\uvec{x}, t) = \frac{1}{2m}(\vec{\hat{P}} - e \vec{A}(\uvec{x}, t))^2 + e \varphi(\uvec{x}, t)  \label{eq:semiClassicalHamilton}    %+ V(x)
\end{equation}
In the dipole Approximation, this can be simplified to:
\begin{equation*}
    \hat{\Hs}(\uvec{x}, t) = \frac{\vec{\hat{P}}^2}{2m} + \frac{e}{m}\vec{\hat{P}}\cdot \vec{A}(t) + \frac{e^2}{2m}\vec{A}^2(t) - e \varphi(\uvec{x}, t) %+ V(x)
\end{equation*}
Note that we could set $\varphi$ to zero because the source of the em wave are outside of our region of interest but the dipole approximation can be made without this assumption. 
We will however set $\varphi$ to zero later. 
Another general assumption one made when working with semi classical Hamiltonians is that only the vectorpotential causes the electron to change its state but not vice versa (Bosßmann). 
This is reasonable approximation because in our case the intensity of the Laser is sufficiently high, so we dont have to worry about that(is it really??).
Now we perform our desired gauge transformation, called length gauge via:
\begin{equation*}
    \chi = -\vec{A}(t)\cdot \uvec{x} %r=(x/3, y/3, z/3)
\end{equation*}
This gauge sets $\vec{A}$ to zero, and $\varphi$ will have the following form:
\begin{equation*}
    \nabla \varphi \to \nabla \cdot (\varphi + \vec{x} \cdot \frac{\partial \vec{A}}{\partial t}) = \nabla \varphi + \frac{\partial \vec{A}}{\partial t} = - \vec{E}
\end{equation*}%i think E is still dependent on x so i cant just integrate it :( becasue phi is still dependent on x but i think i can solve it by just setting phi to zero
Integrating this equation from the origin to $\vec{x}$ gives us the electric potential in the length gauge. Furthermore, $\vec{r}$ is now quantized and our Hamilton therefore reads:
\begin{equation*}
    \hat{\Hs}(\uvec{x}, t) = \frac{\hat{\vec{P}}^2}{2m} - e \hat{\vec{x}} \cdot \vec{E} %+ V(x)
\end{equation*}
We can rewirte the time dependent part $\hat{V}$ of our quantum mechanical Hamiltonian as
\begin{equation}
    \hat{V}_I(t) = -\hat{\vec{d}} \cdot \vec{E}(t) \label{eq:dipoleApprox}
\end{equation}
where $\hat{\vec{d}}=e\vec{\hat{x}}$ is the dipole operator and $\vec{E}(t)$ is the electric field.\\
This is a common way to write the interaction Hamiltonian. 
However if we choose another gauge transformation, the equations will have a different form. look bossmann!!!

%breakdown of dipole approx-> for fast electrons also not only for small wavelengths!!!






%%%%%%%%%%%%%%%%%%%
\newpage
\section{Strong Field Approximation}
%clear defintition of the strong field approximation, and the assumptions that are made.
The difficulty with ionisation arises because we now have in some sense two Hilbert spaces, one for the states in the Hydrogen atom that deals with some distortion of the wavefunction because of the Laser field and one for the continuum states that are affected mainly by the Laser field but also by the binding potential.\\
We will see SFA will take care about the second Hilbertspace and make it easier. 
The main goal of this thesis is to see how much the Laser field has an effect on the wavefunction before ionisation happens. 
Previous work neglects this part so there is just one hilbertspace for the eigenstates unaffectet by the laser field and one hilbertspace for the continuum states that are unaffectet by the binding potential.






%%%%%%%%%%%%%%%%%%%%%
\subsection{Subspaces}
First we project the full timedependent Hamiltonian $\hat{\Hs}(t)$ onto subspaces using projection operators defined by:
\begin{equation*}
    \hat{X} = \sum_{n}\ket{\Psi_n}\bra{\Psi_n}\quad\text{and}\quad\hat{Y}= \hat{1} - \hat{X}
\end{equation*}
With $\ket{\Psi_n}$ being the bound states of our atom. We can write:
\begin{equation*}
    \hat{\Hs}(t) = \underbrace{\hat{X}\hat{\Hs}(t)\hat{X}}_{\hat{\Hs}^{\mathrm{X}}(t)=\hat{\Hs}_{\mathrm{0}}(t)} + \underbrace{\hat{Y}\hat{\Hs}(t)\hat{Y} + \hat{X}\hat{\Hs}(t)\hat{Y} + \hat{Y}\hat{\Hs}(t)\hat{X}}_{\hat{\Hs}^{\mathrm{Y}}(t)+\hat{\Hs}^{\mathrm{XY}}(t)+\hat{\Hs}^{\mathrm{YX}}(t)=\hat{\Hs}_{\mathrm{I}}(t)}
\end{equation*}
Lets think about what these terms mean. $\hat{\Hs}^{\mathrm{X}}(t)$ can be seen as just our quantum system but with a small distortion by our electric field. 
This part causes effects like Stark shift both in the eigenstates and eigenenergies of our atom. 
As long as the laser pulse is not too strong and no ionisation has happened, we can treat the pulse like a pertubation to the system.
The distortion may be time dependent but the good thing is that we can easily determine the propagator with repsect to $\hat{\Hs}^{\mathrm{X}}(t)$ by using the interaction picture and solving a system of coupled differential equations.
In other words this Hamiltonian determines the time evolution of the first hilbert space, as mentioned above. \\
Lets think about the other terms. For that we need to establish a setting. Say our Atom sits in the ground state and gets ionised. 
Based on this image we can say two from three parts are uneccessary. First $\hat{\Hs}^{\mathrm{Y}}(t)$ plays no role because we project the initial state into the continuum state, and since their can be no overlap between them this part will play no role.
This term actually just describes the evolution of a continuum state that remains a continuum state after the interaction so the laser pulse has happened.
Second $\hat{\Hs}^{\mathrm{XY}}(t)$ plays no role either since we projct the initial (bound) state into the continuum. 
This term would determine the time evolution of a continuum state that recombines with the atom after the interaction has happened.
Obviously thats not what we want. 
The only term that remains is $\hat{\Hs}^{\mathrm{YX}}(t)$ which describes the time evolution of a bound state that gets ionised in the continuum by the laser pulse.
This is the perfect part to later start the strong field approximation with.\\





\subsection{Avoiding Dyson series}
For determining the time evolution we need to be carefull since the whole Hamiltonian doesnt commutate with itself at different times so we need to be exact. 
Since the full Dyson series \eqref{eq:dyson} can be cumbersome to deal with, we choose a different way. 
What helps us is the fact that we can split the Hamiltonian by projecting it into subspaces in two parts of which one can be solved directly as mentioned above. 
Lets first write our ansatz:
\begin{equation}
    \hat{\U}(t,t_0)=\hat{\U}_0(t,t_0)-i\int_{t_0}^{t} \hat{\U}(t,t')\hat{\Hs}_{\mathrm{I}}(t')\hat{\U}_0(t',t_0) \dd t'       \label{eq:dysonAnsatz}
\end{equation}
It is very easy to show that \eqref{eq:dysonAnsatz} is a solution to \eqref{eq:schroedinger}.\\
In our setting we start with the ground state of the atom so the time evolution is given by $\ket{\Psi_n(t)}=\hat{\U}(t,t_0)\ket{\Psi_n(t_0)}$.
To make things even simpler, we project this state into a continuum state $\ket{\Pi(t_{\mathrm{c}})}$ at time $t_{\mathrm{c}}$ with $t_{\mathrm{c}}$ being sufficiently big enough for the electron to be in the continuum ($t_{\mathrm{c}} >> t'$).\\
Of course, there is no overlap between $\ket{\Pi(t_{\mathrm{c}})}$ and $\hat{\U}_0(t,t_0)\ket{\Psi_n(t_0)}$ since the electron did not get ionisaed yet. 
Furthermore, if we expand $\hat{\Hs}_{\mathrm{I}}(t')$ and remind ourself about the orthogonality of the bound states and the continuum states, we see that most of the terms of $\hat{\Hs}_{\mathrm{I}}(t')$ vanish.
We are being left with:
\begin{equation}
    \braket{\Pi(t_{\mathrm{c}})|\Psi_n(t)} = -i \int_{t_0}^{t} \bra{\Pi(t_{\mathrm{c}})}\hat{\U}(t,t')\hat{Y}\hat{\Hs}(t')\hat{X}\hat{\U}_0(t',t_0)\ket{\Psi_n(t_0)} \dd t' \label{eq:smatrixlike}
\end{equation}



%%%%%%%%%%%%%%%%%%
\subsection{Strong Field Approximation}
Before maing the strong field approximation, lets think about \eqref{eq:smatrixlike} again.
It is best to read this equation from right to left, starting with the initial state of our system and the propagation of the system in presence of a weak electric field before ionisation. 
At moment $t'$ the Laser starts to interact with the system and it transisions into a virtual state. 
From time $t'$ to the observed time $t$ the system is described by the full Hamiltonian including both Laser Field and the binding potential.\\
In principle, SFA is the neglecting of exactly this binding potential once the electron is in the continuum because the Laser Field is now the dominant force acting on the electron.
We can therefore write time evolution operator after ionisation as:
\begin{equation*}
    %\hat{\Hs}(t) &= \hat{\Hs}_0 + \hat{V}_I(t) = \hat{\Hs}_p + \hat{V}_C + \hat{V}_I(t) 
    %e^{-i \int_{t'}^{t} \hat{\Hs}(t'') \dd t''} &\approx e^{-i \int_{t'}^{t} \hat{\Hs}_{SFA}(t'') \dd t''} \quad \text{with} \quad \hat{\Hs}_{SFA}(t) &= \hat{\Hs}(t) - \hat{V}_C
    \hat{\U}(t,t') \approx \hat{\U}_{\mathrm{SFA}}(t,t')=e^{-i \int_{t'}^{t} \hat{\Hs}_{SFA}(t'') \dd t''} \quad \text{and} \quad \Hs_{\mathrm{SFA}}(t') = \hat{\Hs}(t) - \hat{V}_{\mathrm{C}}      \label{eq:sfa_formula}
\end{equation*}
This is very usefull because for the eigenstates of $\hat{\Hs}_{SFA}$ we know an exact analytical solution; the Volkov states.
Note that we can explicitely write $\hat{\U}_{\mathrm{SFA}}(t,t')$ in \eqref{eq:sfa_formula} since after ionisation the hamiltonian commutes with itself at different times.
For that lets take another look at the semi classical Hamilton \eqref{eq:semiClassicalHamilton}. 
Classically, the physics driven by the momentum operator $\hat{\vec{P}}$ is known as the canonical momentum and given by:
\begin{equation}
    \frac{\partial \Ls}{\partial \dot{\uvec{x}}} = \uvec{P} = m\uvec{\dot{x}} + \frac{e}{c}\vec{A} \stackrel{\mathrm{a.u.}}{=} \vec{p} + \vec{A}    \label{eq:canonicalMomentum}
\end{equation}
With $\Ls$ being the Lagrangian of the system.
In our case the canonical momentum is conserved. To see this, lets finally set $\varphi=0$ so we have $\vec{E} = -\frac{\partial \vec{A}}{\partial t}$ as justified above and recall the equation of motion for a charged particle in an electromagnetic field \cite{LandauLifschitzBand2}:
\begin{equation*}
    \frac{\dd\vec{p}}{\dd t} = \vec{E} + \left(\uvec{\dot{x}}\times\vec{B}\right) \approx -\frac{\partial \vec{A}}{\partial t} = -\frac{\dd \vec{A}}{\dd t}   %why would i use uvec isntead of vec? doesnt make sense => with x its necessary because its natural variable of lagrandian
\end{equation*}
so $\frac{\dd}{\dd t}\uvec{P}=0$.
And also the energy of the system is clear:
\begin{equation}
    E(t) = \dot{\uvec{x}}\frac{\partial \Ls}{\partial \dot{\uvec{x}}}-\Ls = \frac{\vec{p}^2}{2}     \label{eq:energy}
\end{equation}
Note that the energy is not conserved because the argument we made bevor does not hold for the kinetic momentum only for the canonical momentum.
Now comes the interesting part. Clearly $\ket{\Pi(t_{\mathrm{c}})}$ (not $\ket{\vec{p}(t_{\mathrm{c}})}$!) is a solution of $e^{-i \int_{t'}^{t} \hat{\Hs}_{SFA}(t'') \dd t''}$ so combining \eqref{eq:canonicalMomentum} and \eqref{eq:energy} gives us:
\begin{equation*}
    e^{-i \int_{t'}^{t} \hat{\Hs}_{SFA}(t'') \dd t''}\ket{\Pi(t_c)} = e^{-i \int_{t'}^{t} (\uvec{P} - \vec{A}(t''))^2 \dd t''}\ket{\Pi(t_c)}
\end{equation*}
$\uvec{P}$ is of course independent of time, but $\vec{A}$ is not. 
Since dealing with canonical momentum in numerical simulations is not very convenient, we use the fact that its conserved and calculate the momentum at other times.
In particular we are interested in times where the laser field is long gone:
\begin{equation*}
    \uvec{P} = \vec{p}(t'') + \vec{A}(t'') = \vec{p}(t\rightarrow \infty) + \vec{A}(t\rightarrow \infty) = \vec{p}
\end{equation*}
Furthermore (how??)
\begin{equation*}
    \ket{\Pi} = \ket{\uvec{P}} = \ket{\vec{p}+\vec{A}}
\end{equation*}
Combining all these equations give us the following exprression:
\begin{equation*}
    \braket{\Pi(t_{\mathrm{c}})|\Psi_n(t)} = -i \int_{t_0}^{t} e^{-i \int_{t'}^{t} (\uvec{P} - \vec{A}(t''))^2 \dd t''}\hat{\U}_0(t',t_0)\bra{\vec{p}+\vec{A}}\hat{Y}\hat{\Hs}(t')\hat{X}\ket{\Psi_n(t_0)} \dd t' \label{eq:smatrixlike}
\end{equation*}
In this equation, $\bra{\vec{p}+\vec{A}}\hat{Y}\hat{\Hs}(t')\hat{X}\ket{\Psi_n(t_0)}$ with some simplification is just the tranition dipole matrix element $\vec{d}_{\mathrm{n}}(\vec{p})$ between the bound state and the continuum state, generated by the laser pulse (ionisation).
Furthermore we make an ansatz for $\hat{\U}_0(t',t_0)\ket{\Psi_n(t_0)}=\ket{\Psi_n(t)}$ using the interaction picture.
In contrast to other literature (\cite{Theory_NPS}, \cite{Ivanov20012005}) I am not neglecting transitions between different bound states before the ionisation. 
For instance it can happen that the laserpulse excites the electron but doesnt ionise it quite yet.
Our expression then reads:
\begin{equation*}
    \braket{\Pi(t_{\mathrm{c}})|\Psi_n(t)} = -\frac{i}{2}\int_{t_0}^{t} e^{-i \int_{t'}^{t} (\uvec{P} - \vec{A}(t''))^2 \dd t''}E_z(t')\sum_{n}c_n(t)e^{-iE_nt'}\bra{p_z+A_z}\hat{d}_z\ket{\Psi_n} \dd t'
\end{equation*}
Where we used the fact that the electric field is polarized along the z axis.
\begin{equation}
    \braket{\Pi(t_c)|\Psi(t)} = -i \int_{t_0}^{t} \dd t' e^{-\frac{i}{2}\int_{t'}^{\infty}(\vec{p}+\vec{A}(t''))^2\dd t''}e^{-i \hat{\Hs_0} (t')} \sum_{n}c_n(t')\braket{\vec{p}+\vec{A}(t')|\vec{\hat{d}}\cdot\vec{E}(t')|\Psi_n}
\end{equation}
\begin{equation*}
    = -i \int_{0}^{t} \dd t' e^{-\frac{i}{2}\int_{t'}^{\infty}(\vec{p}+\vec{A}(t''))^2\dd t''} \sum_{n}e^{-iE_n(t')}c_n(t')\vec{E}(t')\cdot\vec{d}_n(\vec{p}+\vec{A}(t'))
\end{equation*}
With $\vec{d}_n(\vec{p}) = \braket{\vec{p}|\vec{\hat{d}}|\Psi_n}$
This is the equation where most papers start with \cite{Theory_NPS}.\\\\
Note that SFA is not about strong laser pulses since here we are also dealing with small ionisation proabilities (<0.01) so SFA states that when ionisation does happen (regardless if its unlikely) the laser pulse will then be the dominant force.


I need to derive in paper from manoram 2023 the same thing as app A just instead of $\hat{p}$ i use $\hat{1}$ and use instead of $\ket{\Psi}_0$ i use the expansion in eigenstates from my project plan


%%%%%%%%%%%%%%%%%%%%
\newpage
\section{Derivation of SFA Rate}
This mainly follows \cite{Theory_NPS} with some modification.\\
The ground state is now a superosition $\sum_{n}c_n(t)\ket{\Psi_n}$.\\
We can write the SFA rate as:
\begin{align*}
    \braket{\Psi(t)|\Psi(t)} = &\int_{-\infty}^{\infty} \Gamma(t) \dd t\\ 
    = &\int \dd^3 p \int_{-\infty}^{\infty} \int_{-\infty}^{\infty} \dd t_1  \dd t_2 \, e^{\frac{i}{2}\int_{t_1}^{t_2}(\vec{p}+\vec{A}(t''))^2\dd t''} \vec{E}(t_1)\cdot\vec{E}(t_2)\\
    &\times\left(\sum_{n}e^{iE_nt_1}c_n^*(t_1)\vec{d}_n^*(\vec{p}+\vec{A}(t_1))\right)\cdot\left(\sum_{n}e^{-iE_nt_2}c_n(t_2)\vec{d}_n(\vec{p}+\vec{A}(t_2))\right)
\end{align*}
Changing variables to $t=\frac{t_2+t_1}{2}$ and $T=\frac{t_2-t_1}{2}$ and using the fact that our Laser pulse is polarized along the z Axis gives us:
\begin{align*}
    \Gamma(t) &= \int \dd^3 p \int_{-\infty}^{\infty} \dd T \, e^{2iIpT+\frac{i}{2}\int_{t-T}^{t+T}(\vec{p}+\vec{A}(t''))^2\dd t''} E_z(t-T)\cdot E_z(t+T)\\
    &\times\left(\sum_{n}e^{iE_n(t-T)}c_n^*(t-T)d_{z,n}^*(\vec{p}+\vec{A}(t-T))\right)\cdot\left(\sum_{n}e^{-iE_n(t+T)}c_n(t+T)d_{z,n}(\vec{p}+\vec{A}(t+T))\right)
\end{align*}
\begin{align*}
    \Gamma(t) &= \sum_{n_1}\sum_{n_2}\int \dd^3 p \int_{-\infty}^{\infty} \dd T \, e^{\frac{i}{2}\int_{t-T}^{t+T}(\vec{p}+\vec{A}(t''))^2\dd t''} e^{iE_{n_1}(t-T)-iE_{n_1}(t+T)}\\
    &\times E_z(t-T) E_z(t+T)c_{n_1}^*(t-T)c_{n_2}(t+T) d_{z,n_1}^*(\vec{p}+\vec{A}(t-T))d_{z,n_2}(\vec{p}+\vec{A}(t+T))
\end{align*}


%%%%%%%%%%%%%%%%%%%%%
\newpage
\section{Strong Field Ionization}
Decied to do Phenomenology after theory.\\
Phenomenology of strong field ionization, Different types of Ionization, tunneling Ionization, multiphoton, stark effect.

