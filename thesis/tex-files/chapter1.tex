Ionization with an intense laser is one of the most fundamental processes in attosecond physics and quantum mechanics in general and has a broad range of applications, from high-harmonic generation to photoelectron spectroscopy and medical physics.
Of particular interest is the `response' of the electron to certain parts of the laser pulse at a given moment of time.
From a classical point of view, the instantaneous ionization rate $\Gamma(t)$ provides a way to realize this concept.
However, quantum mechanics presents a fundamental challenge: only probabilities can be measured, not the specific path an electron takes during ionization.
In addition, predicting the \emph{instantaneous} ionization of the electron at a given moment of time is difficult to reconcile with conventional quantum mechanics \cite{Ivanov2018}.
This is particularly due to the fact that the electron's response to the laser pulse is not instantaneous.

\medskip
Significant research effort has been devoted to developing different ways of defining ionization rates \cite{agarwal2025generalapproximatorstrongfieldionization,Ivanov2018}.
The Strong Field Approximation (SFA) has become one of the standard theoretical frameworks for deriving ionization rates due to its simplicity and reasonably good agreement with numerical solutions of the time-dependent Schrödinger equation (TDSE).
The way it reconciles with quantum mechanics is that within SFA, an expression exists that behaves just like an ionization rate, sharing characteristics of both classical and quantum mechanical physics.
One very important property is that it matches the concept of an ionization rate under the assumption that the electron responds instantaneously to the electric field.

\medskip
Two major problems exist with ionization rates within SFA.
The first is their ambiguity, as the rate is not defined via a physically measurable quantity, making it gauge-dependent.
The second is that the SFA itself is a crude approximation, significantly affecting the predicted ionization process in general.
Further, in previous approaches, not only was the SFA applied, but additional approximations were made regarding the influence of the laser pulse on the atom, such as the Stark shift or transitions to excited states.
This could explain why SFA can reconstruct overall ionization dynamics but fails to accurately reproduce off-cycle ionization dynamics.

\medskip
Additionally, a significant discrepancy exists between the order of magnitude of ionization probabilities predicted by SFA and numerical solutions of the TDSE.
This could have various causes, such as the SFA itself.
However, it could also be due to the non-negligible impact of neglecting transitions to excited states.

\medskip
The aim of this thesis is to investigate the influence of neglecting the laser pulse's impact on the atom in ionization dynamics.
In particular, it examines whether the Stark effect or the distortion of the ground state has a greater influence on ionization dynamics.
It also explores whether transitions to excited states could explain the large absolute difference between SFA-predicted ionization probabilities and numerical TDSE solutions.
For this purpose, rigorous analytical expressions for the ionization rate within SFA are developed.

% \medskip
% This thesis is split into different parts. 
% The first Chapter 


% Understanding the dynamics of this process requires theoretical models that can predict ionization probabilities and provide insights into the underlying mechanisms.
% In real life applications mostly we dont have static electric field, more a laser pulse, with an envelope.

%This approximation leads to significant discrepancies when comparing SFA predictions with ab initio solutions of the time-dependent Schrödinger equation (TDSE). 
% Numerical simulations using solvers like tRecX reveal ionization dynamics that differ substantially from standard SFA predictions, particularly in the temporal structure of the ionization process. 
% When the laser pulse is symmetric in time, SFA predicts a symmetric ionization rate, but numerical simulations show this time symmetry is broken.

% The central hypothesis of this thesis is that these discrepancies arise primarily from the neglect of excited atomic states in standard SFA formulations. 
% In intense laser fields, effects such as the AC Stark shift and transient population of excited states can significantly modify the ionization dynamics. By including these effects, we can potentially bridge the gap between analytical SFA models and numerical TDSE solutions.

% This thesis addresses the following key questions:
% \begin{itemize}
%     \item What is the role of excited atomic states in strong field ionization?
%     \item How does the AC Stark shift influence ionization rates compared to changes in ground state population?
%     \item Can we improve the accuracy of ionization rate predictions by incorporating excited state dynamics into the SFA framework?
% \end{itemize}

% To answer these questions, we extend the SFA formalism to include the influence of excited atomic states on ionization rates. 
% We derive an improved expression for the IIR that incorporates time-dependent probability amplitudes for both ground and excited states, obtained by solving the TDSE within the subspace of bound states. 
% The extended model is then validated through comparison with full TDSE solutions using the TIPTOE (Temporal Ionization Probability in Tunneling and Over-barrier Excursions) method.

% The structure of this thesis is as follows: Chapter 2 derives the extended SFA rate and establishes the theoretical framework. 
% Chapter 3 introduces the numerical methods, including the tRecX solver and the TIPTOE analysis technique. Chapter 4 details the implementation of the extended SFA model and discusses the computational challenges. 
% Chapter 5 presents results comparing standard SFA, extended SFA, and full TDSE solutions, demonstrating the importance of excited state dynamics in strong-field ionization.


% Weither you want your system to ionize or not, it may be important to understand, how ionization works.
% However since it is a quantum mechanical process, one has to first ask what information are even experimentally acessible.
% Quantum mechanics at its very fundamental level tells us one can only measure probabilities and how likely the electron get ionized.
% Unfortunately, in an actual experiment thats the only option you got.
% Further one likes to know more about the "path" the electron takes after it gets struck by the laser pulse i.e the response of the electron to the laser pulse.

% Everytime you hear "path" in quantum mechanics, the first name to think of is Heisenberg.
% His uncertainty principle sets the boundaries for these type of questions and does not let measure all the information you wish to had.
% Further to see the electrons path, one has to probe the system just as lightwaves probe the matter around you letting you see.
% However probing a quantum mechanical system effectively changes the quantum mechanical state of the electron so eventually one has to restrict oneself to something different.

% The challenge of a theoretical model is now to predict ionization propabilities and further explain the underlying mechanisms and in the best case allowing some insights what happens between initial state of the atom and measured electron.
% A clever way to do this is by establishing the concept of an instantanious ionization rate (IIR) via the propertry $P_{\mathrm{ion}}=\int_{\R}\Gamma (t)\dd t$ effectively telling you when and where an electron reacts to a incomming laser pulse.
% At first sight, "instantanious" seems to conflict with the energy time uncertainty relation $\Delta E \Delta t \geq \hbar/2$.
% However, instantanious refers to the response of the electron to the laser field, not the and does not contain eny infromation on when exactly does the electron get ionized.

% Lets make an example to show how an IIR can be beneficial.
% Suppose a measurement is made, a laser pulse hits an atom and the outcome is measured, so if the electron is ionized or not.
% Suppose the measurement is repeated sufficiently many times, one can define a ionization propability for instance $1\%$.
% An IIR now allows to investigate which parts of the laser pulse causes what effects in the total ionization propability.
% One could say, an IIR tells us how likely the electron gets ionized within a certain time interval but of course averaged over all the measurements made.
% Note that the IIR does not tell us the electron will get ionized at a certain time because this would violate the energy time uncertainty relation, the value of $\Gamma$ at a certain time does not have any physical interpretation.

% This concept allows us to investigate the ionization process better and understand more about the underlying mechanisms such as defining different limits and regimes (2.4) or make use of the IIR by sampling a light pulse \cite{Park:18}.
% Due to the challange that the IIR is not really an quantum mechanical observable, it is difficult to have a working notion of it that works for different systems and laser pulses.
% Eventually one has to start from scratch and derive  the IIR from the time dependent schroedinger equation (TDSE) and varios approximations.
% One approximation most used in this context is called Strong field aproximation (SFA) mostly for its simplicity and the capability to derive ionization rates \cite{Theory_NPS}.
% However, SFA has its limitations, especially when it comes to comparing ab initio ionization propabilities, i.e. to numerical solutions of the TDSE using dedicated numerical solver.
% An implicit assumption often made while using SFA is that the laser field has no effect on the atom until the moment of ionization, effectively neglecting excited atomic states.
% In principle this does not have anything to do with the SFA itself, making it a possible solution to the differences between SFA and ab initio ionization propabilities.
% This is one of the main goal of this thesis, investigate if the differences from SFA and ab initio ionization propabilities can be explained by the neglect of excited states.
% Further one would like to investigate the dynamics and effects of the electron before it gets ionized and how they contribute to the ionization dynamics.
% Particulary important would be to know what matters more, the stark shift or the distortion of the ground state wavefunction.
% That could lead to an improvement to existing rates, making them besser applicable in the cases mentioned before.

% To achieve this, one has to review how the SFA was made previously and how it can be improved (Chapter 2).
% Later I will implement the extended SFA rate (Chapter 4) and compare the results with existing SFA rates and ab initio ionization propabilities (Chapter 5).




% What motivates this thesis? Background: developement of SFA and GASFIR rates that doesnt have to numerically solve Schroedinger equation.
% Comparison of ion rates from tRecX, SFA and GASFIR. When the laser pulse is an even function in time, the SFA rate is that as well. 
% But numerical simulations from tRecX tell us thats not the case and the time symmetry is broken. Idea: because of the neglected excitedt states in SFA.
% This brings up more questions: What role play excited states in ionization? Does the stark effect play an important role?\\
% Why is this so complicated? First, $[\hat{\Hs}(t), \hat{\Hs}(t')] \neq 0$ because $\hat{\Hs_0}$ and $\hat{V}$ dont share same eigenbasis, -> the electron is free. Also the thing with all these gauges.



% why so promising? coulomb potential has little effect on ionization dynamics (found manoram)\\
% The only thingmissing is excited states

% This thesis tries to answer the following questions:
% \begin{itemize}
%     \item What is the role of excited states in strong field ionization?
%     \item How does the Stark effect influence ionization rates?
%     \item Can we improve the accuracy of ionization rate predictions without solving the TDSE?
% \end{itemize}

% One of the key findings of this thesis is 