\textcolor{red}{not ready...}Multiphoton ionization of atoms in strong laser fields is a fundamental process in attosecond physics. In this work, we extend the strong-field approximation (SFA) by incorporating the influence of excited atomic states on ionization rates. Standard SFA formulations neglect these excited states, assuming that the laser field has no effect on the atom before ionization. However, in intense few-cycle laser pulses, the Stark shift and transient population of excited states can significantly modify ionization dynamics. We numerically solve the time-dependent Schrödinger equation (TDSE) using the tRecX code to extract time-dependent probability amplitudes for hydrogen’s ground and excited states. These amplitudes are then integrated into the SFA formalism to evaluate their impact on ionization rates. 
The main findings are formula \eqref{eq:sfa_rate_improved}, the fact that stark shift plays a slightly bigger role in the ionization yield but both have very little contribution to the ionization dynamics. THe main cause for the improvement of the ionoizatin dynamics in contrast to previous SFA models is not stark effect, not ground state distortion and not transition to excited states. I.e something ODE does not capture but tRecX does.