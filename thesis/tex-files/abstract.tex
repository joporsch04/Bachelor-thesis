Understanding the temporal evolution of strong-field ionization, often described through an ionization rate, is fundamental for controlling and interpreting electron motion on its natural, attosecond timescale.
While the Strong Field Approximation (SFA) is a common theoretical tool for modeling this process, standard SFA models fail to accurately reproduce certain ionization dynamics and predict ionization yields that differ by orders of magnitude from numerical solutions of the time-dependent Schrödinger equation (TDSE).

To address these shortcomings, this work develops an extended SFA formalism that incorporates pre-ionization dynamics, including the Stark effect, ground state distortion, and transitions to excited states. This is achieved by using the time-dependent coefficients to enhance the rate, which are determined by solving the TDSE both within a restricted subspace of bound states and in the full Hilbert space.

The results demonstrate that the extended SFA model, using coefficients from the full Hilbert space simulation, significantly improves the reconstruction of off-cycle ionization dynamics compared to standard SFA. The improvement is almost entirely due to the phase evolution of the ground-state coefficient, with its amplitude having only a negligible effect. Surprisingly, the conventional AC Stark shift, isolated using the subspace-restricted coefficients, was shown to play only a minor role. This suggests that the crucial phase contribution arises from effects that are only captured when ionization is allowed in the simulation.

While the model successfully improves reconstructing the ionization dynamics, the large discrepancy in the absolute ionization yield remains. Preliminary findings suggest that including transitions to excited states within the extended SFA framework could partially bridge this gap, marking a key direction for future research.