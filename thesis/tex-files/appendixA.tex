\section{Derivation of general transition dipole matrix elements}
see mathematica notebook\\
Roadmap: Expand plane wave solution to spherical harmonics and the radial part into bessel functions.\\
Mathematica can do the rest for you\\
Explicitiely for 1s to plane wave as shown here:\\
Some dipole matrix elements:\\
Spherical harmonic: Instead of $Y_{lm}(\theta, \phi)$ you can write $Y_{lm}(\uvec{r})$ since $\uvec{r}=\hat{e}_x \sin(\theta)\cos(\phi)+\hat{e}_y\sin(\theta)\sin(\phi)+\hat{e}_z\cos(\theta)$\\
The transition dipole matrix element is given by
% \begin{equation*}
%     \vec{d}(\vec{p}) = \braket{\vec{p}|\vec{\hat{d}}|\Psi_{nlm}} = \nabla_{\vec{p}}\Psi_{nlm}(\vec{p})
% \end{equation*}
% \begin{equation*}
%     \braket{\vec{p}|100} = \frac{8 \sqrt{\pi }}{\sqrt{\frac{1}{a^3}} \left(a^2 p^2+1\right)^2}\text{if}\Re\left(\frac{1}{a}\right)>0
% \end{equation*}
% \begin{equation*}
%     \braket{\vec{p}|200} = \frac{32 \sqrt{2 \pi } \left(4 a^2 p^2-1\right)}{\sqrt{\frac{1}{a^3}} \left(4 a^2 p^2+1\right)^3}\text{if}\Re\left(\frac{1}{a}\right)>0
% \end{equation*}
% \begin{equation*}
%     \braket{\vec{p}|210} = -\frac{128 i \sqrt{2 \pi } \sqrt{\frac{1}{a^3}} a^4 p}{\left(4 a^2 p^2+1\right)^3}\text{if}\Re\left(\frac{1}{a}\right)>0
% \end{equation*}
% \begin{equation*}
%     \braket{\vec{p}|300} = \frac{72 \sqrt{3 \pi } \left(81 a^4 p^4-30 a^2 p^2+1\right)}{\sqrt{\frac{1}{a^3}} \left(9 a^2p^2+1\right)^4}\text{ if }\Re\left(\frac{1}{a}\right)>0
% \end{equation*}
% \begin{equation*}
%     \braket{\vec{p}|310} = -\frac{864 i \sqrt{2 \pi } \sqrt{\frac{1}{a^3}} a^4 p \left(9 a^2 p^2-1\right)}{\left(9 a^2 p^2+1\right)^4}
% \end{equation*}
% \begin{equation*}
%     \braket{\vec{p}|320} = -\frac{1728 \sqrt{6 \pi } \sqrt{\frac{1}{a^3}} a^5 p^2}{\left(9 a^2 p^2+1\right)^4}
% \end{equation*}
% \begin{equation*}
%     \braket{\vec{p}|400} = \frac{256 \sqrt{\pi } \left(4096 a^6 p^6-1792 a^4 p^4+112 a^2 p^2-1\right)}{\sqrt{\frac{1}{a^3}}\left(16 a^2 p^2+1\right)^5}\text{ if }\Re\left(\frac{1}{a}\right)>0
% \end{equation*}
% \begin{equation*}
%     \braket{\vec{p}|410} = -\frac{2048 i \sqrt{\frac{\pi }{5}} \left(\frac{1}{a^3}\right)^{3/2} a^7 p \left(32 a^2 p^2 \left(40 a^2p^2-7\right)+5\right)}{\left(16 a^2 p^2+1\right)^5}\text{ if }\Re\left(\frac{1}{a}\right)>0
% \end{equation*}
% \begin{equation*}
%     \braket{\vec{p}|420} = -\frac{32768 \sqrt{\pi } \sqrt{\frac{1}{a^3}} a^5 p^2 \left(16 a^2 p^2-1\right)}{\left(16 a^2 p^2+1\right)^5}
% \end{equation*}
% \begin{equation*}
%     \braket{\vec{p}|430} = \frac{262144 i \sqrt{\frac{\pi }{5}} \sqrt{\frac{1}{a^3}} a^6 p^3}{\left(16 a^2 p^2+1\right)^5}
% \end{equation*}
% \begin{equation*}
%     \braket{\vec{p}|900} = \frac{1944 \sqrt{\pi } \left(27 a^2 p^2-1\right) \left(243 a^2 p^2-1\right) \left(243 a^2 p^2 \left(6561
%     a^4 p^4-729 a^2 p^2+11\right)-1\right) \left(729 \left(243 a^6 p^6-99 a^4 p^4+a^2
%     p^2\right)-1\right)}{\sqrt{\frac{1}{a^3}} \left(81 a^2 p^2+1\right)^{10}}\text{ if
%     }\Re\left(\frac{1}{a}\right)>0
% \end{equation*}
% \begin{equation*}
%     \braket{\vec{p}|510} = -\frac{4000 i \sqrt{10 \pi } \sqrt{\frac{1}{a^3}} a^4 p \left(15625 a^6 p^6-3375 a^4 p^4+135 a^2
%     p^2-1\right)}{\left(25 a^2 p^2+1\right)^6}\text{ if }\Re\left(\frac{1}{a}\right)>0
% \end{equation*}