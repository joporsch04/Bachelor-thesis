The simulation of quantum mechanical systems presents a formidable challenge for classical computers. Representing the state of an n-particle quantum system requires tracking $2^N$ complex amplitudes, leading to an exponential explosion in both memory requirements and computational time as N grows. Consequently, accurately simulating phenomena such as high-temperature superconductivity, the intricate reaction pathways of large molecules in quantum chemistry, or other strongly correlated many-body systems quickly becomes infeasible, even on the most powerful supercomputers. In 1982, Richard Feynman envisioned a groundbreaking solution: a computer built from quantum mechanical elements could effectively simulate these complex quantum problems \cite{Feynman}. Modern quantum computers aim to realize this vision by implementing arbitrary unitary operators on a system of qubits. While significant progress has been made, current quantum computers are still limited in scale and often struggle to outperform classical computers using tensor networks for many relevant problems \cite{Fraxanet_2023}.\\
As a powerful near-term alternative, (analogue) quantum simulators have emerged. These devices are designed to directly implement a specific target Hamiltonian, leveraging the inherent quantum nature of their constituents—such as entanglement and many-particle behavior—to explore complex physical phenomena. Quantum simulators can be build using different architectures - ranging from trapped ions, quantum dots, superconducting circuits to ultracold neutral atoms \cite{qstext}. Ultracold neutral atoms trapped in optical lattices have proven to be a particularly versatile and promising platform for quantum simulation. They provide a high degree of controllability, allowing single-atom and single-site resolution to measure (non-local) correlation functions and counting statistics via quantum gas microscopes \cite{Bloch2012,impertro2024local}.\\
A compelling direction in this field is the convergence of gate-based quantum computation and quantum simulation. The FermiQP project, for example, aims to develop a quantum processor based on ultracold fermionic Lithium ($^6$Li) operable in two distinct modes: an analogue mode for quantum gas microscopy and simulation of quantum materials, and a digital mode for quantum computation. The goal of such platforms is to combine the strengths of both approaches – leveraging analogue simulation capabilities while introducing the programmability of gate-based operations.\\
Nevertheless, the system restricts operations to single- or two-qubit gates, limiting atom manipulations to individual or adjacent lattice sites. Only recently, \cite{impertro2024local} demonstrated complete access to arbitrary states on the Bloch sphere using nearest-neighbor (NN) gates, enabling the preparation of any desired initial state. However, if precise control over these local interactions is possible, the central question becomes: how can we systematically combine them to achieve any desired global N-atom unitary? \\
This thesis investigates algorithms to systematically construct arbitrary N-dimensional unitary operations using sequences of these experimentally accessible local gates. It  will explore the application of this decomposition to implement crucial quantum operations such as 
\begin{itemize}
    \item \textbf{Atom Rearrangement (Permutation Matrices)}: Essential for preparing arbitrary initial Fock states, correcting defects in atom loading, and all-to-all connectivity in the simulator
    \item \textbf{Discrete (Fractional) Fourier Transformations}: Necessary for transitioning between position and momentum space representations, measuring momentum distributions, probing kinetic energy terms, and implementing protocols for entanglement measurement or quantum phase estimation.
\end{itemize}
Furthermore, the impact of realistic experimental noise on the fidelity of these decomposed operations will be analyzed and the extension of these one-dimensional decomposition concepts to two-dimensional lattice structures discussed.


\section{Fermi-Hubbard Model}
The Fermi-Hubbard Model is a simplified model which is often used to describe fermions in a solid and is based on the tight binding model. Its Hamiltonian considers a one-particle contribution, which describes hopping, and a two-particle on-site interaction which is the shortest possible. Although the Hubbard model can be viewed as the simplest model of correlated electrons, it is still describing many interesting phenomena in nature like magnetic ordering, metal-insulator transition, or (high-temperature) superconductivity. Therefore, the Hubbard model is the basis for researchers interested in these properties. Its Hamiltonian is given by
\begin{align}
    \hat{H}_{hub} = \hat{H}_{kin} + \hat{H}_{int} = \sum_{x,y\in V,\sigma} t_{xy}\,c_{x,\sigma}^\dagger\,c_{y,\sigma}
\;+\;\sum_x U_x\,c_{x\uparrow}^\dagger\,c_{x\downarrow}^\dagger\,c_{x\downarrow}\,c_{x\uparrow} 
\end{align}
where x and y are spatial directions, $c_i^\dagger$ and $c_i$ are the fermionic creation and annihilation operators, V is the vertex set, $t_{xy}$ is the hopping amplitude between sites x and y and $U_x$ is the on-site interaction, often the Coulomb interaction. For a lattice, one usually only considers nearest-neighbor hopping, meaning that $t_{xy}=t$ for $|x-y|=1$ and t=0 otherwise \cite{hubbardmielke}.

