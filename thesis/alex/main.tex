\documentclass[12pt,twoside]{report}  % or scrreprt, book, etc.

% ------------------------------------------------------------------------------
%   PACKAGES & SETTINGS
% ------------------------------------------------------------------------------
\usepackage[utf8]{inputenc}     % For UTF-8 encoding
\usepackage[T1]{fontenc}        % For correct output of accented characters in PDF
\usepackage[ngerman,english]{babel} % German & English hyphenation (pick your main language)
\usepackage{amsmath,amssymb}    % Math packages
\usepackage{graphicx}           % For including images
\usepackage{geometry}           % Easy page setup
\geometry{
    a4paper,
    left=25mm,
    right=25mm,
    top=25mm,
    bottom=25mm
}
\usepackage{setspace}           % For line spacing
\onehalfspacing                 % 1.5 line spacing (common in theses)
\usepackage{csquotes}           % Recommended for proper quotation marks
\usepackage{hyperref}           % For hyperlinks in PDF (especially for references)
\hypersetup{
    colorlinks=true,
    linkcolor=black,
    urlcolor=blue,
    citecolor=black
}
\usepackage{afterpage}
\newcommand\blankpage{%
    \null
    \thispagestyle{empty}%
    \addtocounter{page}{-1}%
    \newpage}
\usepackage[utf8]{inputenc}
\usepackage[T1]{fontenc}
\usepackage{fancyhdr}
% Use fancyhdr for custom headers/footers
\pagestyle{fancy}
\fancyhf{} % Clear default header/footer

\renewcommand{\chaptermark}[1]{%
  \markboth{\chaptername\ \thechapter:\ #1}{}%
}
\renewcommand{\sectionmark}[1]{%
  \markright{\thesection.\ #1}%
}

% Set header: even pages get chapter title, odd pages get section title.
\fancyhead[LE]{\nouppercase{\leftmark}} % Left header on even pages: chapter title
\fancyhead[RO]{\nouppercase{\rightmark}} % Right header on odd pages: section title

% Also place the page number on the outer corners.
\fancyhead[LO]{\thepage} % Left header on odd pages: page number
\fancyhead[RE]{\thepage} % Right header on even pages: page number

% Option A: If you want to remove headers from the chapter-opening page,
% leave the following redefinition (plain pages will be empty).
\makeatletter
\let\ps@plain\ps@empty
\makeatother

% Listings package for code formatting
\usepackage{listings}
\usepackage{xcolor}

% Configure listings settings for Python
\lstset{ %
  language=Python,
  basicstyle=\ttfamily\footnotesize,
  numbers=left,
  numberstyle=\tiny,
  stepnumber=1,
  numbersep=5pt,
  backgroundcolor=\color{white},
  showspaces=false,
  showstringspaces=false,
  showtabs=false,
  frame=single,
  rulecolor=\color{black},
  tabsize=2,
  captionpos=b,
  breaklines=true,
  breakatwhitespace=false,
  keywordstyle=\color{blue}\bfseries,
  commentstyle=\color{green!50!black},
  stringstyle=\color{red},
  columns=fullflexible
}
\usepackage{braket}

% ------------------------------------------------------------------------------
%   BIBLIOGRAPHY SETTINGS
% ------------------------------------------------------------------------------
% You can choose a bibliography style. For example, plain, unsrt, alpha, apa, etc.
%\bibliographystyle{plain}

% ------------------------------------------------------------------------------
%   BEGIN DOCUMENT
% ------------------------------------------------------------------------------
\begin{document}

% ------------------------------------------------------------------------------
%   TITLE PAGE (example adapted for a Bachelor Thesis)
% ------------------------------------------------------------------------------
\begin{titlepage}
    \begin{center}
        \Large
        \textbf{Ludwig-Maximilians-Universität München}\\
        Faculty of Physics
        
        \vspace{1cm}
        
        {\Huge \textbf{Your Thesis Title Here}}

        \vspace{1cm}

\vspace{1cm}
        
        \Large
        \textbf{Bachelor Thesis}\\
        submitted by\\
        \textbf{Alexander Roth}
        
        \vspace{1cm}
        
        Supervised by\\
        \vspace{0.5cm}
        \textbf{Prof.\ Dr.\ Immanuel Bloch} \\
        Chair of Quantum Optics, Ludwig-Maximilians-Universität München \\
        and Max Planck Institute of Quantum Optics \\
        \vspace{0.5cm}
        \textbf{Dr.\ Titus Franz, Dr.\ Philipp Preiss} \\
        Max Planck Institute of Quantum Optics
        
        \vfill
        
        Munich, Month Day, 2025
    \end{center}
\end{titlepage}
\afterpage{\blankpage}

% ------------------------------------------------------------------------------
%   FRONT MATTER
% ------------------------------------------------------------------------------
\pagenumbering{roman}
\tableofcontents
\listoffigures        % optional
\listoftables         % optional

\clearpage
\pagenumbering{arabic}

% ------------------------------------------------------------------------------
%   MAIN CONTENT
% ------------------------------------------------------------------------------
\chapter{Introduction}
% Introduce your topic, motivation, and outline the thesis structure.

\chapter{Theoretical Background}
\section{Trapping on a Lattice}
An optical lattice is being created by a standing laser light which acts as a periodic potential due to the Stark Shift. 
\section{Gates}
To run quantum algorithms on a lattice, one needs to perform gates on it. Experimentally, not all gates are equally easy to implement. \\

\section{Fermions vs. Qubits}

\section{Fermi-Hubbard Model}


\section{Jordan-Wigner Mapping}

\chapter{Implementing a N-dimensional Unitary on a Lattice}
\section{Interferometry}
In interferometry, Reck et al. have shown in 1994 that any $N \times N$ Unitary matrix can be implemented using only 50:50 beam splitters and phase shifters. This is done by decomposing the U(N) matrix using U(2) matrices acting on a 2-dimensional subspace of the Hilbert space \cite{Reck1994}. These U(2) matrices can be implemented in the optical setup by using two 50:50 beam splitters and two phase shifters acting on wires n,m that can be represented by the following matrix:
\begin{align}
    T_{n,m}(\theta,\phi) = \begin{pmatrix}
1 &   0     & \cdots       &        &        &        &     & 0\\
0  & 1 &        &        &        &  & & \vdots    \\
\vdots &  &  \ddots      &        &        &   &    \\
  &    & & e^{i\phi}\cos\theta & -\sin\theta & & \\
  &    & & e^{i\phi}\sin\theta & \cos\theta \\
  &        &        &        &        & \ddots \\
    &        &        &        &        & &1 &0\\
0 &   \cdots &      &        &        &  &   0   &      1\\
\end{pmatrix}_{N\times N}
\end{align}
Mathematically speaking, the matrix $T_{n,m}(\theta,\phi)$ resembles a Givens Rotation acting on wires n,m. A Givens rotation is a rotation in a plane and, if applied from the right, mixes two columns in the same row. It is often used to null specific elemtents of a matrix in numerical methods. Following the QR decomposition algorithm, we can find a decomposition of U into an upper triangular matrix (R) and a number of Givens rotations (Q):
\begin{align}
    U(N) = R*Q
\end{align}
Since U is unitary, R is not only an upper diagonal matrix but has to be a diagonal matrix with modulus 1 and complex entries which we will call D in the following.\\
The decomposition of an arbitrary Unitary of dimension N in a diagonal matrix and Givens Rotations is therefore:
\begin{align}
    U(N) \;=\; D\;\!\Bigl(\,\prod_{(m,n)\in S} T_{m,n}\Bigr) \;\; ; \;\; n,m \;\epsilon \;[0, N-1]
    \label{Clementsdecomp}
\end{align}
where S is a specific ordered sequence of two-mode transformations \cite{Clements:16}.\\
The order in which the Givens rotations are applied is not arbitrary. One starts from the bottom-left corner and moves to the nearest subdiagonal, moving from top to bottom \cite{cilluffo2024}. This is also illustrated in Figure \ref{decompositioncircuit}. \\

\section{Application on a Quantum Circuit}
In the context of quantum simulation this can be represented using Pauli-Z ($\hat{Z}$) and Pauli-X ($\hat{X}$) Rotations. The $\hat{X}$ gate resembles the beam splitter and the $\hat{Z}$ gate the phase shift. 
\begin{align}
    \hat{X}(\theta) \;=\; \begin{pmatrix}
\cos\!\bigl(\tfrac{\theta}{2}\bigr) & i\,\sin\!\bigl(\tfrac{\theta}{2}\bigr)
\\[6pt]
i\,\sin\!\bigl(\tfrac{\theta}{2}\bigr) & \cos\!\bigl(\tfrac{\theta}{2}\bigr)
\end{pmatrix},
\qquad
\hat{Z}(\theta) \;=\; \begin{pmatrix}
e^{-i\theta/2} & 0
\\[6pt]
0 & e^{i\theta/2}
\end{pmatrix}.
\end{align}
For $\hat{X}(\theta)$ to resemble a 50:50 beam splitter, we set $\theta=\pi/2$. The angle for the $\hat{Z}$-gate is freely choosable, providing the necessary degrees of freedom to resemble any matrix U(N). One can easily show, that
\begin{align}
    T_{n,m}(\theta,\phi) = e^{i\alpha}\hat{X}(\pi/2)\hat{Z}(\omega)\hat{X}(\pi/2)\hat{Z}(\psi)
\end{align}
since
\begin{align}
    e^{i\alpha}\hat{X}(\pi/2)\hat{Z}(\omega)\hat{X}(\pi/2)\hat{Z}(\psi)= i*e^{i(\alpha +\psi/2)}\begin{pmatrix}
\sin(\omega/2)\,e^{-i\psi} & \cos(\omega/2)\\[2mm]
\cos(\omega/2)\,e^{-i\psi} & \sin(\omega/2)
\end{pmatrix}.
\end{align}
We see that by choosing $\alpha= -\pi + \phi/2$, $\omega= 2\theta - \pi$ and $\psi = -(\pi + \phi)$ we get $T_{n,m}(\theta,\phi)$. \\
We can therefore find the decomposition of any Unitary into a combination of X2Z2 acting on two spatial modes in the lattice with two global tunneling gates ($\hat{X}(\pi/2)$) and two $\hat{Z}$-rotations. Both gates act on the 2 qubit subspace $\{\ket{01},\ket{10}\}$.
\begin{figure}[t]
    \centering
    \includegraphics[width=0.7\linewidth]{5dim_example_decomposition.png}
    \caption{Example circuit to decompose a 5 dimensional Fourier transformation (or any five dimensional unitary matrix)}
    \label{decompositioncircuit}
\end{figure}
To implement this in PennyLane, the \textit{givens\_decomposition} function from the PennyLane Python library, v.0.40.0, was used to create the decomposition of an arbitrary input unitary in equation \ref{Clementsdecomp}. The code for the decomposition of the single Givens rotations into X2Z2 can be seen in Appendix \ref{one_qubit_Decomp}. \\
It is important to note, that all calculations are in the non-interacting case. Therefore, we can limit ourselves to the N-dimensional non-interacting subspace of the $2^N \times 2^N$ total Hilbert space.


\section{Use Cases}
The use cases of being able to implement any unitary matrix on the lattice are numerous.\\
To measure momentum space, a standard method is to switch the optical lattice and
harmonic trapping potential off and perform a time-of-flight imaging. This method, though, has a number of limitations. The measuring quality is affected by the inhomogeneity of the trapping potential and the accuracy of the absolute number of atoms is around $\pm 10\%$ \cite{Esslinger_2010}. With the above described algorithm, we are able to implement a discrete Fourier Transformation on the lattice. This allows to measure momentum space directly without workarounds and would only be limited by the gate fidelities.
\newpage

\chapter{Methods}
% Experimental setup or methodology.

\chapter{Results}
% Present your results (figures, tables, analysis).

\chapter{Discussion}
% Interpret and discuss the results in context of the theory.

\chapter{Conclusion and Outlook}
% Summarize key findings, highlight contributions, and propose future work.

% ------------------------------------------------------------------------------
%   BIBLIOGRAPHY
% ------------------------------------------------------------------------------
\clearpage
\bibliographystyle{unsrt}
\bibliography{references}    % references.bib is your BibTeX file


% ------------------------------------------------------------------------------
%   APPENDICES (IF ANY)
% ------------------------------------------------------------------------------
\appendix

\chapter{Decomposition into XZXZ}
\label{one_qubit_Decomp}

\begin{lstlisting}[language=Python, caption={Python code for one_qubit_decomposition.}]
import numpy as np
from scipy import optimize
import pennylane as qml

def normalize_angle(angle):
    """
    Normalizes an angle into the interval [0, 4*pi].
    """
    return np.mod(angle, 4*np.pi)

# -----------------------------------------------------------------------------
# Helper: Convert a 2x2 unitary U into an SU(2) matrix by factoring out a global phase.
# Returns U_su2 and the phase factor such that U = exp(i*alpha) * U_su2.
def convert_to_su2(U, return_global_phase=True):
    U = np.array(U, dtype=complex)
    det = np.linalg.det(U)
    # Let alpha be such that U = exp(i*alpha) * (U * exp(-i*alpha)) has determinant 1.
    alpha = np.angle(det) / 2
    alpha = normalize_angle(alpha)
    U_su2 = U * np.exp(-1j * alpha)
    if return_global_phase:
        return U_su2, alpha
    return U_su2

# -----------------------------------------------------------------------------
# 1. Decomposition of the form:
#    U = phase * RZ(psi) RX(phi) RZ(theta)
# We derive an analytic solution by writing out the matrix products.
def ZXZ_decomp(U, wire, return_global_phase=False):
    U_su2, alpha = convert_to_su2(U, return_global_phase=True)
    # Let U_su2 = [ a   b ]
    #             [ c   d ]
    # and note that for our decomposition the product is:
    #   RZ(psi) RX(phi) RZ(theta) = 
    #   [ cos(phi/2) exp(-i(psi+theta)/2)   -i sin(phi/2) exp(-i(psi-theta)/2) ]
    #   [ -i sin(phi/2) exp(i(psi-theta)/2)    cos(phi/2) exp(i(psi+theta)/2)  ]
    #
    # Hence we can set:
    #   phi = 2 arctan2(|b|, |a|)
    #   psi+theta = -2 arg(a)
    #   psi-theta = -2 ( arg(b) + pi/2 )
    a = U_su2[0, 0]
    b = U_su2[0, 1]
    
    # Calculate angles
    phi = 2 * np.arctan2(np.abs(b), np.abs(a))
    psi_plus_theta = -2 * np.angle(a)
    psi_minus_theta = -2 * (np.angle(b) + np.pi/2)
    
    psi = 0.5 * (psi_plus_theta + psi_minus_theta)
    theta = 0.5 * (psi_plus_theta - psi_minus_theta)
    
    # Normalize all angles into the interval [0, 4*pi]
    phi = normalize_angle(phi)
    psi = normalize_angle(psi)
    theta = normalize_angle(theta)
    
    ops = []
    if return_global_phase:
        ops.append(qml.GlobalPhase(alpha))
    ops.extend([
        qml.RZ(psi, wires=wire),
        qml.RX(phi, wires=wire),
        qml.RZ(theta, wires=wire)
    ])
    return ops

# -----------------------------------------------------------------------------
# 2. Decomposition of the form:
#    U = phase * RX(pi/2) RZ(phi) RX(pi/2) RZ(psi)
# Here the two RX(pi/2) rotations are fixed; we numerically determine the free angles phi and psi.
def XZXZ_decomp(U, wire, return_global_phase=False):
    U_su2, alpha = convert_to_su2(U, return_global_phase=True)
    
    # Define the basic gates with PennyLane's matrix definitions.
    def RX(theta):
        return np.array([[np.cos(theta/2), -1j*np.sin(theta/2)],
                         [-1j*np.sin(theta/2), np.cos(theta/2)]], dtype=complex)
    
    def RZ(theta):
        return np.array([[np.exp(-1j*theta/2), 0],
                         [0, np.exp(1j*theta/2)]], dtype=complex)
    
    # Fixed gate RX(pi/2)
    F = RX(np.pi/2)
    
    # Define the candidate matrix in SU(2) (ignoring the overall phase)
    def M(params):
        phi, psi = params
        # Circuit order: first RX(pi/2), then RZ(phi), then RX(pi/2), then RZ(psi)
        return F @ RZ(phi) @ F @ RZ(psi)
    
    def cost(params):
        return np.linalg.norm(M(params) - U_su2)**2
    
    # Initial guess
    initial_guess = np.array([0.0, 0.0])
    res = optimize.minimize(cost, initial_guess, method="BFGS")
    phi_opt, psi_opt = res.x
    
    # Normalize angles
    phi_opt = normalize_angle(phi_opt)
    psi_opt = normalize_angle(psi_opt)
    
    ops = []
    if return_global_phase:
        ops.append(qml.GlobalPhase(alpha))
    ops.extend([
        qml.RX(np.pi/2, wires=wire),
        qml.RZ(phi_opt, wires=wire),
        qml.RX(np.pi/2, wires=wire),
        qml.RZ(psi_opt, wires=wire)
    ])
    return ops

# -----------------------------------------------------------------------------
# 3. Decomposition of the form:
#    U = phase * RZ(psi) RX(pi/2) RZ(phi) RX(pi/2) RZ(theta)
# Again we have fixed RX(pi/2) gates. We now determine the free angles psi, phi, theta.
def ZXZXZ_decomp(U, wire, return_global_phase=False):
    U_su2, alpha = convert_to_su2(U, return_global_phase=True)
    
    def RX(theta):
        return np.array([[np.cos(theta/2), -1j*np.sin(theta/2)],
                         [-1j*np.sin(theta/2), np.cos(theta/2)]], dtype=complex)
    
    def RZ(theta):
        return np.array([[np.exp(-1j*theta/2), 0],
                         [0, np.exp(1j*theta/2)]], dtype=complex)
    
    F = RX(np.pi/2)
    
    def M(params):
        psi, phi, theta_val = params
        # Circuit order: first RZ(psi), then RX(pi/2), then RZ(phi), then RX(pi/2), then RZ(theta)
        return RZ(psi) @ F @ RZ(phi) @ F @ RZ(theta_val)
    
    def cost(params):
        return np.linalg.norm(M(params) - U_su2)**2
    
    initial_guess = np.array([0.0, 0.0, 0.0])
    res = optimize.minimize(cost, initial_guess, method="BFGS")
    psi_opt, phi_opt, theta_opt = res.x
    
    # Normalize angles
    psi_opt = normalize_angle(psi_opt)
    phi_opt = normalize_angle(phi_opt)
    theta_opt = normalize_angle(theta_opt)
    
    ops = []
    if return_global_phase:
        ops.append(qml.GlobalPhase(alpha))
    ops.extend([
        qml.RZ(psi_opt, wires=wire),
        qml.RX(np.pi/2, wires=wire),
        qml.RZ(phi_opt, wires=wire),
        qml.RX(np.pi/2, wires=wire),
        qml.RZ(theta_opt, wires=wire)
    ])
    return ops

# -----------------------------------------------------------------------------
# Dispatch function: choose the desired decomposition form.
def one_qubit_decomposition(U, wire, rotations="ZYZ", return_global_phase=False):
    supported_rotations = {
        "ZXZ": ZXZ_decomp,
        "XZXZ": XZXZ_decomp,
        "ZXZXZ": ZXZXZ_decomp,
    }

    if rotations in supported_rotations:
        return supported_rotations[rotations](U, wire, return_global_phase)

    raise ValueError(
        f"Value {rotations} passed to rotations is either invalid or currently unsupported."
    )
    
\end{lstlisting}

\end{document}