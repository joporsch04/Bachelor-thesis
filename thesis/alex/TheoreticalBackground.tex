\section{Quantum Information}

\section{Trapping on a Lattice}
To simulate Hubbard dynamics, the FermiQP experiment traps ultracold fermions in an optical lattice.
An optical lattice is created by a standing laser light which acts as a periodic potential due to the Stark shift. The FermiQP supperlattice structure is created by two superimposed lattices, creating double wells, allowing for four fermionic modes, two spin modes on two sites ($\ket{L,\uparrow},\ket{L,\downarrow},\ket{R,\uparrow},\ket{R,\downarrow}$). In this thesis, only the noninteracting case is being considered. This reduces the amount of fermionic modes to two ($\ket{L}, \ket{R}$).


\section{Motional Gates}
To run quantum algorithms on a lattice, one needs to perform gates on it. Experimentally, not all gates are equally easy to implement. \\
Some of the most fundamental gates are the Pauli matrices
\begin{align}
    \hat{X}= \begin{pmatrix}
        0 & 1 \\
        1 & 0 \\
    \end{pmatrix}
    \;;\; 
    \hat{Y} = \begin{pmatrix}
        0 & -i \\
        i & 0 \\
    \end{pmatrix}
    \;;\;
    \hat{Z} = \begin{pmatrix}
        1 & 0 \\
        0 & -1 \\
    \end{pmatrix}
\end{align}
and Pauli-Rotation Gates:
\begin{align}
    \widehat{X}(\theta)= e^{-i\frac{\theta}{2}\hat{X}} \;;\;
    \widehat{Y}(\theta)= e^{-i\frac{\theta}{2}\hat{Y}} \;;\;
    \widehat{Z}(\theta)= e^{-i\frac{\theta}{2}\hat{Z}} 
\end{align}
Especially important for us is the $\widehat{X}(\frac{\pi}{2})$ rotation gate which is a real tunneling operation (see also Figure \ref{fig:LRXandLRZ}). This also corresponds to a 50:50 beam splitter in optics. The matrix representation of the operator looks like
\begin{align}
    \widehat{X}(\frac{\pi}{2}) = \frac{1}{\sqrt{2}}\begin{pmatrix}
        1 & i \\
        i & 1 \\
    \end{pmatrix}.
    \label{RXgate}
\end{align}
Figure \ref{fig:bloch_rx} shows how this gate acts on the two modes of a site, if in the site sits no fermion, we label it $\ket{0}$, otherwise if there is a fermion, we label it with $\ket{1}$. 

\begin{figure}[t]
  \centering
  % erste Subfigure
  \begin{subfigure}[t]{0.48\linewidth}
    \subcaption{} \label{fig:rx0}
  \end{subfigure}
  \hfill
  % zweite Subfigure
  \begin{subfigure}[t]{0.48\linewidth}
    \subcaption{} \label{fig:rx1}
  \end{subfigure}

  \caption[Visual effect of the $\widehat{X}(\tfrac\pi2)$ on initial states]{The $\widehat{X}(\tfrac\pi2)$‐gate applied on the initial state (a) $\ket{0}$ and (b) $\ket{1}$ shown on the Bloch-sphere.}
  \label{fig:bloch_rx}
\end{figure}

The $\widehat{Z}(\theta)$ can be implemented in the lattice through using the chemical potential (see Figure \ref{fig:LRXandLRZ}). The optics analogy would be a phase shifter.\\
As a last gate, we want to take a look at the SWAP operator. This gate simply swaps the two lattice sites, so the effect on an atom in the double well is
\begin{align}
    \widehat{SWAP} = \begin{pmatrix}
        0 & 1\\
        1 & 0
    \end{pmatrix} \Longrightarrow \widehat{SWAP}\ket{L} = \ket{R} \;;\; \widehat{SWAP}\ket{R} = \ket{L}
\end{align}
\begin{figure}[t]
    \centering
    \caption[$\widehat{X}$ and $\widehat{Z}$ implementation on the lattice]{On a lattice, a the $\widehat{X}(\pi/2)$ matrix corresponds to real tunneling in the double well and the $\widehat{Z}(\theta$ matrix can be implemented by a chemical potential, tilting one of the potential wells.}
    \label{fig:LRXandLRZ}
\end{figure}

\begin{figure}
    \centering
    \caption[Periodicity of the symmetric and asymmetric $\widehat{Z}$ gate]{The symmetric $\widehat{Z}$ gate has a periodicity of $4\pi$ whereas the asymmetric $\widehat{Z}$ gate has a periodicity of $2\pi$}
    \label{fig:enter-label}
\end{figure}

\section{Fidelity of Quantum Gates}
An important criteria for benchmarking the implementation of a circuit on a quantum simulator is the fidelity. The fidelity gives the probability, how well the desired state can be implemented using the circuit. The fidelity of two quantum states $\ket{psi_\rho}$ and $\ket{psi_\sigma}$ is defined as
\begin{align}
    \mathcal{F}= \bigg(tr\sqrt{\sqrt\rho\sigma\sqrt{\rho}}\bigg)^2
\end{align}
where tr is the trace operator and $\rho$ and $\sigma$ are the density matrices of the two states. For pure states, this reduces to
\begin{align}
    \mathcal{F} = |\braket{\psi_{\rho}|\psi_{\sigma}}|^2 \equiv |\braket{\psi_{ideal}|\psi_{obs}}|^2.
\end{align}

\section{Interferometry}
The solution on how to implement a random unitary matrix through simpler pieces has long been solved in quantum optics. \\
In 1994, Reck et al. have shown that any $N \times N$ Unitary matrix can be implemented using only 50:50 beam splitters and phase shifters. The main idea is that the U(N) matrix can be factorized into many two-dimensional beam splitters. These beam splitters are implemented using a mesh of Mach-Zehnder interferometers (MZI). These consist of two 50:50 beam splitters and, depending on the convention, one or two phase shifters (see figure \ref{MZI}). We will use the convention with only one phase shifter, as it is displayed in figure \ref{MZI}, in the following.

\begin{figure}[t]
    \centering
    \caption[Mach-Zehnder Interferometer]{MZI representation with one phase shifter (asymmetric MZI)}
    \label{MZI}
\end{figure}

\subsection{Reck and Clements Scheme}
In 1994, Reck and Zeilinger showed for the first time, that any Unitary can be realised by a mesh of beam splitters. The Reck decomposition method, implements any unitary through a triangular mesh of beam splitters (see figure \ref{ReckClements} (a)). It uses the minimal number of beam splitters but the optical depth is $2N-3$ \cite{Reck1994}. Recks scheme can be used for universal multiport interferometers which are used in many areas such as Boson sampling \cite{Carolan_2015}, discrete-time quantum walks \cite{quantumwalk} or as a universal mode analyzer, allowing measurements in any basis \cite{Reck1994}.

2016, Clements published a paper that improved on Reck's scheme. The Clements scheme uses a rectangular matrix decomposition, which uses the same minimal number of beam splitters, but is more symmetric and also provides the minimal optical depth of N (figure \ref{ReckClements} (b)) \cite{Clements:16}. Therefore, this decomposition method is much more scalable and even applicable for quantum simulation purposes. 

\begin{figure}[t]
    \centering
    \caption[Reck and Clements Scheme]{(a) Reck's scheme is based on a triangle. The optical depth is 2N-3 (in this example 7 which can be seen by following the longest possible path). Every blue rectangle represents a MZI with one phase shifter before and one after it. (b) The Clements scheme is based on a bricklayer structure providing the optimal optical depth of N with the same number of N(N-1)/2 MZI.}
    \label{ReckClements}
\end{figure}

\subsection{A Note on Conventions}
The real Givens rotation matrix can also be seen as a Mach-Zehnder Interferometer (MZI) or a variable beam splitter with a phase shifter in front of the MZI. A MZI consists of two 50:50 beam splitters and a phase shifter in one of the two arms. A basic layout can be seen in figure \ref{MZI}. \\
This can be described by the following matrix:
\begin{align}
    MZI = \begin{pmatrix}
        cos(\theta)& sin(\theta) \\
        sin(\theta) & cos(\theta)
    \end{pmatrix}
\end{align}
The 50:50 beam splitter matrix is not unique, equation \ref{RXgate} is one possibility and the one we are going to use. But one might also use the below matrices:
\begin{align}
    \frac{1}{\sqrt{2}}
    \begin{pmatrix}
        1& 1 \\
        1 & -1
    \end{pmatrix} 
    \;\; or \;\; 
        \frac{1}{\sqrt{2}}\begin{pmatrix}
        -1& 1 \\
        1 & 1
    \end{pmatrix} 
\end{align}
To get to the imaginary givens matrix in equation \ref{Tmn}, we have to add another phase shifter after every MZI. This also agrees with the theorem, that every 2x2 matrix can be represented by using three Pauli rotation matrices with free parameters. \cite{Clements:16} is using a different convention for the beamsplitter, already incorporating the last phase shifter in the variable Beamsplitter:
\begin{align}
    T_{m,n}(\theta,\phi) = BS(\theta,0)PS(\phi)
\end{align}
We will not use the Clements notation and stick to the longer form. \\
Since the 50:50 beam splitter can be represented by a $\widehat{X}(\frac{\pi}{2})$ gate and a phase shifter by a $\widehat{Z}(\theta)$ gate, we arrive at the gate sequence 
\begin{align}
    T_{m,n}(\alpha,\beta)=\widehat{Z}(\theta)\widehat{X}(\frac{\pi}{2})\widehat{Z}(\phi)\widehat{X}(\frac{\pi}{2})\widehat{Z}(\psi)
    \label{Z3X2}
\end{align}
which we will call in the following Z3X2.