\section{Random Noise on the Gate}

\textbf{Multiplicative Noise Model}
\begin{figure}[htb]
  \centering
  \caption[Multiplicative Noise Model for DFT]{Noise is applied on every Z gate's angle via $\theta=\theta_0(1+\sigma)$. Each system size was 100 runs. The fidelity plotted against the noise is shown for a random unitary. Lines are drawn to improve readability. To reach a Fidelity of better than $1-\mathcal{F} = 10^{-3}$, we need a noise in the range of $10^{-3}$}
  \label{fig:multnoise}
\end{figure}

\begin{figure}[htb]
  \centering
  \caption[Multiplicative Noise Model for permutation matrix]{Noise is applied on every Z gate's angle via $\theta=\theta_0(1+\sigma)$.Each system size was 100 runs. The fidelity plotted against the noise is shown for a permutation matrix where the input states were randomly chosen. Lines are drawn to improve readability. To reach a Fidelity of better than $1-\mathcal{F} = 10^{-3}$, we need a noise in the range of $10^{-3}$}
  \label{fig:multnoise}
\end{figure}

\begin{figure}[t]
    \centering
    \caption[Fidelity plotted against the System Size]{(a) The error scales linearly with the system size N. The line is a linear fit to improve readability. The data was taken for an error of $\sigma=10^{-3}$. (b) The matrix decomposition is robust against different input states.}
    \label{fig:multvsN}
\end{figure}

\textbf{Additive Noise Model}
\begin{figure}[htb]
  \centering
  \begin{subfigure}[t]{0.49\linewidth}
    \label{fig:mult1}
  \end{subfigure}
  \hfill
  \begin{subfigure}[t]{0.49\linewidth}
    \label{fig:mult2}
  \end{subfigure}
 \caption[Additive Noise Error Model for DFT and permutation]{Noise is applied on every Z gate's angle via $\theta=\theta_0 + \sigma$. The diagrams are for 100 runs each. (a) Fidelity for a random Unitary, (b) for a permutation matrix. Lines are drawn to improve readability. To reach a Fidelity of better than $1-\mathcal{F} = 10^{-3}$, we need a noise in the range of $10^{-3}$ to $10^{-2}$}
  \label{fig:multnoise}
\end{figure}

\section{Correlated Errors}
When applying the $\widehat{Z}$ gates locally on the lattice, it is possible that a tail of the potential is leaking into the neighboring lattice site. To simulate how concise the beam waist has to be, we are applying a nearest neighbour crosstalk error onto the $\widehat{Z}$ gates per layer. We are including only nearest neighbors because we model the beam as an exponential, therefore the second order is negligible. 
\begin{figure}[t]
  \centering
  \caption[Correlated Crosstalk Error Model vs $\varepsilon$ for DFT]{(a) The diagram shows the crosstalk error for a DFT (N=10). The multiplicative noise model has been used and was applied on all $\widehat{Z}$ gates. To get below a Fidelity of $1-\mathcal{F} = 10^{-3}$, $\varepsilon < 10^{-3}$.}
  \label{fig:addnoise}
\end{figure}

\begin{figure}[t]
  \centering
  \caption[Correlated Crosstalk Error Model vs N for DFT]{If the system size increases, $\varepsilon$ has to decrease as well to still allow to get below the $10^{-3}$ threshold.}
  \label{fig:addnoise}
\end{figure}

\section{Focus on the Permutation Matrix}
We can consider different cases for the rearrangement. Lets first consider only one atom which is being rearranged in an otherwise empty lattice. Additionally, we are only working with computational basis states. If we are now decomposing the single nearest neighbor $\widehat{SWAP}$ matrix into the Z3X2 layout, we are finding the angles (0,0,0) for the three $\widehat{Z}$ gates. Therefore, when applying a multiplicative (or also additive) error to the gate's angle, we are not getting any diagonal elements in our reconstructed $\widehat{SWAP}$ gate and we will not pick up any error on our $\widehat{SWAP}$ besides a phase. But we do pick up an off diagonal error for the Identity matrix, the angles in the decomposition are ($\pi$,$\pi$,0). This does not result in a reduced fidelity in this case since we are only looking at one atom in a lattice and since there are no other atoms which can pick up an error through the identity the fidelity should always be 1. \\
This in itself is already a pretty interesting result but it gets even more interesting: We now want to consider one swapping atom but also one atom which is only exposed to identity gates (figure (b)). The swapping atom, as discussed, does not pick up an error during the rearrangement process. The other atom, on the other hand, will pick up an error due to the imperfect implementation of the identity matrix. The off-diagonal error is of order $\mathcal{O}(\varepsilon)$ which can be seen by Taylor-expanding the because of the error non-zero off-diagonal element $1-e^{-i\pi\varepsilon}$. There is also the option to be able to implement the identity perfectly and have the error on the $\widehat{SWAP}$ gate by switching the second $\widehat{X}$ gate to $\widehat{X}(3\pi/2)$. This can be advantageous, depending on the desired matrix.
Another interesting question is what happens if when implementing the identity, there is another Fermion in the second site. Pauli blocking should prevent any interaction??????

\begin{figure}[t]
    \centering
    \caption[]{(a) With only one atom in the lattice, the fidelity in our error model will always be 1 since we do not pick up any error in the SWAP gate. (b) The swapping atom in site 2 is not picking up any error, the atom in lattice site 3, on the other hand, will pick up an error because of the imperfect implementation of the identity. (c) In this case, any operation on the lattice sites 3 and 4 is not being implemented because of pauli blocking. Therefore, we will not pick up any error by applying the identity so our fidelity will be 1}
    \label{fig:enter-label}
\end{figure}