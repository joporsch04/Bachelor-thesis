\documentclass[12pt, reqno, titlepage]{amsart}
\usepackage{amsmath, amsfonts, amsthm, amssymb, amsxtra, enumerate, mathtools, mathabx}
\usepackage[a4paper, top=3cm, bottom=3cm, left=2.1cm, right=2.1cm]{geometry}
\usepackage[normalem]{ulem}
\usepackage[utf8]{inputenc}
\usepackage[T1]{fontenc}
\usepackage{lmodern}
\usepackage{bbm}
\usepackage{mathrsfs}
\usepackage{enumerate}
\usepackage{comment}
\usepackage{placeins}
\usepackage{pgfplots}
\usepackage{mathabx}
\pgfplotsset{compat=newest}
\usepgfplotslibrary{fillbetween}
\usepackage{tikz}
\usepackage{csquotes}
\usepackage{float}
\usepackage{braket}

\renewcommand{\chaptername}{}

\newcommand{\A}{\mathbf{A}}
\newcommand{\B}{\mathfrak{B}}
\newcommand{\ball}{B}
\newcommand{\C}{\mathbb{C}}
\newcommand{\cl}{\mathrm{cl}}
\newcommand{\const}{\mathrm{const}\ }
\newcommand{\D}{\mathcal{D}}
\newcommand{\dd}{\, \mathrm{d}}
\newcommand{\eps}{\varepsilon}
\renewcommand{\epsilon}{\varepsilon}
\newcommand{\e}{\mathrm{e}}
\newcommand{\ii}{\mathrm{i}}
\newcommand{\id}{\mathbb{I}}
\newcommand{\I}{\mathbb{I}}
\newcommand{\Lc}{\mathcal{L}}
\newcommand{\F}{\mathcal{F}}
\newcommand{\loc}{{\rm loc}}
\newcommand{\mg}{\mathrm{mag}}
\newcommand{\N}{\mathbb{N}}
\newcommand{\norm}[2][]{{\left\|#2\right\|}} \newcommand{\ope}{\mathrm{op}}
\renewcommand{\phi}{\varphi}
\newcommand{\R}{\mathbb{R}}
\newcommand{\Rp}{\text{Re\,}}
\newcommand{\sclp}[2][]{{\left\langle#2\right\rangle} _{#1}}
\newcommand{\abs}[2][]{{\left\vert#2\right\vert}} 
\newcommand{\Sph}{\mathbb{S}}
\newcommand{\T}{\mathbb{T}}
\newcommand{\w}{\mathrm{weak}}
\newcommand{\Z}{\mathbb{Z}}
\newcommand{\Hs}{\mathcal{H}}
\newcommand{\Hh}{\mathbb{H}}
\newcommand{\Gs}{\mathcal{G}}
\newcommand{\Fs}{\mathcal{F}}
\newcommand{\Ks}{\mathcal{K}}
\newcommand{\Ac}{\mathscr{A}}
\newcommand{\Bc}{\mathscr{B}}
\newcommand{\Fc}{\mathscr{F}}
\newcommand{\lap}{\frac{\mathrm{d}^2}{\mathrm{d}x^2}}
\renewcommand{\epsilon}{\varepsilon}

\DeclareMathOperator{\codim}{codim}
\DeclareMathOperator{\dist}{dist}
\DeclareMathOperator{\Div}{div}
\DeclareMathOperator{\dom}{dom}
\DeclareMathOperator{\im}{Im}
\DeclareMathOperator{\ran}{ran}
\DeclareMathOperator{\re}{Re}
\DeclareMathOperator{\spec}{spec}
\DeclareMathOperator{\supp}{supp}
\DeclareMathOperator{\sgn}{sgn}
\DeclareMathOperator{\Tr}{Tr}
\DeclareMathOperator{\tr}{Tr}
%%magnetic/antisymmetric laplacian
\DeclareMathOperator{\nablaAB}{\nabla_{{\bf A}}}
\DeclareMathOperator{\DeltaAB}{\Delta_{{\bf A}}}
\DeclareMathOperator{\nablaAnti}{\nabla_{{\bf as}}}
\DeclareMathOperator{\DeltaAnti}{\Delta_{{\bf as}}}
\DeclareMathOperator{\BSop}{\mathcal{B}}
\DeclareMathOperator{\BSopAB}{\mathcal{B}_{{\bf A}}}
\DeclareMathOperator{\BSopAnti}{\mathcal{B}_{{\bf as}}}




\begin{document}

\begin{titlepage}
    \centering
    \vspace*{1.5cm}
    
    {\Large Ludwig-Maximilians-Universität München\\[0.3cm]
    Fakultät für Mathematik und Naturwissenschaften}
    
    \vspace{1.5cm}
    
    {\huge\bfseries Titel der Bachelorarbeit\\[0.4cm]
    \Large Untertitel (falls gewünscht)}
    
    \vspace{2cm}
    
    {\Large test title bachelor6}
    
    \vspace{2cm}
    
    {\Large Vor- und Nachname\\[0.2cm]
    Matrikelnummer: 12345678}
    
    \vfill
    
    {\Large \today}
    
\end{titlepage}






% Abstract (optional)
\begin{abstract}
    Hier folgt eine kurze Zusammenfassung der Arbeit. Erläutere in wenigen Sätzen das Thema, die Methodik und die wichtigsten Ergebnisse.
\end{abstract}
\thispagestyle{empty}
\newpage

% Danksagung (optional)
\section*{Danksagung}
Hier kannst du dich bei denjenigen bedanken, die dich während der Arbeit unterstützt haben. 
\thispagestyle{empty}
\newpage

% Inhaltsverzeichnis
\tableofcontents
\thispagestyle{empty}
\newpage

% Abbildungsverzeichnis (optional)
\listoffigures
\thispagestyle{empty}
\newpage

% Tabellenverzeichnis (optional)
\listoftables
\thispagestyle{empty}
\newpage

% Beginn des Hauptteils
\section{Einleitung}
Die Einleitung gibt einen Überblick über das Thema der Arbeit, die Motivation, sowie den Aufbau der Arbeit.

\section{Grundlagen}
Beschreibe hier die theoretischen Grundlagen, Notationen und Definitionen, die in der Arbeit benötigt werden.

\section{Hauptteil}
\subsection{Erster Unterabschnitt}
Beschreibe hier den ersten inhaltlichen Teil deiner Arbeit.

\subsection{Zweiter Unterabschnitt}
Weiterführende Erklärungen oder Resultate können hier dargestellt werden.

\begin{equation*}
    \partial \A = \B
\end{equation*}

\medskip

\begin{equation}
    \int_{\R^d} \abs{f(x)}^2 \dd x = \int_{\R^d} \abs{\F f(\xi)}^2 \dd \xi
\end{equation}

\medskip

\begin{equation}
    %schrödinger equation
    \ii \partial_t u = \mathcal{H}(t) \Ket{a}
\end{equation}

\section{Ergebnisse und Diskussion}
Fasse hier die wichtigsten Ergebnisse zusammen und diskutiere sie im Kontext des Themas.

\section{Fazit und Ausblick}
Gib ein abschließendes Fazit und einen Ausblick auf mögliche weitere Forschungen oder Anwendungen.

% Literaturverzeichnis
\begin{thebibliography}{99}
    \bibitem{ref1} Autor, \emph{Titel}, Verlag, Jahr.
    % Weitere Literaturangaben
\end{thebibliography}

% Anhang (optional)
\appendix
\section{Anhang}
Hier können ergänzende Berechnungen, Tabellen oder Abbildungen eingefügt werden.

\end{document}
