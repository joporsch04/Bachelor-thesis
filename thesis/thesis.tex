\documentclass[12pt]{report}
\usepackage[a4paper, top=3cm, bottom=3cm, left=2.1cm, right=2.1cm]{geometry}


\usepackage[normalem]{ulem}
\usepackage[utf8]{inputenc}
\usepackage[T1]{fontenc}
\usepackage{lmodern}
\usepackage{hyperref}


\usepackage{comment}
\usepackage{placeins}
\usepackage{pgfplots}

\pgfplotsset{compat=newest}
\usepgfplotslibrary{fillbetween}

\usepackage{tikz}
\usepackage{csquotes}
\usepackage{float}

\usepackage{amsmath, amsfonts, amssymb, enumerate, mathtools, mathabx, bbm, mathrsfs} % mathematical packages 
\usepackage{braket}


\newcommand{\uvec}[1]{\underline{#1}}
\renewcommand{\vec}[1]{\mathbf{#1}}

\newcommand{\U}{\mathcal{U}}
\newcommand{\Ls}{\mathcal{L}}
\newcommand{\Os}{\mathcal{O}}

\newcommand{\A}{\mathbf{A}}
\newcommand{\B}{\mathfrak{B}}
\newcommand{\ball}{B}
\newcommand{\C}{\mathbb{C}}
\newcommand{\cl}{\mathrm{cl}}
\newcommand{\const}{\mathrm{const}\ }
\newcommand{\D}{\mathcal{D}}
\newcommand{\dd}{\mathrm{d}}
\newcommand{\eps}{\varepsilon}
\newcommand{\e}{\mathrm{e}}
\newcommand{\ii}{\mathrm{i}}
\newcommand{\id}{\mathbb{I}}
\newcommand{\I}{\mathbb{I}}
\newcommand{\Lc}{\mathcal{L}}
\newcommand{\F}{\mathcal{F}}
\newcommand{\loc}{{\rm loc}}
\newcommand{\mg}{\mathrm{mag}}
\newcommand{\N}{\mathbb{N}}
\newcommand{\norm}[2][]{{\left\|#2\right\|}} \newcommand{\ope}{\mathrm{op}}
\newcommand{\R}{\mathbb{R}}
\newcommand{\Rp}{\text{Re\,}}
\newcommand{\sclp}[2][]{{\left\langle#2\right\rangle} _{#1}}
\newcommand{\abs}[2][]{{\left\vert#2\right\vert}} 
\newcommand{\Sph}{\mathbb{S}}
\newcommand{\T}{\mathbb{T}}
\newcommand{\w}{\mathrm{weak}}
\newcommand{\Z}{\mathbb{Z}}
\newcommand{\Hs}{\mathcal{H}}
\newcommand{\Hh}{\mathbb{H}}
\newcommand{\Gs}{\mathcal{G}}
\newcommand{\Fs}{\mathcal{F}}
\newcommand{\Ks}{\mathcal{K}}
\newcommand{\Ac}{\mathscr{A}}
\newcommand{\Bc}{\mathscr{B}}
\newcommand{\Fc}{\mathscr{F}}
\newcommand{\Hc}{\mathscr{H}}
\newcommand{\lap}{\frac{\mathrm{d}^2}{\mathrm{d}x^2}}

\DeclareMathOperator{\codim}{codim}
\DeclareMathOperator{\dist}{dist}
\DeclareMathOperator{\Div}{div}
\DeclareMathOperator{\dom}{dom}
\DeclareMathOperator{\im}{Im}
\DeclareMathOperator{\ran}{ran}
\DeclareMathOperator{\re}{Re}
\DeclareMathOperator{\spec}{spec}
\DeclareMathOperator{\supp}{supp}
\DeclareMathOperator{\sgn}{sgn}
\DeclareMathOperator{\Tr}{Tr}
\DeclareMathOperator{\tr}{Tr}
%%magnetic/antisymmetric laplacian
\DeclareMathOperator{\nablaAB}{\nabla_{{\bf A}}}
\DeclareMathOperator{\DeltaAB}{\Delta_{{\bf A}}}
\DeclareMathOperator{\nablaAnti}{\nabla_{{\bf as}}}
\DeclareMathOperator{\DeltaAnti}{\Delta_{{\bf as}}}
\DeclareMathOperator{\BSop}{\mathcal{B}}
\DeclareMathOperator{\BSopAB}{\mathcal{B}_{{\bf A}}}
\DeclareMathOperator{\BSopAnti}{\mathcal{B}_{{\bf as}}}




\begin{document}


\begin{titlepage}
    \centering
    
    {\Large Bachelorarbeit}
    
    \vspace{1.5cm}
    
    {\huge\bfseries Rekonstruktion der Starkfeldionisierungsdynamik\\[0.4cm]
    \Large Attosekundenphysik}
    
    \vspace{2cm}
    
    {\Large Johannes Porsch}
    
    \vspace{2cm}
    
    {\Large Ludwig-Maximilians-Universität München}
    
    \vfill
    
    \includegraphics[width = 0.4\textwidth]{figures/sigillum.png}

    \vfill
    
    {\Large \today}
    
\end{titlepage}
\newpage


\begin{abstract}
    Multiphoton ionization of atoms in strong laser fields is a fundamental process in attosecond physics. In this work, we extend the strong-field approximation (SFA) by incorporating the influence of excited atomic states on ionization rates. Standard SFA formulations neglect these excited states, assuming that the laser field has no effect on the atom before ionization. However, in intense few-cycle laser pulses, the Stark shift and transient population of excited states can significantly modify ionization dynamics. We numerically solve the time-dependent Schrödinger equation (TDSE) using the tRecX code to extract time-dependent probability amplitudes for hydrogen’s ground and excited states. These amplitudes are then integrated into the SFA formalism to evaluate their impact on ionization rates. 
\end{abstract}


\tableofcontents
\newpage


\chapter{Einleitung}
Ionization with an intense laser is one of the most fundamental processes in attosecond physics and quantum mechanics in general and has a broad range of applications such as from high-harmonic generation to photoelectron spectroscopy and medical physics.
Of prior interest is the `response' of the electron to certain parts of the laser pulse at a given moment of time.
From a classical point of view, the instantaneous ionization rate $\Gamma(t)$ is a way of making making this `wish' come true.
% Understanding the dynamics of this process requires theoretical models that can predict ionization probabilities and provide insights into the underlying mechanisms.
% In real life applications mostly we dont have static electric field, more a laser pulse, with an envelope.
However, quantum mechanics presents a fundamental challenge: we can only measure probabilities, not the specific path an electron takes during ionization.
Further, predicting the \emph{instantaneous} ionization of the electron at a given moment of time is difficult to reconcile with conventional quantum mechanics \cite{Ivanov2018}. 

\medskip
Lots of research effort has gtten into developing different ways of defining ionization rates \cite{agarwal2025generalapproximatorstrongfieldionization,Ivanov2018}.
The Strong Field Approximation (SFA) has become one of the standard theoretical frameworks for deriving ionization rates due to its simplicity and reasonably good agreement with numerical solutions of the time dependent schroedinger equation. 
The way, that it reconciles with quanum mechanics is that within SFA there is an expression that can be interpreted as an ionization rate that share similar propetrties of both vlassical and quanum mechanical physics.A very

However, standard SFA formulations make a crucial simplification: they assume the laser field has no effect on the atom before ionization, effectively neglecting excited atomic states and treating the electron as remaining in the ground state until the moment of ionization.

This approximation leads to significant discrepancies when comparing SFA predictions with ab initio solutions of the time-dependent Schrödinger equation (TDSE). 
Numerical simulations using solvers like tRecX reveal ionization dynamics that differ substantially from standard SFA predictions, particularly in the temporal structure of the ionization process. 
When the laser pulse is symmetric in time, SFA predicts a symmetric ionization rate, but numerical simulations show this time symmetry is broken.

The central hypothesis of this thesis is that these discrepancies arise primarily from the neglect of excited atomic states in standard SFA formulations. 
In intense laser fields, effects such as the AC Stark shift and transient population of excited states can significantly modify the ionization dynamics. By including these effects, we can potentially bridge the gap between analytical SFA models and numerical TDSE solutions.

This thesis addresses the following key questions:
\begin{itemize}
    \item What is the role of excited atomic states in strong field ionization?
    \item How does the AC Stark shift influence ionization rates compared to changes in ground state population?
    \item Can we improve the accuracy of ionization rate predictions by incorporating excited state dynamics into the SFA framework?
\end{itemize}

To answer these questions, we extend the SFA formalism to include the influence of excited atomic states on ionization rates. 
We derive an improved expression for the IIR that incorporates time-dependent probability amplitudes for both ground and excited states, obtained by solving the TDSE within the subspace of bound states. 
The extended model is then validated through comparison with full TDSE solutions using the TIPTOE (Temporal Ionization Probability in Tunneling and Over-barrier Excursions) method.

The structure of this thesis is as follows: Chapter 2 derives the extended SFA rate and establishes the theoretical framework. 
Chapter 3 introduces the numerical methods, including the tRecX solver and the TIPTOE analysis technique. Chapter 4 details the implementation of the extended SFA model and discusses the computational challenges. 
Chapter 5 presents results comparing standard SFA, extended SFA, and full TDSE solutions, demonstrating the importance of excited state dynamics in strong-field ionization.


% Weither you want your system to ionize or not, it may be important to understand, how ionization works.
% However since it is a quantum mechanical process, one has to first ask what information are even experimentally acessible.
% Quantum mechanics at its very fundamental level tells us one can only measure probabilities and how likely the electron get ionized.
% Unfortunately, in an actual experiment thats the only option you got.
% Further one likes to know more about the "path" the electron takes after it gets struck by the laser pulse i.e the response of the electron to the laser pulse.

% Everytime you hear "path" in quantum mechanics, the first name to think of is Heisenberg.
% His uncertainty principle sets the boundaries for these type of questions and does not let measure all the information you wish to had.
% Further to see the electrons path, one has to probe the system just as lightwaves probe the matter around you letting you see.
% However probing a quantum mechanical system effectively changes the quantum mechanical state of the electron so eventually one has to restrict oneself to something different.

% The challenge of a theoretical model is now to predict ionization propabilities and further explain the underlying mechanisms and in the best case allowing some insights what happens between initial state of the atom and measured electron.
% A clever way to do this is by establishing the concept of an instantanious ionization rate (IIR) via the propertry $P_{\mathrm{ion}}=\int_{\R}\Gamma (t)\dd t$ effectively telling you when and where an electron reacts to a incomming laser pulse.
% At first sight, "instantanious" seems to conflict with the energy time uncertainty relation $\Delta E \Delta t \geq \hbar/2$.
% However, instantanious refers to the response of the electron to the laser field, not the and does not contain eny infromation on when exactly does the electron get ionized.

% Lets make an example to show how an IIR can be beneficial.
% Suppose a measurement is made, a laser pulse hits an atom and the outcome is measured, so if the electron is ionized or not.
% Suppose the measurement is repeated sufficiently many times, one can define a ionization propability for instance $1\%$.
% An IIR now allows to investigate which parts of the laser pulse causes what effects in the total ionization propability.
% One could say, an IIR tells us how likely the electron gets ionized within a certain time interval but of course averaged over all the measurements made.
% Note that the IIR does not tell us the electron will get ionized at a certain time because this would violate the energy time uncertainty relation, the value of $\Gamma$ at a certain time does not have any physical interpretation.

% This concept allows us to investigate the ionization process better and understand more about the underlying mechanisms such as defining different limits and regimes (2.4) or make use of the IIR by sampling a light pulse \cite{Park:18}.
% Due to the challange that the IIR is not really an quantum mechanical observable, it is difficult to have a working notion of it that works for different systems and laser pulses.
% Eventually one has to start from scratch and derive  the IIR from the time dependent schroedinger equation (TDSE) and varios approximations.
% One approximation most used in this context is called Strong field aproximation (SFA) mostly for its simplicity and the capability to derive ionization rates \cite{Theory_NPS}.
% However, SFA has its limitations, especially when it comes to comparing ab initio ionization propabilities, i.e. to numerical solutions of the TDSE using dedicated numerical solver.
% An implicit assumption often made while using SFA is that the laser field has no effect on the atom until the moment of ionization, effectively neglecting excited atomic states.
% In principle this does not have anything to do with the SFA itself, making it a possible solution to the differences between SFA and ab initio ionization propabilities.
% This is one of the main goal of this thesis, investigate if the differences from SFA and ab initio ionization propabilities can be explained by the neglect of excited states.
% Further one would like to investigate the dynamics and effects of the electron before it gets ionized and how they contribute to the ionization dynamics.
% Particulary important would be to know what matters more, the stark shift or the distortion of the ground state wavefunction.
% That could lead to an improvement to existing rates, making them besser applicable in the cases mentioned before.

% To achieve this, one has to review how the SFA was made previously and how it can be improved (Chapter 2).
% Later I will implement the extended SFA rate (Chapter 4) and compare the results with existing SFA rates and ab initio ionization propabilities (Chapter 5).




% What motivates this thesis? Background: developement of SFA and GASFIR rates that doesnt have to numerically solve Schroedinger equation.
% Comparison of ion rates from tRecX, SFA and GASFIR. When the laser pulse is an even function in time, the SFA rate is that as well. 
% But numerical simulations from tRecX tell us thats not the case and the time symmetry is broken. Idea: because of the neglected excitedt states in SFA.
% This brings up more questions: What role play excited states in ionization? Does the stark effect play an important role?\\
% Why is this so complicated? First, $[\hat{\Hs}(t), \hat{\Hs}(t')] \neq 0$ because $\hat{\Hs_0}$ and $\hat{V}$ dont share same eigenbasis, -> the electron is free. Also the thing with all these gauges.



% why so promising? coulomb potential has little effect on ionization dynamics (found manoram)\\
% The only thingmissing is excited states

% This thesis tries to answer the following questions:
% \begin{itemize}
%     \item What is the role of excited states in strong field ionization?
%     \item How does the Stark effect influence ionization rates?
%     \item Can we improve the accuracy of ionization rate predictions without solving the TDSE?
% \end{itemize}

% One of the key findings of this thesis is 
\newpage


\chapter{Grundlagen}
Lorem ipsum dolor sit amet, consetetur sadipscing elitr, sed diam nonumy eirmod tempor invidunt ut labore et dolore magna aliquyam erat, sed diam voluptua. At vero eos et accusam et justo duo dolores et ea rebum. Stet clita kasd gubergren, no sea takimata sanctus est Lorem ipsum dolor sit amet. Lorem ipsum dolor sit amet, consetetur sadipscing elitr, sed diam nonumy eirmod tempor invidunt ut labore et dolore magna aliquyam erat, sed diam voluptua. At vero eos et accusam et justo duo dolores et ea rebum. Stet clita kasd gubergren, no sea takimata sanctus est Lorem ipsum dolor sit amet.

\begin{equation}
    \partial_t u = \mathcal{H}(t)  \lambda 
\end{equation}

\begin{figure}[H]
    \centering
    \includegraphics[width=0.5\textwidth]{figures/plot.pdf}
    \caption{Sine function}
    \label{fig:sinus}
\end{figure}




\newpage
\section{Basic Formalism}

Basic Definitions of schröfinger qe, light dyson series, and sotrong field s matrix


\newpage
\section{Strong Field Approximation}

clear defintition of the strong field approximation, and the assumptions that are made.



\newpage
\section{Strong Field Ionization}

Derivation of 

\begin{equation}
    \lim_{t \to \infty} \ket{\Psi(t)}  = -i \int d^3 p\, \ket{\mathbf{p}} \int_{-\infty}^{\infty} dt'\, e^{-\frac{i}{2}\int_{t'}^{\infty} [\mathbf{p}+\mathbf{A}(t')]^2 \, dt'} e^{i I_\mathrm{p} t'} \langle \mathbf{p} + \mathbf{A}(t') | \hat{\mathbf{d}} \cdot \mathbf{E}(t') | \Psi_0 \rangle
\end{equation}



\newpage
\section{Multiphoton Ionization}

Different types of Ionization, tunneling Ionization, multiphoton
\newpage


\chapter{Hauptteil}
This chapter is about introducing the methods used in the thesis.
This includes numerical methods for solving or implementing some type of equation as well as more physical methods that are used for comparing results or get more meaning to the data.

The main reason for making these ionization models is because one want to avoid solveng the schroedinger equation numerically because its time consuming and contains numerical difficulties.
However, for comparing and verifying the models, one needs to solve it numerically.\\
Furthermore, numerically solving the TDSE is just good for ionization propabilities and not rates.\\
Especially for expanding an already existing model, it is important to compare and see if the changes of the model are going into the right direction.



%%%%%%%%%%%%%%%%%%%%%%%
\section{Numerical Methods}
To implement formula \ref{eq:smatrix} we need to solve the TDSE in two different ways numerically in order to obtain the coefficients $c_n(t)$. 
For that I used two different methods (tRecX and ODE) each with its own advantages and disadvantages.
Further, there is this dilemma with ionization rates not being quantum mechaincal observables, so we need to have trustfull ionization propabilities to compare my results.



\subsection{tRecX}
tRecX is a C++ code for solving generalized inhomogeneous time-dependent
Schroedinger-type equations in arbitrary dimensions and in a variety of coordinate systems \cite{Scrinzi_trecx}.
Usually ionization is difficult for an numerical solver because the electron "leaves" the subspace of the bound states, making the calculations very time consuming.
tRecX uses different techniques (irECS and tSURFF) making it especially suitable for light atom interaction. 
Further, tRecX allows to specify the gauge in which the TDSE is solved, making it more flexible. 
This will be important later.

% \paragraph{\textcolor{red}{irECS}}
% Lorem ipsum dolor sit amet, consetetur sadipscing elitr, sed diam nonumy eirmod tempor invidunt ut labore et dolore magna aliquyam erat, sed diam voluptua. At vero eos et accusam et justo duo dolores et ea rebum. Stet clita kasd gubergren, no sea takimata sanctus est Lorem ipsum dolor sit amet. Lorem ipsum dolor sit amet, consetetur sadipscing elitr, sed diam nonumy eirmod tempor invidunt ut labore et dolore magna aliquyam erat, sed diam voluptua. At vero eos et accusam et justo duo dolores et ea rebum. Stet clita kasd gubergren, no sea takimata sanctus est Lorem ipsum dolor sit amet.
% \paragraph{\textcolor{red}{tSURFF}}
% Lorem ipsum dolor sit amet, consetetur sadipscing elitr, sed diam nonumy eirmod tempor invidunt ut labore et dolore magna aliquyam erat, sed diam voluptua. At vero eos et accusam et justo duo dolores et ea rebum. Stet clita kasd gubergren, no sea takimata sanctus est Lorem ipsum dolor sit amet. Lorem ipsum dolor sit amet, consetetur sadipscing elitr, sed diam nonumy eirmod tempor invidunt ut labore et dolore magna aliquyam erat, sed diam voluptua. At vero eos et accusam et justo duo dolores et ea rebum. Stet clita kasd gubergren, no sea takimata sanctus est Lorem ipsum dolor sit amet.



% \paragraph{Challenges}
% dangling pointer: interesting problem actually, how to solve it, how to find it, etc




\subsection{ODE}
To acess the coefficients of the wavefunction inside the subspace of the bound states I used the interaction picture and the coupled system of differential equations. 
To implement the calculation I used Python and the ODE solver from SciPy integrate called \texttt{solve\_ivp}.
For initial values I choosed \texttt{c\_n(0) = 1} for the ground state and \texttt{c\_n(0) = 0} for all other states.
The code can be found in the electronic appendix in the class \texttt{HydrogenSolver}.
For more details about the implementation, see chapter 4.






%%%%%%%%%%%%%%%%%%%%%%%%%
\section{TIPTOE}
This section mostly follows \cite{Park:18} and \cite{manorammasterthesis}.
To verify the ionization rates determined by the models, a comparison with experimentally accessible quantities is necessary.
However, since an ionization rate is not a quantum mechanical observable, only the ionization yield $\int \Gamma(t) \dd t$ can be measured.

TIPTOE \cite{Park:18} (Tunneling Ionization with a Perturbation for the Time-domain Observation of an Electric field) is a method for direct sampling of an electric pulse in the femtosecond to attosecond regime using quasistatic subcycle tunneling ionization in a gaseous medium or air.
Tunneling ionization differs from multiphoton ionization in that the laser field is strong enough to deform the Coulomb barrier of the atom, allowing the electron to tunnel through it.

A typical TIPTOE simulation consists of two linearly polarized laser pulses: a "fundamental" and a "signal" pulse, similar to common pump-probe experiments.
The drive pulse is the pulse to be sampled, with the ionization yield of a certain medium providing the measurement.
In first order, the ionization rate can be approximated as
\begin{equation}
\Gamma(E_{\mathrm{F}}+E_{\mathrm{S}})\approx\Gamma(E_{\mathrm{F}})+\left.E_{\mathrm{S}}\frac{\dd \Gamma(E_{\mathrm{S}})}{\dd E}\right|{E=E{\mathrm{F}}}
\end{equation}
In this approximation, depletion of the ground state is neglected.
The total ionization yield $N$ obtained by the two pulses is given by
\begin{equation*}
N_{\mathrm{total}}=N_0+\delta N = \int \dd t,\Gamma(E_{\mathrm{F}}(t))+\int \dd t,E_{\mathrm{S}}(t)\left.\frac{\dd \Gamma(E_{\mathrm{S}}(t))}{\dd E}\right|{E=E{\mathrm{F}}(t)}
\end{equation*}
By varying the delay $\tau$ between the two pulses, the ionization yield takes on different values.
From this, it follows that:
\begin{equation}
\delta N(\tau)\propto E_{\mathrm{S}}(\tau) \label{eq:tiptoeprop}
\end{equation}
Thus, the field amplitude of the signal pulse can be sampled by measuring the ionization yield for different delays.
The TIPTOE method can be applied across a broad spectral range of the signal pulse, as long as the fundamental pulse is shorter than $1.5$ optical cycles.

This method provides a way to compare both approaches and gain insight into the dynamics of the electron.
TIPTOE is particularly useful because numerical simulations can provide good predictions about ionization probabilities, while analytical models describe ionization rates.
A TIPTOE simulation can help reconstruct the ionization dynamics, which is especially relevant in the context of this thesis.
Later, the ionization rate of $E_{\mathrm{F}}+E_{\mathrm{S}}$ will be integrated over the full time domain, and the ionization yield for different delays will be compared with the results from the numerical solution of the TDSE.
The results of the TIPTOE simulation from the numerical solution of the TDSE (tRecX) compared to the approximate methods (variations of SFA) are shown in the figures in Chapter 5.

For better visualization of the underlying physics in TIPTOE, the "background" ionization from $E_F(t)$ is subtracted, and the ionization yield is normalized, so the formula in plot ???? reads
\begin{equation*}
\frac{N_{\mathrm{total}}-N_0}{N_{\mathrm{max}}}=\frac{\delta N(\tau)}{N_{\mathrm{max}}}
\end{equation*}
However, interesting physics can also be observed by comparing the net ionization yield $N_{\mathrm{total}}$, as discussed later in Chapter 5.

Typically, TIPTOE is not used for this kind of analysis but rather for its sampling capabilities.
Instantaneous ionization rates are highly useful because TIPTOE enables sampling of the electric field of a laser pulse in the femtosecond to attosecond regime, which has broad applications in fields such as laser spectroscopy and medical physics.






%%%%%%%%%%%%%%%%%%
\section{GASFIR}
GASFIR is short for general approximator for strong field ionization rates.
It is an analytical retrieval tool for reconstructing data obtaned from numerical solutions of the TDSE. 
It was validated for hydrogen and shows good agreement with existing theories in the quasi static limit of tunneling ionization not only for Hydrogen, also for Helim and Neon [cite GASFIR].
The way how GASFIR works is that it uses ionization probabilities to retrieve ionization rates. 
It uses the idea from SFA that the rates can be written as $\int \dd T K(t,T)$ with $K(t,T)$ being a kernel function.
Later in the code you see also the kernel function, where the modifications happened. 
Its not really part of this thesis, but motivates it.
Goal: make GASFIRs predictions better by improving SFA rates




% \begin{align*}
%     K(t,T) &= E_\mathrm{+}E_\mathrm{-} \int_0^{\infty} dp\, p^2 \int_0^\pi d\theta \sin\theta e^{ i T (p^2+2\overline{\Delta A} p \cos{\theta})} \\
%            &\quad \times \int_0^{2\pi}d\phi \,d_z^*\bigl(\bm{p} + \bm{e}_z A_\mathrm{+}\bigr) d_z\bigl(\bm{p} + \bm{e}_z A_\mathrm{-}\bigr) e^{ i T (2I_\mathrm{p} + \overline{\Delta A^2})}.
% \end{align*}

\newpage


\begin{thebibliography}{99}
    \bibitem{ref1} Autor, \emph{Titel}, Verlag, Jahr.
\end{thebibliography}
\newpage


\appendix
\chapter{Anhang}
Solving TDSE for Hydrogen atom because it will be important later. also note the structure of wavefunction:
\begin{equation*}
    \psi_{nlm}(\uvec{x}) = \psi_{nlm}(r,\theta,\phi) = R_{nl}(r)Y_{lm}(\theta, \phi)
\end{equation*}
$E_m=\frac{Z^2}{2n^2}$
\newpage

\chapter{Weiterer Anhang}
% Solving TDSE for Hydrogen atom because it will be important later. also note the structure of wavefunction:
% \begin{equation*}
%     \psi_{nlm}(\uvec{x}) = \psi_{nlm}(r,\theta,\phi) = R_{nl}(r)Y_{lm}(\theta, \phi)
% \end{equation*}
% $E_n=\frac{Z^2}{2n^2}$

\label{sec:dipolematrixelements}


Here, the general transition dipole matrix elements into the continuum for a hydrogen atom are derived.     %hydrogenlike???
The general matrix element in this case is given by:
\begin{equation*}
    \uvec{d}_{nlm}(\uvec{p}) = \braket{\Pi|\vec{\hat{d}}|\Psi_{nlm}} \stackrel{\mathrm{a.u.}}{=} \braket{p|\vec{\hat{r}}|\Psi_{nlm}}
\end{equation*}
where $\ket{p}$ is a plane wave.
The wave function for the hydrogen atom is well known:
\begin{equation}
    \Psi_{nlm}(\uvec{x}) = \braket{\uvec{x}|\Psi_{nlm}} = R_{nl}(r)Y_{lm}(\theta, \phi) \label{eq:psi_position_hydrogen}
\end{equation}
with $R_{nl}(r)$ being the radial part of the wavefunction and $Y_{lm}(\theta, \phi)$ being the spherical harmonics.

By partitioning the $\hat{\vec{1}}$ operator and using the fact that $\vec{\hat{r}} \rightarrow i\nabla_{\uvec{p}}$ %Interesting because by choosing a certain way to display the operator we also choose a natural basis for the representation of the operator. Therefore I write \nabla_{\uvec{p}} instead of \nabla_{\vec{p}}.
in momentum representation, a general formula for the transition is obtained:
\begin{equation*}
    \uvec{d}_{nlm}(\uvec{p}) = i\nabla_{\uvec{p}}\int \dd^3\uvec{x}\,\psi_{nlm}(\uvec{x}) e^{-i\uvec{p}\cdot\uvec{x}} = i\nabla_{\uvec{p}}\phi_{nlm}(\uvec{p})
\end{equation*}
In principle, this integral (i.e the Fourier transform) is all that is required.
Due to the structure of $\psi_{nlm}$, a result similar to \eqref{eq:psi_position_hydrogen} can be expected.
A posteriori, it can be shown that:
\begin{equation*}
    \F\{\psi_{nlm}(\uvec{x})\} = \phi_{nlm}(\uvec{p}) = F_{nl}(p)Y_{lm}(\theta_p, \phi_p)
\end{equation*}
where $F_{nl}(p)$ is the Fourier transform of the radial part of the wavefunction and $Y_{lm}(\theta_p, \phi_p)$ are the spherical harmonics in momentum space, analogous to the hydrogen atom in position space.






%%%%%%%%%%%%%%%%%%%%%%
\subsection*{Momentum space}
The so-called plane wave expansion \cite{Jackson:1998nia} of the exponential part of the integral is given by:
\begin{equation*}
    e^{i\uvec{p}\cdot\uvec{x}} = \sum_{l'=0}^\infty (2l'+1)i^{l'} j_{l'}(pr) P_{l'}(\uvec{p}\cdot\uvec{x}) = 4\pi\sum_{l'=0}^\infty \sum_{m'=-l'}^{l'} i^{l'} j_{l'}(pr) Y_{l'm'}(\theta_p, \phi_p) Y_{l'm'}^*(\theta_x, \phi_x)
\end{equation*}
Here, $j_l(pr)$ represents the spherical Bessel functions, and the integration is performed over spherical coordinates. 
While the expression initially appears complex, the orthogonality of the spherical harmonics can be used to simplify the integral to:
\begin{equation*}
    \phi_{nlm}(\uvec{p}) = 4\pi \sum_{m=-l}^{l}Y_{lm}(\theta_p,\phi_p)i^l\underbrace{\int_{0}^{\infty}\dd r\,r^2j_l(pr)R_{nl}(r)}_{\tilde{R}_{nl}(p)}
\end{equation*}
This structure is the desired form. The focus now shifts to the radial part $\tilde{R}_{nl}(p)$ of the integral.
The term $R_{nl}(r)$ corresponds to the radial function of the hydrogen atom in position space and is independent of the magnetic quantum number $m$.
It consists of an exponential term dependent on $r$, a polynomial term dependent on $r$, the generalized Laguerre polynomials, and the normalization constant.
A closed expression for the generalized Laguerre polynomials would be convenient. They are represented as:
\begin{equation*}
    L_n^l(r) = \sum_{\iota=0}^{n} \frac{(-1)^{\iota}}{\iota!}\binom{n+l}{n-\iota}r^{\iota}
\end{equation*}
Thus, the Laguerre polynomials depend only on an exponential term and finitely many polynomial terms.
$\tilde{R}_{nl}(p)$ can be expressed (without prefactors and summation over $\iota$) as:
\begin{equation*}
    \int_{0}^{\infty}\dd r\,r^{2+l+\iota} e^{-\frac{Zr}{n}} j_l(pr)
\end{equation*}
Before solving the integral using computational methods, the spherical Bessel function must be transformed into ordinary Bessel functions:
\begin{equation*}
    j_l(pr) = \sqrt{\frac{\pi}{2pr}}J_{l+\frac{1}{2}}(pr)
\end{equation*}
At this stage, it is useful to combine all prefactors and summations into a single expression and examine the integral as a whole:
\begin{equation}
\label{eq:phi_nlm_momentum}
\begin{aligned}
    \phi_{nlm}(\uvec{p}) =\ & \frac{\pi^{3/2}}{\sqrt{2p}}\sqrt{\left(\frac{2}{n}\right)^3\frac{(n-l-1)!}{n(n+1)!}}\\
    & \times \sum_{m=-l}^{l}\sum_{\iota=0}^{n-l-1}i^l\frac{(-1)^{\iota}}{\iota!}\left(\frac{2}{n}\right)^{l+\iota}\binom{n+l}{n-l-1}\underbrace{\int_{0}^{\infty}\dd r\,r^{l+\iota+\frac{3}{2}}e^{-\frac{Zr}{n}}J_{l+\frac{1}{2}}(pr)}_{(*)} Y_{lm}(\theta_p,\phi_p)
\end{aligned}
\end{equation}
The remaining integral was computed using Mathematica, so a detailed derivation is not provided. Interestingly, an analytical solution exists.
The result for $(*)$ is:
\begin{equation*}
    (*) = {}_2\tilde{F}_1\left(2 + l + \frac{\iota}{2}, \frac{1}{2}(5 + 2l + \iota); \frac{3}{2} + l; -\frac{n^2 p^2}{Z^2}\right)
\end{equation*}
Here, ${}_2\tilde{F}_1$ denotes the regularized hypergeometric function, defined as:
\begin{equation*}
    {}_2\tilde{F}_1(a,b;c;z) = \frac{{}_2F_1(a,b;c;z)}{\Gamma(c)} = \frac{1}{\Gamma(a)\,\Gamma(b)} \sum_{n=0}^{\infty} \frac{\Gamma(a+n)\,\Gamma(b+n)}{\Gamma(c+n)} \frac{z^n}{n!}
\end{equation*}
The final expression for $\phi_{nlm}(\uvec{p})$, which can also be found in \cite{Bransdenatomsmolecules} in a slightly different form, is given by:
\begin{align}
    \label{eq:phi_nlm}
    \phi_{nlm}(\uvec{p}) = \sum_{\iota=0}^{2l+1} \;
        & \frac{(-1)^{\iota} \; 2^{\iota + \frac{1}{2}} \; n \; (i n)^l \; (p^2)^{l/2} \; Z^{-l-3} \; \Gamma(2l+\iota+3)}{\iota!} \nonumber \\
        & \times \binom{l+n}{-l+n-\iota-1} 
        \sqrt{\frac{Z^3 \Gamma(n-l)}{\Gamma(l+n+1)}} \nonumber \\
        & \times Y_l^m(\theta_p, \phi_p) \,
        {}_2\tilde{F}_1\left(l+\frac{\iota}{2}+2, \frac{1}{2}(2l+\iota+3); l+\frac{3}{2}; -\frac{n^2 p^2}{Z^2}\right)
\end{align}






%%%%%%%%%%%%%%%%%%%%%%%
\subsection*{Improved rate}
Usually all that remains is to differentiate \eqref{eq:phi_nlm} with respect to $\uvec{p}$.
However, before that, the SFA rate can be improved directly by performing the integration over $\phi_p$ analytically.
The rate according to chapter 2 is given by:
\begin{align*}
    \Gamma_{\mathrm{SFA}}(t) &= \sum_{n_1}\sum_{n_2} \int_0^{\infty} \dd p\,p^2\int_0^{\pi} \dd\theta_p\,\sin\theta_p \int_0^{2\pi}\dd \phi_p\int_{-\infty}^{\infty} \dd T\\
    &\times \exp\left(i\vec{p}^2T + ip\cos\theta_p\bar{\Delta}_z^k +  \frac{i}{2}\bar{\Delta}_z^2  + i(t-T)E_{n_1}-i(t+T)E_{n_2}\right)\\
    &\times E_z(t-T) E_z(t+T)c_{n_1}^*(t-T)c_{n_2}(t+T) d_{z,n_1}^*(\uvec{p}+\Delta_z(-T))d_{z,n_2}(\uvec{p}+\Delta_z(T))
\end{align*}
As mentioned before, the dipole transition matrix element can be written in the following convenient form:
\begin{equation*}
    d_{z,nlm}(\uvec{p}) = \left[i\nabla_{\uvec{p}}\phi_{nlm}(\uvec{p})\right]_z = iY_{lm}(\theta_p, \phi_p)\left(\cos\theta_p\frac{\partial F_{nl}}{\partial p}-\frac{\sin\theta_p}{p}\frac{\partial F_{nl}}{\partial p}\right)
\end{equation*}
Note that here the $z$ component of a gradient in spherical coordinates was taken. The exact formula can be easily computed using the transformation between Cartesian and spherical coordinates.

The integration over $\theta_p$ is difficult and might not be possible to perform analytically.
However, for the integral over $\phi_p$, only the following needs to be solved:
\begin{align*}
    \int_{0}^{2\pi}\dd \phi_p\, Y_{l'm'}^*(\theta_p, \phi_p) Y_{lm}(\theta_p, \phi_p) &= \frac{1}{4\pi}\sqrt{(2l+1)(2l'+1)\frac{(l-m)!(l'-m')}{(l+m)!(l'+m')!}}\int_{0}^{2\pi}\dd \phi_p\, \e^{i(m-m')\phi_p}\\
    &= \frac{1}{2}\sqrt{(2l+1)(2l'+1)\frac{(l-m)!(l'-m)}{(l+m)!(l'+m)!}}
\end{align*}
This means that the magnetic quantum number is conserved during the transition between two arbitrary states.
Since the initial state is the ground state $1s$, $m$ is zero here and therefore everywhere.
This significantly simplifies the expression.






% Further since only the $z$ component of the gradient is important only the partial derivative of $\phi_{nlm}(\uvec{p})$ with respect to $p$ is needed.
% This can be also done analytically.
% \begin{align*}
%     \frac{\partial F_{nl}(p)}{\partial p} = \sum_{\iota=0}^{2l+1} \;
%         & \frac{(-1)^{\iota} \; 2^{\iota + \frac{1}{2}} \; n \; (i n)^l \; \; Z^{-l-3} \; \Gamma(2l+\iota+3)}{\iota!} \nonumber \\
%         & \times \binom{l+n}{-l+n-\iota-1} 
%         \sqrt{\frac{Z^3 \Gamma(n-l)}{\Gamma(l+n+1)}} \nonumber \\
%         & \times  \,
%         (lp^{l-1}{}_2\tilde{F}_1\left(l+\frac{\iota}{2}+2, \frac{1}{2}(2l+\iota+3); l+\frac{3}{2}; -\frac{n^2 p^2}{Z^2}\right)\\
%         &\times\frac{p^ln^2}{Z^2}{}_2\tilde{F}_1\left(l+\frac{\iota}{2}+1, \frac{1}{2}(2l+\iota+1); l+\frac{1}{2}; -\frac{n^2 p^2}{Z^2}\right))
% \end{align*}

% Now lets see what the integral over $\theta$ and $\phi$ will look like:
% \begin{equation*}
%     \frac{\partial F_{n'l'}^*}{\partial p}\frac{\partial F_{nl}}{\partial p}\int_0^{\pi}\int_0^{2\pi}\sin\theta_p \e^{ip\cos\theta_p\bar{\Delta}_z^k} Y_{l'm'}^* Y_{lm}(\theta_p, \phi_p)\left(\cos\theta_p - \frac{\sin\theta_p}{p}\right)^2\,\dd\theta_p\dd \phi_p
% \end{equation*}
% The rest is independent of $\theta$ and $\phi$.


% %%%%%%%%%%%%%%%%%%%%%%%
% \subsection*{\textcolor{red}{Transition Element}}
% Now all thats left is to differentiate \eqref{eq:phi_nlm} with respect to $\uvec{p}$. 


% new eq
% \begin{align}
%     \sum _{\iota =0}^{2 l+1} \left(-\frac{(-1)^{\iota } n \text{Ip}^{-l-3} 2^{\iota -l-1} (i n)^l \left(p^2\right)^{l/2}
%    \left(\frac{1}{\sqrt{p^2}}-\frac{\text{pz}^2}{\left(p^2\right)^{3/2}}\right) \Gamma (2 l+\iota +3) \binom{l+n}{-l+n-\iota -1} \sqrt{\frac{\text{Ip}^3 \Gamma
%    (n-l)}{\Gamma (l+n+1)}} \, _2\tilde{F}_1\left(l+\frac{\iota }{2}+2,\frac{1}{2} (2 l+\iota +3);l+\frac{3}{2};-\frac{n^2 p^2}{4 \text{Ip}^2}\right) \left(\frac{m
%    \text{pz} Y_l^m\left(\cos ^{-1}\left(\frac{\text{pz}}{\sqrt{p^2}}\right),\tan ^{-1}(\text{px},\text{py})\right)}{\sqrt{p^2}
%    \sqrt{1-\frac{\text{pz}^2}{p^2}}}+\frac{\sqrt{\Gamma (l-m+1)} \sqrt{\Gamma (l+m+2)} e^{-i \tan ^{-1}(\text{px},\text{py})} Y_l^{m+1}\left(\cos
%    ^{-1}\left(\frac{\text{pz}}{\sqrt{p^2}}\right),\tan ^{-1}(\text{px},\text{py})\right)}{\sqrt{\Gamma (l-m)} \sqrt{\Gamma (l+m+1)}}\right)}{\iota !
%    \sqrt{1-\frac{\text{pz}^2}{p^2}}}+\frac{(-1)^{\iota } l n \text{pz} \text{Ip}^{-l-3} 2^{\iota -l-1} (i n)^l \left(p^2\right)^{\frac{l}{2}-1} \Gamma (2 l+\iota +3)
%    \binom{l+n}{-l+n-\iota -1} \sqrt{\frac{\text{Ip}^3 \Gamma (n-l)}{\Gamma (l+n+1)}} \, _2\tilde{F}_1\left(l+\frac{\iota }{2}+2,\frac{1}{2} (2 l+\iota
%    +3);l+\frac{3}{2};-\frac{n^2 p^2}{4 \text{Ip}^2}\right) Y_l^m\left(\cos ^{-1}\left(\frac{\text{pz}}{\sqrt{p^2}}\right),\tan
%    ^{-1}(\text{px},\text{py})\right)}{\iota !}-\frac{(-1)^{\iota } n^3 \text{pz} \text{Ip}^{-l-5} 2^{\iota -l-3} \left(\frac{\iota }{2}+l+2\right) (\iota +2 l+3) (i
%    n)^l \left(p^2\right)^{l/2} \Gamma (2 l+\iota +3) \binom{l+n}{-l+n-\iota -1} \sqrt{\frac{\text{Ip}^3 \Gamma (n-l)}{\Gamma (l+n+1)}} \,
%    _2\tilde{F}_1\left(l+\frac{\iota }{2}+3,\frac{1}{2} (2 l+\iota +3)+1;l+\frac{5}{2};-\frac{n^2 p^2}{4 \text{Ip}^2}\right) Y_l^m\left(\cos
%    ^{-1}\left(\frac{\text{pz}}{\sqrt{p^2}}\right),\tan ^{-1}(\text{px},\text{py})\right)}{\iota !}\right)
% \end{align}
% \begin{align}
%     \sum_{\iota=0}^{2l+1} \Bigg(
%         & -\frac{(-1)^{\iota} n\, \text{Ip}^{-l-3} 2^{\iota-l-1} (i n)^l (p^2)^{l/2}
%         \left(\frac{1}{\sqrt{p^2}} - \frac{\text{pz}^2}{(p^2)^{3/2}}\right) \Gamma(2l+\iota+3)
%         \binom{l+n}{-l+n-\iota-1}
%         \sqrt{\frac{\text{Ip}^3 \Gamma(n-l)}{\Gamma(l+n+1)}} }{\iota! \sqrt{1-\frac{\text{pz}^2}{p^2}}} \nonumber \\
%         & \times {}_2\tilde{F}_1\left(l+\frac{\iota}{2}+2, \frac{1}{2}(2l+\iota+3); l+\frac{3}{2}; -\frac{n^2 p^2}{4 \text{Ip}^2}\right)
%         \left(
%             \frac{m\, \text{pz}\, Y_l^m\left(\cos^{-1}\left(\frac{\text{pz}}{\sqrt{p^2}}\right), \tan^{-1}(\text{px},\text{py})\right)}
%             {\sqrt{p^2} \sqrt{1-\frac{\text{pz}^2}{p^2}}}
%         \right. \nonumber \\
%         & \qquad \left.
%             + \frac{
%                 \sqrt{\Gamma(l-m+1)} \sqrt{\Gamma(l+m+2)} e^{-i \tan^{-1}(\text{px},\text{py})}
%                 Y_l^{m+1}\left(\cos^{-1}\left(\frac{\text{pz}}{\sqrt{p^2}}\right), \tan^{-1}(\text{px},\text{py})\right)
%             }{
%                 \sqrt{\Gamma(l-m)} \sqrt{\Gamma(l+m+1)}
%             }
%         \right) \nonumber \\
%         & + \frac{(-1)^{\iota} l n\, \text{pz}\, \text{Ip}^{-l-3} 2^{\iota-l-1} (i n)^l (p^2)^{\frac{l}{2}-1}
%             \Gamma(2l+\iota+3) \binom{l+n}{-l+n-\iota-1}
%             \sqrt{\frac{\text{Ip}^3 \Gamma(n-l)}{\Gamma(l+n+1)}} }{\iota!}
%         {}_2\tilde{F}_1\left(l+\frac{\iota}{2}+2, \frac{1}{2}(2l+\iota+3); l+\frac{3}{2}; -\frac{n^2 p^2}{4 \text{Ip}^2}\right)
%         Y_l^m\left(\cos^{-1}\left(\frac{\text{pz}}{\sqrt{p^2}}\right), \tan^{-1}(\text{px},\text{py})\right) \nonumber \\
%         & - \frac{(-1)^{\iota} n^3 \text{pz} \text{Ip}^{-l-5} 2^{\iota-l-3} \left(\frac{\iota}{2}+l+2\right) (\iota+2l+3) (i n)^l (p^2)^{l/2}
%             \Gamma(2l+\iota+3) \binom{l+n}{-l+n-\iota-1}
%             \sqrt{\frac{\text{Ip}^3 \Gamma(n-l)}{\Gamma(l+n+1)}} }{\iota!}
%         {}_2\tilde{F}_1\left(l+\frac{\iota}{2}+3, \frac{1}{2}(2l+\iota+3)+1; l+\frac{5}{2}; -\frac{n^2 p^2}{4 \text{Ip}^2}\right)
%         Y_l^m\left(\cos^{-1}\left(\frac{\text{pz}}{\sqrt{p^2}}\right), \tan^{-1}(\text{px},\text{py})\right)
%     \Bigg)
% \end{align}


% \phi_{nlm}(\uvec{p}) =\ & \sqrt{2} \left(\frac{1}{n}\right)^{-2 l-3} 
% \sqrt{\frac{(-l+n-1)!}{n^4 (l+n)!}} \\
% & \times \sum _{\iota =0}^{-l+n-1} 
% \frac{(-2)^{\iota } i^l p^l \Gamma (2 l+\iota +3)
% \binom{l+n}{-l+n-\iota -1} \, 
% {}_2\tilde{F}_1\left(l+\frac{\iota }{2}+2,\frac{1}{2} (2 l+\iota +3);l+\frac{3}{2};-n^2 p^2\right)}
% {\iota !} Y_{lm}(\theta_p,\phi_p)




% Some dipole matrix elements:\\
% First start with transforming the Schroedinger equation into momentum space %https://physics.stackexchange.com/questions/249400/schr%C3%B6dinger-equation-in-momentum-space
% Note that the prefactor does NOT depend on magentic quantumnumber m.\\
% Spherical harmonic: Instead of $Y_{lm}(\theta, \phi)$ you can write $Y_{lm}(\uvec{r})$ since $\uvec{r}=\hat{e}_x \sin(\theta)\cos(\phi)+\hat{e}_y\sin(\theta)\sin(\phi)+\hat{e}_z\cos(\theta)$\\
% The transition dipole matrix element is given by
% \begin{equation*}
%     % Analytical result for the integral
%     [ \int_0^\infty x^{\mu} e^{-\alpha x} J_\nu(\beta x) dx = \frac{\beta^\nu \Gamma(\mu+\nu+1)}{2^\nu \alpha^{\mu+\nu+1}} , {}_2F_1\left(\frac{\mu+\nu+1}{2}, \frac{\mu+\nu+2}{2}; \nu+1; -\frac{\beta^2}{\alpha^2}\right) ]
% \end{equation*}



























% \begin{equation*}
%     \vec{d}(\vec{p}) = \braket{\vec{p}|\vec{\hat{d}}|\Psi_{nlm}} = \nabla_{\vec{p}}\Psi_{nlm}(\vec{p})
% \end{equation*}
% \begin{equation*}
%     \braket{\vec{p}|100} = \frac{8 \sqrt{\pi }}{\sqrt{\frac{1}{a^3}} \left(a^2 p^2+1\right)^2}\text{if}\Re\left(\frac{1}{a}\right)>0
% \end{equation*}
% \begin{equation*}
%     \braket{\vec{p}|200} = \frac{32 \sqrt{2 \pi } \left(4 a^2 p^2-1\right)}{\sqrt{\frac{1}{a^3}} \left(4 a^2 p^2+1\right)^3}\text{if}\Re\left(\frac{1}{a}\right)>0
% \end{equation*}
% \begin{equation*}
%     \braket{\vec{p}|210} = -\frac{128 i \sqrt{2 \pi } \sqrt{\frac{1}{a^3}} a^4 p}{\left(4 a^2 p^2+1\right)^3}\text{if}\Re\left(\frac{1}{a}\right)>0
% \end{equation*}
% \begin{equation*}
%     \braket{\vec{p}|300} = \frac{72 \sqrt{3 \pi } \left(81 a^4 p^4-30 a^2 p^2+1\right)}{\sqrt{\frac{1}{a^3}} \left(9 a^2p^2+1\right)^4}\text{ if }\Re\left(\frac{1}{a}\right)>0
% \end{equation*}
% \begin{equation*}
%     \braket{\vec{p}|310} = -\frac{864 i \sqrt{2 \pi } \sqrt{\frac{1}{a^3}} a^4 p \left(9 a^2 p^2-1\right)}{\left(9 a^2 p^2+1\right)^4}
% \end{equation*}
% \begin{equation*}
%     \braket{\vec{p}|320} = -\frac{1728 \sqrt{6 \pi } \sqrt{\frac{1}{a^3}} a^5 p^2}{\left(9 a^2 p^2+1\right)^4}
% \end{equation*}
% \begin{equation*}
%     \braket{\vec{p}|400} = \frac{256 \sqrt{\pi } \left(4096 a^6 p^6-1792 a^4 p^4+112 a^2 p^2-1\right)}{\sqrt{\frac{1}{a^3}}\left(16 a^2 p^2+1\right)^5}\text{ if }\Re\left(\frac{1}{a}\right)>0
% \end{equation*}
% \begin{equation*}
%     \braket{\vec{p}|410} = -\frac{2048 i \sqrt{\frac{\pi }{5}} \left(\frac{1}{a^3}\right)^{3/2} a^7 p \left(32 a^2 p^2 \left(40 a^2p^2-7\right)+5\right)}{\left(16 a^2 p^2+1\right)^5}\text{ if }\Re\left(\frac{1}{a}\right)>0
% \end{equation*}
% \begin{equation*}
%     \braket{\vec{p}|420} = -\frac{32768 \sqrt{\pi } \sqrt{\frac{1}{a^3}} a^5 p^2 \left(16 a^2 p^2-1\right)}{\left(16 a^2 p^2+1\right)^5}
% \end{equation*}
% \begin{equation*}
%     \braket{\vec{p}|430} = \frac{262144 i \sqrt{\frac{\pi }{5}} \sqrt{\frac{1}{a^3}} a^6 p^3}{\left(16 a^2 p^2+1\right)^5}
% \end{equation*}
% \begin{equation*}
%     \braket{\vec{p}|900} = \frac{1944 \sqrt{\pi } \left(27 a^2 p^2-1\right) \left(243 a^2 p^2-1\right) \left(243 a^2 p^2 \left(6561
%     a^4 p^4-729 a^2 p^2+11\right)-1\right) \left(729 \left(243 a^6 p^6-99 a^4 p^4+a^2
%     p^2\right)-1\right)}{\sqrt{\frac{1}{a^3}} \left(81 a^2 p^2+1\right)^{10}}\text{ if
%     }\Re\left(\frac{1}{a}\right)>0
% \end{equation*}
% \begin{equation*}
%     \braket{\vec{p}|510} = -\frac{4000 i \sqrt{10 \pi } \sqrt{\frac{1}{a^3}} a^4 p \left(15625 a^6 p^6-3375 a^4 p^4+135 a^2
%     p^2-1\right)}{\left(25 a^2 p^2+1\right)^6}\text{ if }\Re\left(\frac{1}{a}\right)>0
% \end{equation*}
\newpage

\end{document}
